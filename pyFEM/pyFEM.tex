\chapter{pyFEM}
\label{cha:pyFEM}

En la figura \ref{fig:space-frame} se presenta un elemento \emph{i} tipo \emph{pórtico} con sus nodos \emph{j} y \emph{k} empotrados. El sistema de coordenadas locales del elemento tiene como origen el nodo \emph{j}. El eje $ x $ coincide con el eje centroidal del elemento y es positivo en el sentido del nodo \emph{j} al nodo \emph{k}. Los ejes $ y $ y $ z $ son los ejes principales del elemento de manera que los planos $ xy $ y $ zx $ son los planos principales de flexión. Se asume que el centro de cortante y el centroide del elemento coinciden de tal forma que la flexión y la torsión se presentan una independiente de la otra. Los grados de libertad se numeran del 1 al 12, empezando por las translaciones y las rotaciones del nodo \emph{j}, tomados en orden $ x $, $ y $, $ z $ respectivamente.

\begin{figure}[ht]
  \centering
  \begin{tikzpicture}[coords]
    % points
    \dpoint{a}{0}{0}{0};
    \dpoint{b}{4}{0}{0};

    \dpoint{alabel}{0}{0}{0.25};
    \dpoint{blabel}{4}{0}{0.25};

    \dpoint{beamlabel}{2}{0}{0.25};

    \dpoint{jux}{-1}{0}{0};
    \dpoint{juy}{0}{-1.5}{0.25};   
    \dpoint{juz}{0}{0}{-1.25};

    \dpoint{jrx}{-2}{0}{0};
    \dpoint{jry}{0}{-2.75}{0.25};   
    \dpoint{jrz}{0}{0}{-2.5};

    \dpoint{kux}{4.5}{0}{0};
    \dpoint{kuy}{4}{-1.5}{0.25};   
    \dpoint{kuz}{4}{0}{-1.25};

    \dpoint{krx}{5.25}{0}{0};
    \dpoint{kry}{4}{-2.75}{0.35};   
    \dpoint{krz}{4}{0}{-2.5};   
   
    % beams
    \dbeam{1}{a}{b};

    \dnotation{6}{beamlabel}{\emph{i}};

    % supports
    \dsupport{2}{a}[yz];
    \dsupport{2}{b}[yz];

    \dnotation{1}{alabel}{\emph{j}};
    \dnotation{1}{blabel}{\emph{k}};
    % restrains
    % j
    \dload{1}{a}[270][0][1];
    \dload{1}{a}[270][90][1][0.5];   
    \dload{1}{a}[180][180][1][0.5];

    \dload{3}{a}[270][0][1][1.25];
    \dload{3}{a}[270][90][1][1.75];   
    \dload{3}{a}[180][180][1][1.75];

    \dnotation{1}{jux}{1};
    \dnotation{1}{juy}{2};
    \dnotation{1}{juz}{3};

    \dnotation{1}{jrx}{4};
    \dnotation{1}{jry}{5};
    \dnotation{1}{jrz}{6};   
    
    % k
    \dload{2}{b}[90][0][1];
    \dload{1}{b}[270][90][1][0.5];   
    \dload{1}{b}[180][180][1][0.5];

    \dload{4}{b}[90][0][1][1.25];
    \dload{3}{b}[270][90][1][1.75];   
    \dload{3}{b}[180][180][1][1.75];

    \dnotation{1}{kux}{7};
    \dnotation{1}{kuy}{8};
    \dnotation{1}{kuz}{9};

    \dnotation{1}{krx}{10};
    \dnotation{1}{kry}{11};
    \dnotation{1}{krz}{12};
    
    % axes
    \dscaling{3}{1};
    \daxis{1}{0, 0, 0};
  \end{tikzpicture}
  \caption{Elemento tipo pórtico en coordenadas locales.}
  \label{fig:space-frame}
\end{figure}

Según \cite{weaver1990matrixanalysis}, (\ref{eq:matriz-rigidez-elemento-portico}) es la matrix de rigidez del elemento tipo pórtico en coordenadas locales, donde $ E $ es el módulo de elasticidad del material, $ G $ es el módulo de elasticidad a cortante del material, $ L $ es la longitud del elemento y $ A_x $, $ I_x $, $ I_y $ y $ I_z $ son el área, la constante de torsión y los momentos principales de inercia con respecto a los ejes $ y $ y $ z $ de la sección transversal.\\

\begin{equation}
  \begin{bNiceArray}{CCCCCCCCCCCC}[small,
    first-row,
    first-col,
    code-for-first-row = \mathbf{\arabic{jCol}},
    code-for-first-col = \mathbf{\arabic{iRow}}
    ]
    & & & & & & & & & & & & \\
    & \frac{EA_x}{L} & 0 & 0 & 0 & 0 & 0 & -\frac{EA_x}{L} & 0 & 0 & 0 & 0 & 0 \\
    & & \frac{12EI_z}{L^3} & 0 & 0 & 0 & \frac{6EI_z}{L^2} & 0 & -\frac{12EI_z}{L^3} & 0 & 0 & 0 & \frac{6EI_z}{L^2} \\
    & & & \frac{12EI_y}{L^3} & 0 & -\frac{6EI_y}{L^2} & 0 & 0 & 0 & -\frac{12EI_y}{L^3} & 0 & -\frac{6EI_y}{L^2} & 0 \\
    & & & & \frac{GI_x}{L} & 0 & 0 & 0 & 0 & 0 & -\frac{GI_x}{L} & 0 & 0 \\
    & & & & & \frac{4EI_y}{L} & 0 & 0 & 0 & \frac{6EI_y}{L^2} & 0 & \frac{2EI_y}{L} & 0 \\
    & & & & & & \frac{4EI_z}{L} & 0 & -\frac{6EI_z}{L^2} & 0 & 0 & 0 & \frac{2EI_z}{L} \\
    & & & & & & & \frac{EA_x}{L} & 0 & 0 & 0 & 0 & 0 \\
    & & & & & & & & \frac{12EI_z}{L^3} & 0 & 0 & 0 & -\frac{6EI_z}{L^2} \\
    & & & & & & & & & \frac{12EI_y}{L^3} & 0 & \frac{6EI_y}{L^2} & 0 \\
    & & & & & & & & & & \frac{GI_x}{L} & 0 & 0 \\
    & & & & & & & & & & & \frac{4EI_y}{L} & 0 \\
    & \emph{sim.} & & & & & & & & & & & \frac{4EI_z}{L}
    \label{eq:matriz-rigidez-elemento-portico}
  \end{bNiceArray}
\end{equation}  

Según \cite{dunn20023d}, la orientación en tres dimensiones se puede describir usando \emph{cuaterniones} que se definen como
\begin{equation}
  q = w + x\mathbf{i} + y\mathbf{j} + z\mathbf{k}
\end{equation}

Los cuaterniones extienden el sistema de números complejos al tener tres números imaginarios los cuales se relacionan de la siguiente manera
\begin{gather}
  i^2 = j^2 = k^2 = -1 \\
  ij = k, ji = -k \nonumber \\
  jk = i, kj = -i \nonumber \\
  ki = j, ik = -j \nonumber
\end{gather}