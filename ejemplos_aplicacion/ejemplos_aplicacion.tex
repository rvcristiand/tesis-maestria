\chapter{Ejemplos de aplicación}
\label{chap:ejemplos_aplicacion}

Con el fin de presentar el funcionamiento de los programas desarrollados, en este capítulo se presentan diferentes modelos de estructuras tridimensionales de elementos rectos sometidos a cargas estáticas, los cuales se resuelven usando la librería \textit{pyFEM} o el programa de computador \textit{StressUN}. \\

\section{Funcionamiento de la librería \textit{pyFEM}} \label{sec:libreriaPyFem}
Se presentan varios ejemplos provenientes de los libros de la referencia bibliográfica, los cuales se resuelven usando el programa \textit{StressUN} o la librería \textit{pyFEM}, con el objetivo de comparar la respuesta obtenida con la presentada en dichos textos.

\subsection{Cercha plana}

En el libro \textit{Microcomputadores en Ingeniería Estructural}, del ingeniero \textit{Jairo Uribe Escamilla} (\cite{escamilla1995microcomputadores}), se presenta el ejemplo y la solución del modelo de una cercha simple plana estáticamente determinada (una cercha simple plana es una estructura compleja generada a partir de una estructura triangular base \cite{beer1997mecanica}). El ejercicio consiste en encontrar los desplazamientos de los nudos, las fuerzas de las reacciones y las fuerzas de cada elemento de la cercha mostrada en la figura \ref{fig:primer_punto}. La solución presenta las matrices de rigidez de cada uno de los elementos de la cercha, la matriz de rigidez de toda la estructura, la solución de los desplazamientos de los nodos, las reacciones de los apoyos y las fuerzas internas de cada elemento.\\

A continuación se presenta la solución de dicho ejemplo, el cual se resuelve usando la librería \texttt{pyFEM}. \\

\textit{\textbf{Ejemplo 7.1 -} Resuelva completamente la cercha mostrada por el método matricial de los desplazamientos. El material es acero estructural con $ E = 2040\; t / cm^2 $. Las áreas están dadas entre paréntesis en $ cm^2 $.}\\

\textit{\textbf{Solución -}} Para solucionar el ejemplo anterior usando la librería \texttt{pyFEM}, se ingresa la información necesaria para describir completamente el modelo de la estructura mediante instrucciones en el lenguaje de programación \texttt{python}, las cuales se almacenan en un archivo de texto plano. Dicha información consiste en:
\begin{inparaenum} [$ (1) $]
    \item los materiales,
    \item las secciones transversales,
    \item los nodos,
    \item los elementos tipo cercha,
    \item los apoyos y
    \item las cargas.
\end{inparaenum} \\

\begin{figure}[t]
    \centering
    
    \begin{tikzpicture}
        % Scaling
        % \scaling{.25};
        
        % Points
        \point{1}{0}{0};
        \point{2}{8}{0};
        \point{3}{4}{3};
        \point{4}{4}{0};
        
        \point{p}{4}{-1.3};
        \point{q}{5.05}{3.7875};
        
        \point{1-3}{2}{1.5};
        \point{1-4}{2}{0};
        \point{3-2}{6}{1.5};
        \point{4-2}{6}{0};
        \point{4-3}{4}{1.5};
        
        % Beams
        \beam{2}{1}{3}[0][0];
        \beam{2}{1}{4}[1][1];
        \beam{2}{3}{2}[1][1];
        \beam{2}{4}{2}[1][1];
        \beam{2}{4}{3}[1][1];
        
        % Supports
        \support{1}{1};
        \support{2}{2};
        
        % Joints
        \hinge{1}{1};
        \hinge{1}{2};
        \hinge{1}{3};
        \hinge{1}{4};
        
        % Loads
        \load{1}{q}[216.87];
        \load{1}{p}[90];
        
        % notation
        \notation{1}{1}{1}[above left];
        \notation{1}{2}{2}[above right];
        \notation{1}{3}{3}[above left];
        \notation{1}{4}{4}[below left];
        
        \notation{1}{3}{$ \SI{5}{\tonne} $}[above right=10mm];
        \notation{1}{4}{$ \SI{20}{\tonne} $}[below right=5mm and 0mm];
        
        \notation{5}{1}{3}[$ (100) $ ][.5][above][0.5];
        \notation{5}{1}{4}[$ (40) $ ][.5][above][0.5];
        \notation{5}{3}{2}[$ (150) $ ][.5][above][0.5];
        \notation{5}{4}{2}[$ (40) $ ][.5][above][0.5];
        \notation{5}{4}{3}[$ (30) $ ][.5][right][1];
        
        
        % Dimensions
        \dimensioning{1}{1}{4}{-1.5}[$ \SI{4}{m} $];
        \dimensioning{1}{4}{2}{-1.5}[$ \SI{4}{m} $];
        \dimensioning{2}{2}{3}{9}[$ \SI{3}{m} $];
    \end{tikzpicture}
    
    \caption{Cercha simple plana del \textit{Ejemplo 7.1} de \cite{escamilla1995microcomputadores}.}
    \label{fig:primer_punto}
\end{figure}

En el algoritmo \ref{alg:cercha_plana} se presentan las instrucciones que debe recibir \texttt{pyFEM} para solucionar el modelo de la estructura. Las instrucciones consisten en crear un nuevo objeto tipo \texttt{Structure}, al cual se le ha dado el nombre de \textit{structure} y, seguidamente, se agregan los materiales, las secciones, los nodos, los elementos tipo cercha, los apoyos, los patrones de carga y las cargas en los nodos. Una vez los datos del modelo de la estructura han sido ingresados, se puede solucionar el modelo mediante la instrucción \texttt{structure.solve()}. \\

\begin{lstlisting}[language=Python,caption=Ingreso de los datos del modelo de la estructura a \textit{pyFEM}.,label=alg:cercha_plana, frame=single]
# create structure
structure = Structure()

# add material
structure.materials.add("acero", 2040e4)

# add sections
structure.sections.add("section1", "acero", 30e-4)
structure.sections.add("section2", "acero", 40e-4)
structure.sections.add("section3", "acero", 100e-4)
structure.sections.add("section4", "acero", 150e-4)

# add nodes
structure.nodes.add('1', 0, 0, 0)
structure.nodes.add('2', 8, 0, 0)
structure.nodes.add('3', 4, 3, 0)
structure.nodes.add('4', 4, 0, 0)

# add trusses
structure.trusses.add('1-3', '1', '3', "section3")
structure.trusses.add('1-4', '1', '4', "section2")
structure.trusses.add('3-2', '3', '2', "section4")
structure.trusses.add('4-2', '4', '2', "section2")
structure.trusses.add('4-3', '4', '3', "section1")

# add support
structure.supports.add('1', True, True, True)
structure.supports.add('2', False, True, True)
structure.supports.add('3', False, False, True)
structure.supports.add('4', False, False, True)

# add load pattern
structure.load_patterns.add("point loads")

# add point loads
structure.load_patterns["point loads"].point_loads.add('4', 0, -20, 0)
structure.load_patterns["point loads"].point_loads.add('3', 5*0.8, 5*0.6, 0)

# solve the structure
structure().solve()
\end{lstlisting}

Cuando se ejecuta la instrucción \texttt{structure.solve()}, \texttt{pyFEM} comienza a solucionar el modelo de la estructura con base en la información ingresada. Los pasos que se efectúan para solucionar la estructura consisten en: 
\begin{inparaenum}[$ (1) $]
    \item asignar los grados de libertad de los nodos, 
    \item ensamblar la matriz de rigidez del modelo de la estructura, 
    \item imponer las condiciones de apoyo en la matriz de rigidez del modelo, 
    \item ensamblar el vector de fuerzas en los nodos para cada uno de los patrones de carga, 
    \item imponer las condiciones de apoyo en el vector de fuerzas en los nodos para cada caso de carga, 
    \item encontrar los desplazamientos de los nodos para cada patrón de carga, 
    \item encontrar las reacciones en los apoyos para cada patrón de carga y
    \item guardar la solución en los nodos y en los apoyos para cada patrón de carga.
\end{inparaenum} \\

A continuación se presenta el resultado de cada uno de los pasos realizados por \texttt{pyFEM} para solucionar la cercha del ejemplo 7.1. \\

\subsubsection{Grados de libertad de los nodos}
Para realizar el ensamblaje de la matriz de rigidez del modelo de la estructura y del vector de fuerzas de los nodos, se asignan números a los grados de libertad de los nodos de la estructura en el orden en que fueron ingresados. \\

Con base en ésto y en el algoritmo \ref{alg:cercha_plana}, el nodo \textit{'1'} se le han asignado los grados de libertad \textit{0, 1} y \textit{2}, al nodo \textit{'2'} los grados de libertad \textit{3, 4} y \textit{5}, y así sucesivamente.

\subsubsection{Matrices de rígidez}
Una vez se establecen los grados de libertad de los nodos del modelo de la estructura, se ensambla la matriz de rigidez del modelo de la estructura. Este proceso consiste en calcular una a una las matrices de rigidez de los elementos ensamblandolas en la matriz de rigidez del modelo, la cual, inicialmente, es una matriz de ceros. \\

Aunque no se almacenan las matrices de rigidez de cada uno de los elementos del modelo de la estructura, el usuario puede consultarlas. En la tabla \ref{tab:k_1_3} se presenta la matriz de rigidez en coordenadas globales del elemento \textit{1-3}, con sus respectivos grados de libertad, la cual se obtiene mediante la instrucción \textit{\textit{structure.trusses['1-3'].get\_global\_stiff\_matrix()}}. \\

\begin{table}[ht]
    \centering
    \begin{blockarray}{ccccccc}
        & \textit{0} & \textit{1} & \textit{2} & \textit{6} & \textit{7} & \textit{8} \\
        \begin{block}{c[cccccc]}
            \textit{0} & 26112 & 19584 & 0 & -261112 & -19584 & 0 \\
            \textit{1} & 19584 & 14688 & 0 & -19584 & -14688 & 0 \\
            \textit{2} & 0 & 0 & 0 & 0 & 0 & 0 \\
            \textit{3} & -26112 & -19584 & 0 & 26112 & 19584 & 0 \\
            \textit{4} & -19584 & -14688 & 0 & 19584 & 14688 & 0 \\
            \textit{5} & 0 & 0 & 0 & 0 & 0 & 0 \\
        \end{block}
    \end{blockarray} \si[per-mode=symbol]{\tonne\per\meter}
    \caption{Matriz de rigidez en coordenadas globales del elemento \textit{1-3}.}
    \label{tab:k_1_3}
\end{table}

En las tablas \ref{tab:k_1_4} a la \ref{tab:k_4_3} se presentan las matrices de rigidez en coordenadas globales de los demás elementos de la estructura, las cuales se obtienen con instrucciones similares a la anterior. \\

\begin{table}[h]
    \centering
    \begin{blockarray}{ccccccc}
        & \textit{0} & \textit{1} & \textit{2} & \textit{9} & \textit{10} & \textit{11} \\
        \begin{block}{c[cccccc]}
            \textit{0} & 20400 & 0 & 0 & -20400 & 0 & 0 \\
            \textit{1} & 0 & 0 & 0 & 0 & 0 & 0 \\
            \textit{2} & 0 & 0 & 0 & 0 & 0 & 0 \\
            \textit{9} & -20400 & 0 & 0 & 20400 & 0 & 0 \\
            \textit{10} & 0 & 0 & 0 & 0 & 0 & 0 \\
            \textit{11} & 0 & 0 & 0 & 0 & 0 & 0 \\
        \end{block} 
    \end{blockarray} \si[per-mode=symbol]{\tonne\per\meter}
    \caption{Matriz de rigidez en coordenadas globales del elemento \textit{1-4}.}
    \label{tab:k_1_4}
\end{table}

\begin{table}[ht]
    \centering
    \begin{blockarray}{ccccccc}
        & \textit{6} & \textit{7} & \textit{8} & \textit{3} & \textit{4} & \textit{5} \\
        \begin{block}{c[cccccc]}
            \textit{6} & 39168 & -29376 & 0 & -39168 & 29376 & 0 \\
            \textit{7} & -29376 & 22032 & 0 & 29376 & -22032 & 0 \\
            \textit{8} & 0 & 0 & 0 & 0 & 0 & 0 \\
            \textit{3} & -39168 & 29376 & 0 & 39168 & -29376 & 0 \\
            \textit{4} & 29376 & -22032 & 0 & -29376 & 22032 & 0 \\
            \textit{5} & 0 & 0 & 0 & 0 & 0 & 0 \\
        \end{block}
    \end{blockarray} \si[per-mode=symbol]{\tonne\per\meter}
    \caption{Matriz de rigidez en coordenadas globales del elemento \textit{3-2}.}
    \label{tab:k_3_2}
\end{table}

\begin{table}[ht]
    \centering
    \begin{blockarray}{ccccccc}
        & \textit{9} & \textit{10} & \textit{11} & \textit{3} & \textit{4} & \textit{5} \\
        \begin{block}{c[cccccc]}
            \textit{9} & 20400 & 0 & 0 & -20400 & 0 & 0 \\
            \textit{10} & 0 & 0 & 0 & 0 & 0 & 0 \\
            \textit{11} & 0 & 0 & 0 & 0 & 0 & 0 \\
            \textit{3} & -20400 & 0 & 0 & 20400 & 0 & 0 \\
            \textit{4} & 0 & 0 & 0 & 0 & 0 & 0 \\
            \textit{5} & 0 & 0 & 0 & 0 & 0 & 0 \\
        \end{block}
    \end{blockarray} \si[per-mode=symbol]{\tonne\per\meter}
    \caption{Matriz de rigidez en coordenadas globales del elemento \textit{4-2}.}
    \label{tab:k_4_2}
\end{table}

\begin{table}[H]
    \centering
    \begin{blockarray}{ccccccc}
        & \textit{9} & \textit{10} & \textit{11} & \textit{6} & \textit{7} & \textit{8} \\
        \begin{block}{c[cccccc]}
            \textit{9} & 0 & 0 & 0 & 0 & 0 & 0 \\
            \textit{10} & 0 & 20400 & 0 & 0 & -20400 & 0 \\
            \textit{11} & 0 & 0 & 0 & 0 & 0 & 0 \\
            \textit{6} & 0 & 0 & 0 & 0 & 0 & 0 \\
            \textit{7} & 0 & -20400 & 0 & 0 & 20400 & 0 \\
            \textit{8} & 0 & 0 & 0 & 0 & 0 & 0 \\
        \end{block}
    \end{blockarray} \si[per-mode=symbol]{\tonne\per\meter}
    \caption{Matriz de rigidez en coordenadas globales del elemento \textit{4-3}.}
    \label{tab:k_4_3}
\end{table}

Al comparar los resultados obtenidos con las matrices de rigidez de cada uno de los elementos presentados en \cite{escamilla1995microcomputadores} se encuentra que se trata de los mismos valores. \\

Así mismo como el usuario puede indagar por la matriz de rigidez en coordenadas globales de cada elemento, puede hacerlo para el modelo de la estructura, mediante la instrucción \textit{\textit{structure.get\_k()}}. Al hacerlo, obtiene los valores presentados en la tabla \ref{tab:k_global}.\\

\begin{table}[ht]
    \centering
    \begin{blockarray}{ccccccccccccc}
        & \textit{0} & \textit{1} & \textit{2} & \textit{3} & \textit{4} & \textit{5} & \textit{6} & \textit{7} & \textit{8} & \textit{9} & \textit{10} & \textit{11} \\
        \begin{block}{c[cccccccccccc]}
            \textit{0} & 46512 & 19584 & 0 & 0 & 0 & 0 & -26112 & -19584 & 0 & -20400 & 0 & 0 \\
            \textit{1} & 19584 & 14688 & 0 & 0 & 0 & 0 & -19584 & -14688 & 0 & 0 & 0 & 0 \\
            \textit{2} & 0 & 0 & 0 & 0 & 0 & 0 & 0 & 0 & 0 & 0 & 0 & 0 \\
            \textit{3} & 0 & 0 & 0 & 59568 & -29376 & 0 & -39168 & 29376 & 0 & -20400 & 0 & 0 \\
            \textit{4} & 0 & 0 & 0 & -29376 & 22032 & 0 & 29376 & -22032 & 0 & 0 & 0 & 0 \\
            \textit{5} & 0 & 0 & 0 & 0 & 0 & 0 & 0 & 0 & 0 & 0 & 0 & 0 \\
            \textit{6} & -26112 & -19584 & 0 & -39168 & 29376 & 0 & 65280 & -9792 & 0 & 0 & 0 & 0 \\
            \textit{7} & -19584 & -14688 & 0 & 29376 & -22032 & 0 & -9792 & 57120 & 0 & 0 & -20400 & 0 \\
            \textit{8} & 0 & 0 & 0 & 0 & 0 & 0 & 0 & 0 & 0 & 0 & 0 & 0 \\
            \textit{9} & -20400 & 0 & 0 & -20400 & 0 & 0 & 0 & 0 & 0 & 40800 & 0 & 0 \\
            \textit{10} & 0 & 0 & 0 & 0 & 0 & 0 & 0 & -20400 & 0 & 0 & 20400 & 0 \\
            \textit{11} & 0 & 0 & 0 & 0 & 0 & 0 & 0 & 0 & 0 & 0 & 0 & 0 \\
        \end{block}
    \end{blockarray} \si[per-mode=symbol]{\tonne\per\meter}
    \caption{Matriz de rigidez en coordenadas globales del modelo de la estructura.}
    \label{tab:k_global}
\end{table}

Una vez más, al comparar los resultados obtenidos con la matriz de rigidez del modelo de la estructura presentado en \cite{escamilla1995microcomputadores} se encuentra que se trata de los mismos valores. Sin embargo, se debe reordenar una de las dos matrices para encontrar los valores en las mismas posiciones de la matriz. \\

Obtenida la matriz de rigidez de la estructura, se procede a imponer las condiciones de apoyo del modelo de la estructura. Esto se realiza modificando la estructura, tal como se mostró en el capítulo \ref{chap:metodologia}. En la tabla \ref{tab:k_global_support_applied} se presenta dicho resultado. \\

\begin{table}[ht]
    \centering
    \begin{blockarray}{ccccccccccccc}
        & \textit{0} & \textit{1} & \textit{2} & \textit{3} & \textit{4} & \textit{5} & \textit{6} & \textit{7} & \textit{8} & \textit{9} & \textit{10} & \textit{11} \\
        \begin{block}{c[cccccccccccc]}
            \textit{0} & 1 & 0 & 0 & 0 & 0 & 0 & 0 & 0 & 0 & 0 & 0 & 0 \\
            \textit{1} & 0 & 1 & 0 & 0 & 0 & 0 & 0 & 0 & 0 & 0 & 0 & 0 \\
            \textit{2} & 0 & 0 & 1 & 0 & 0 & 0 & 0 & 0 & 0 & 0 & 0 & 0 \\
            \textit{3} & 0 & 0 & 0 & 59568 & 0 & 0 & -39168 & 29376 & 0 & -20400 & 0 & 0 \\
            \textit{4} & 0 & 0 & 0 & 0 & 1 & 0 & 0 & 0 & 0 & 0 & 0 & 0 \\
            \textit{5} & 0 & 0 & 0 & 0 & 0 & 1 & 0 & 0 & 0 & 0 & 0 & 0 \\
            \textit{6} & 0 & 0 & 0 & -39168 & 0 & 0 & 65280 & -9792 & 0 & 0 & 0 & 0 \\
            \textit{7} & 0 & 0 & 0 & 29376 & 0 & 0 & -9792 & 57120 & 0 & 0 & -20400 & 0 \\
            \textit{8} & 0 & 0 & 0 & 0 & 0 & 0 & 0 & 0 & 1 & 0 & 0 & 0 \\
            \textit{9} & 0 & 0 & 0 & -20400 & 0 & 0 & 0 & 0 & 0 & 40800 & 0 & 0 \\
            \textit{10} & 0 & 0 & 0 & 0 & 0 & 0 & 0 & -20400 & 0 & 0 & 20400 & 0 \\
            \textit{11} & 0 & 0 & 0 & 0 & 0 & 0 & 0 & 0 & 0 & 0 & 0 & 1 \\
        \end{block}
    \end{blockarray} \si[per-mode=symbol]{\tonne\per\meter}
    \caption{Matriz de rigidez en coordenadas globales del modelo de la estructura con las condiciones de apoyo impuestas.}
    \label{tab:k_global_support_applied}
\end{table}

\subsubsection{Vector de fuerzas}

Así como se deben encontrar las matrices de rigidez de cada uno de los elementos del modelo de la estructura para posteriormente \textit{ensamblarlas}, se deben encontrar las acciones en los nodos de los elementos para cada patrón de carga. \\

El usuario puede indagar por dicho vector, para el patrón de carga \textit{point loads} mediante la instrucción \texttt{structure.load\_patterns['point loads'].get\_f()}. Con dicha instrucción, el usuario obtiene los datos que se muestran en la tabla \ref{tab:f_global}. \\

Obtenido el vector de fuerzas para dicho patrón de carga, se imponen las condiciones de apoyo del modelo de la estructura. Debido a que los desplazamiento en los apoyos son iguales a cero, el vector de fuerzas en los nodos no varia. \\

\begin{table}[ht]
    \centering
    \begin{blockarray}{cccccccccccc}
        \textit{0} & \textit{1} & \textit{2} & \textit{3} & \textit{4} & \textit{5} & \textit{6} & \textit{7} & \textit{8} & \textit{9} & \textit{10} & \textit{11} \\
        \begin{block}{\{cccccccccccc\}}
            0 & 0 & 0 & 0 & 0 & 0 & 4 & 3 & 0 & 0 & -20 & 0 \\
        \end{block}
    \end{blockarray} $ ^{t} $ \si{\tonne}
    \caption{Vector de fuerzas de los nodos del modelo de la estructura para el patrón de carga \textit{point loads}.}
    \label{tab:f_global}
\end{table}

\subsubsection{Vector de desplazamientos}
Al imponerse las condiciones de apoyo a la matriz de rigidez del modelo de la estructura, se determinan los desplazamientos de los nodos de la estructura para cada uno de los patrones de carga, donde se ha calculado el vector de fueras en los nodos de cada patrón de cargas y se le han impuesto las condiciones de apoyo. \\

En la tabla \ref{tab:u_point_loads} se presenta el vector de desplazamientos en los nodos del modelo de la estructura para el patrón de carga \textit{point loads}, los cuales son iguales a los presentados en \cite{escamilla1995microcomputadores}.

\begin{table}[ht]
    \centering
    \begin{blockarray}{cccccccccccc}
        \textit{0} & \textit{1} & \textit{2} & \textit{3} & \textit{4} & \textit{5} & \textit{6} & \textit{7} & \textit{8} & \textit{9} & \textit{10} & \textit{11} \\
        \begin{block}{\{cccccccccccc\}}
            0 & 0 & 0 & \num{1.307} & 0 & 0 & \num{0.645} & \num{-1.337} & 0 & \num{0.654} & \num{-2.317} & 0 \\
        \end{block} 
    \end{blockarray} $ ^{t} $ \SI{1e-3}{\meter} 
    \caption{Vector de desplazamientos de los nodos del modelo de la estructura para el patrón de carga \textit{point loads}.}
    \label{tab:u_point_loads}
\end{table}

\subsubsection{Vector de reacciones}
Una vez se encuentran los desplazamientos en los nodos del modelo de la estructura, para cada uno de los patrones de carga, se procede a encontrar las fuerzas en los nodos producto de dichos desplazamientos. \\

En la tabla \ref{tab:f_point_loads_solution} se presenta el vector de fuerzas en los nodos del modelo de la estructura para el patrón de carga \textit{point loads}, los cuales son iguales a los presentados en \cite{escamilla1995microcomputadores}.

\begin{table}[ht]
    \centering
    \begin{blockarray}{cccccccccccc}
        \textit{0} & \textit{1} & \textit{2} & \textit{3} & \textit{4} & \textit{5} & \textit{6} & \textit{7} & \textit{8} & \textit{9} & \textit{10} & \textit{11} \\
        \begin{block}{\{cccccccccccc\}}
            -4 & 7 & 0 & 0 & 10 & 0 & 4 & 3 & 0 & 0 & -20 & 0 \\
        \end{block} 
    \end{blockarray} $ ^{t} $ \si{\tonne} 
    \caption{Vector de fuerzas de los nodos del modelo de la estructura para el patrón de carga \textit{point loads}.}
    \label{tab:f_point_loads_solution}
\end{table}

\subsubsection{Procesamiento de los resultados}

Hasta aquí, se ha solucionado el modelo de la estructura sometida al patrón de carga \textit{point loads}. Con los resultados almacenados, se pueden determinar otros resultados que son interesantes. Para el caso concreto del modelo objeto de estudio en esta sección, se puede querer determinar las fuerzas internas de los elementos o las defomaciones en los mismos. \\

A modo de ejemplo, en la tabla \ref{tab:internal_forces} se presenta las fuerzas internas de cada uno de los elementos del modelo de la estructura sometidos al patrón de carga \textit{point loads}, los cuales son prácticamente iguales a los presentados en \cite{escamilla1995microcomputadores}.
\begin{table}[h]
    \centering
    \begin{tabular}{|c|c|}
        \hline
        Elemento & Fuerza [\si{\tonne}] \\
        \hline
        1-3 & -11.667 \\
        \hline
        1-4 & 13.333 \\
        \hline
        3-2 & -16.667 \\
        \hline
        4-2 & 13.333 \\
        \hline
        4-3 & 20 \\
        \hline
    \end{tabular}
    \caption{Fuerzas internas de los elementos del modelo de la estructura para el patrón de carga \textit{point loads}.}
    \label{tab:internal_forces}
\end{table}

\subsection{Pórtico tridimensional}

En el libro \textit{Microcomputadores en Ingeniería Estructural}, del ingeniero \textit{Jairo Uribe Escamilla} (\cite{escamilla1995microcomputadores}), se presenta el ejemplo y la solución del modelo de un pórtico tridimensional. A continuación se presenta la solución de dicho ejemplo, el cual se resuelve usando la librería \texttt{pyFEM}. \\

\textit{\textbf{Ejemplo 7.6 -} Resuelva matricialmente el pórtico de la figura.} \\

\begin{figure}[h]
    \setcoords{0}{90}[.5][.5][0.5][225]
    \centering
    \begin{tikzpicture}[coords]
        % Scaling
        \dscaling{1}{3};
        \dscaling{2}{3};
        \dscaling{3}{1};
        \dscaling{4}{1.5};
        \dscaling{5}{2};
        
        % Points
        \dpoint{o}{0}{0}{0};
        
        \dpoint{1}{0}{3}{3};
        \dpoint{2}{5}{3}{3};
        \dpoint{3}{0}{0}{3};
        \dpoint{4}{0}{3}{0};
        
        \dpoint{23}{5}{0}{3};
        \dpoint{24}{5}{3}{0};
        \dpoint{234}{5}{0}{0};
        
        \dpoint{a}{2.5}{3}{3};
        \dpoint{b}{0}{3}{1.5};
        \dpoint{c}{0}{1.5}{3};
        
        % Beams
        \dbeam{1}{1}{2};
        \dbeam{1}{3}{1};
        \dbeam{1}{4}{1};
        
        \dbeam{3}{3}{o};
        \dbeam{3}{4}{o};
        \dbeam{3}{2}{23};
        \dbeam{3}{3}{23};
        \dbeam{3}{2}{24};
        \dbeam{3}{4}{24};
        \dbeam{3}{o}{234};
        \dbeam{3}{24}{234};
        \dbeam{3}{23}{234};
        
        % Global coordinate system
        \daxis{1}{o};
        
        % Supports
        \dsupport{2}{2}[yz];
        \dsupport{2}{3}[xz];
        \dsupport{2}{4}[xy];
        
        % Loads
        \dlineload{1}{xy}{1}{2}[.5][.5];
        \dlineload{1}{yz}{1}{4}[1][1];
        
        % Dimensions
        \ddimensioning{xy}[0]{3}{2}{-8}[$ \SI{5}{\metre} $][3];
        \ddimensioning{yz}[0]{3}{1}{11}[$ \SI{3}{\metre} $][3];
        \ddimensioning{zx}[0]{1}{4}{-12}[$ \SI{3}{\metre} $][3];
        
        % Labels
        \dnotation{1}{1}{ $ 1 $ }[above left];
        \dnotation{1}{2}{ $ 2 $ }[below right];
        \dnotation{1}{3}{ $ 3 $ }[above left];
        \dnotation{1}{4}{ $ 4 $ }[below right];
        
        \dnotation{1}{a}{$ \SI[per-mode=symbol]{2.4}{\tonne\per\metre} $}[above right=10mm and 0mm];
        \dnotation{1}{4}{$ \SI[per-mode=symbol]{3.5}{\tonne\per\metre} $}[above left=8mm and 0mm];
        
        \dnotation{1}{a}{\textit{(0.30, 0.40)}}[below right];
        \dnotation{1}{b}{\textit{(0.40, 0.25)}}[right];
        \dnotation{1}{c}{\textit{(0.30, 0.40)}}[right];
        
        \dnotation{1}{23}{$ E=220\:\mathrm{t/cm^2}$}[above right=5mm and 10mm];
        \dnotation{1}{23}{$ G=85\:\mathrm{t/cm^2}$}[above right=0mm and 10mm];
    % here we construct our structure
    \end{tikzpicture}
    \caption{Pórtico tridimensional del \textit{Ejemplo 7.6} de \cite{escamilla1995microcomputadores}.}
    \label{fig:segundo_punto}
\end{figure}

\textit{\textbf{Solución -}} La figura a la cual hace referencia el ejemplo 7.6 se presenta en la figura \ref{fig:segundo_punto}. En ésta se presenta la geometría del modelo de la estructura, así como las propiedades del material, las cargas a las que están sometidos y las condiciones de apoyo. \\

Para solucionar el ejemplo anterior usando la librería \texttt{pyFEM}, se ingresa la información necesaria para describir completamente el modelo de la estructura mediante instrucciones en el lenguaje de programación \texttt{python}, las cuales se almacenan en un archivo de texto plano. Dicha información consiste en:
\begin{inparaenum}[$ (1) $]
    \item los materiales, 
    \item las secciones transversales, 
    \item los nodos, 
    \item los elementos tipo pórtico, 
    \item los apoyos y
    \item las cargas.
\end{inparaenum}

En el algoritmo \ref{alg:portico_tridimensional} se presenta las instrucciones que debe recibir \texttt{pyFEM} para solucionar el modelo de la estructura. Las instrucciones consisten en crear un nuevo objeto tipo \texttt{Structure}, al cual se le ha dado el nombre de \textit{structure} y, seguidamente, se agregan los materiales, las secciones, los nodos, los elementos tipo pórtico, los apoyos, los patrones de carga y las cargas en los elementos. Una vez los datos del modelo de la estructura han sido ingresado, se puede solucionar el modelo mediante la instrucción \texttt{structure.solve()}.

\begin{lstlisting}[language=Python,caption=Ingreso de los datos del modelo de la estructura a \texttt{pyFEM}.,label=alg:portico_tridimensional, frame=single]
# structure
structure = Structure()

# add material
structure.materials.add('material1', 220e4, 85e4)

# add sections
structure.sections.add('section1', 'material1', 0.12, 9e-4, 1.6e-3, 1.944e-3)
structure.sections.add('section2', 'material1', 0.10, 1.333e-3, 5.208e-4, 1.2734e-3)

# add nodes
structure.nodes.add('1', 0, 3, 3)
structure.nodes.add('2', 5, 3, 3)
structure.nodes.add('3', 0, 0, 3)
structure.nodes.add('4', 0, 3, 0)

# add frames
structure.frames.add('1-2', '1', '2', 'section1')
structure.frames.add('3-1', '3', '1', 'section1')
structure.frames.add('4-1', '4', '1', 'section2')

# add supports
structure.supports.add('2', *6 * (True,))
structure.supports.add('3', *6 * (True,))
structure.supports.add('4', *6 * (True,))

# add load pattern
structure.load_patterns.add("distributed loads")

# add distributed loads
structure.load_patterns["distributed loads"].distributed_loads.add('1-2', 0, -2.4, 0)
structure.load_patterns["distributed loads"].distributed_loads.add('4-1', 0, -3.5, 0)

# solve
structure.solve()
\end{lstlisting}

Cuando se ejecuta la instrucción \texttt{structure.solve()}, \texttt{pyFEM} comienza a solucionar el modelo de la estructura con base en la información ingresada. A continuación se realiza la comparación entre los resultados obtenidos contra los resultados presentados en \cite{escamilla1995microcomputadores}.

\subsubsection{Desplazamientos de los nodos}

En la tabla \ref{tab:nodes_displacement} se presenta el desplazamientos de los nodos del modelo de la estructura para el patrón de carga \textit{distributed loads}, los cuales son iguales a los presentados en \cite{escamilla1995microcomputadores}.
\begin{table}[h]
    \centering
    \begin{tabular}{|c|c|c|c|c|}
        \hline
        Nodo & 1 & 2 & 3 & 4 \\
        \hline
        $ \mathrm{u_x} $ [\si{\meter}] & \num{2.687e-5} & 0 & 0 & 0 \\
        \hline
        $ \mathrm{u_y} $ [\si{\meter}] & \num{-1.157e-4} & 0 & 0 & 0 \\
        \hline
        $ \mathrm{u_z} $ [\si{\meter}] & \num{-1.001e-5} & 0 & 0 & 0 \\
        \hline
        $ \mathrm{r_x} $ [\si{\radian}] & \num{-5.669e-4} & 0 & 0 & 0 \\
        \hline
        $ \mathrm{r_y} $ [\si{\radian}] & \num{7.905e-6} & 0 & 0 & 0 \\
        \hline
        $ \mathrm{r_z} $ [\si{\radian}] & \num{-6.309e-4} & 0 & 0 & 0 \\
        \hline
    \end{tabular}
    \caption{Desplazamientos en los nodos del modelo para el patrón de carga \textit{distributed loads}.}
    \label{tab:nodes_displacement}
\end{table}

\subsubsection{Reacciones de los apoyos}

En la tabla \ref{tab:support_reactions} se presenta las reacciones de los apoyos del modelo de la estructura para el patrón de carga \textit{distributed loads}, las cuales son practicamente iguales a las presentadas en \cite{escamilla1995microcomputadores}.

\begin{table}[h]
    \centering
    \begin{tabular}{|c|c|c|c|}
        \hline
        Nodo & 2 & 3 & 4 \\
        \hline
        $ \mathrm{F_x} $ [\si{\tonne}] & \num{-1.419} & \num{1.438} & \num{-0.02} \\
        \hline
        $ \mathrm{F_y} $ [\si{\tonne}] & \num{6.572} & \num{1.019e1} & \num{5.742} \\
        \hline
        $ \mathrm{F_z} $ [\si{\tonne}] & \num{5.658e-3} & \num{-7.394e-1} & \num{0.734} \\
        \hline
        $ \mathrm{T} $ [\si{\tonne\meter}] & \num{1.873e-1} & \num{-7.350e-1} & \num{-3.146} \\
        \hline
        $ \mathrm{M_y} $ [\si{\tonne\meter}] & \num{1.102e-2} & \num{-4.354e-3} & \num{-0.037} \\
        \hline
        $ \mathrm{M_z} $ [\si{\tonne\meter}] & \num{-5.986} & \num{-1.417} & \num{0.228} \\
        \hline
    \end{tabular}
    \caption{Reacciones de los apoyos del modelo para el patrón de carga \textit{distributed loads}.}
    \label{tab:support_reactions}
\end{table}



% hacer la comparación entre la respuesta obtenida con la presentada en dichos libros, para el primer caso, o para comparar el tiempo de cálculo requerido entre las herramientas objeto de este trabajo con programas comerciales, para el segundo caso.  \\
% Dichos modelos corresponden, bien sea, a 
% \begin{inparaenum}[ $(1)$]
%     \item ejemplos de los libros de las referencias bibliográficas o
%     \item planteados por el autor,
% \end{inparaenum}
