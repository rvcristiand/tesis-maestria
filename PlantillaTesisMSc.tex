\documentclass[12pt,spanish,fleqn,openany,letterpaper,pagesize]{scrbook}

\usepackage[utf8]{inputenc}
\usepackage[spanish]{babel}
\usepackage{fancyhdr}
\usepackage{epsfig}
\usepackage{epic}
\usepackage{eepic}
\usepackage{amsmath}

\usepackage{amsthm}

\usepackage{nccmath}
\usepackage{threeparttable}
\usepackage{amscd}
\usepackage{here}
\usepackage{graphicx}
\usepackage{lscape}
\usepackage{tabularx}
\usepackage{subfigure}
\usepackage{longtable}
\usepackage{paralist}
\usepackage{blkarray}

% Me
\usepackage{multicol}
\setlength{\columnseprule}{0.4pt}

\usepackage{tikz}
\usepackage[hidelinks]{hyperref}
\usepackage{siunitx}
\usepackage{stanli}

\usepackage{listings}
\usepackage{color}
\usepackage{breakcites}

\usepackage[style=apa,citestyle=apa]{biblatex}
\DefineBibliographyStrings{spanish}{%
  andothers = {et al}
  }
\addbibresource{BibliMSc.bib}

\usepackage{nicematrix}
% \usepackage{titlesec}

\usepackage{tikz-imagelabels}
\imagelabelset{
	outer dist = -0.5 cm
}

\usepackage{dirtree}

\theoremstyle{plain}
\newtheorem{example}{Ejemplo}

\usepackage{caption}

\renewcommand{\lstlistingname}{Algoritmo}% Listing -> Algorithm
\renewcommand{\lstlistlistingname}{Lista de \lstlistingname s}% List of Listings -> List of Algorithm

%New colors defined below
\definecolor{codegreen}{rgb}{0,0.6,0}
\definecolor{codegray}{rgb}{0.5,0.5,0.5}
\definecolor{codepurple}{rgb}{0.58,0,0.82}
\definecolor{backcolour}{rgb}{0.95,0.95,0.92}

%Code listing style named "mystyle"
\lstdefinestyle{mystyle}{
%   backgroundcolor=\color{backcolour},   commentstyle=\color{codegreen},
  keywordstyle=\color{magenta},
  numberstyle=\tiny\color{codegray},
  stringstyle=\color{codepurple},
  basicstyle=\footnotesize,
  breakatwhitespace=false,         
  breaklines=true,                 
  captionpos=top,                    
  keepspaces=true,                 
%   numbers=left,                    
  numbersep=5pt,                  
  showspaces=false,                
  showstringspaces=false,
  showtabs=false,                  
  tabsize=2
}

%"mystyle" code listing set
\lstset{style=mystyle}

% Formato de definiciones y ejemplos 
\usepackage[listingsutf8,breakable,many]{tcolorbox}
% \tcbuselibrary{minted}

% Caja sin título
\newtcolorbox{cajasimple}{
	breakable,
	%    colframe = red!80!black,
	boxrule = 0.1mm,
	arc = 0.5mm,
	fonttitle = \bfseries
}

% Caja con título
\newtcolorbox[auto counter]{caja}[1][]{
	breakable,
	title = \thetcbcounter. #1,
	%    colframe = red!80!black,
	boxrule = 0.1mm,
	arc = 0.5mm,
	fonttitle = \bfseries
}

% Definiciones
\newtcolorbox{defin}[1][]{
	title = #1,
	colframe = red!80!black,
	boxrule = 0.1mm,
	arc = 0.5mm,
	fonttitle = \bfseries
}

% Ejemplos
\newtcolorbox{ejemplo}{
	breakable,
	title={Ejemplo},
	colframe = blue!80!black,
	boxrule = 0.1mm,
	arc = 0.5mm,
	fonttitle = \bfseries
}

\newtcblisting{mylisting}{
  colframe=cyan,
  colback=cyan!10,
  listing only,
  listing engine=minted,
  minted language=cpp,
  minted options={fontsize=\small,linenos,numbersep=3mm},
}

% Códigos de programación
\newtcblisting{codigoprog}[1][]{
	listing only,
	breakable,
	title = #1,
	%colframe = green!30!black,
	boxrule = 0.1mm,
	arc = 0.5mm,
	fonttitle = \bfseries
}

\usepackage{rotating} %Para rotar texto, objetos y tablas seite. No se ve en DVI solo en PS. Seite 328 Hundebuch
                        %se usa junto con \rotate, \sidewidestable ....


\renewcommand{\theequation}{\thechapter-\arabic{equation}}
\renewcommand{\thefigure}{\textbf{\thechapter-\arabic{figure}}}
\renewcommand{\thetable}{\textbf{\thechapter-\arabic{table}}}


\pagestyle{fancyplain}%\addtolength{\headwidth}{\marginparwidth}
\textheight22.5cm \topmargin0cm \textwidth16.5cm
\oddsidemargin0.5cm \evensidemargin-0.5cm%
\renewcommand{\chaptermark}[1]{\markboth{\thechapter\; #1}{}}
\renewcommand{\sectionmark}[1]{\markright{\thesection\; #1}}
\lhead[\fancyplain{}{\thepage}]{\fancyplain{}{\rightmark}}
\rhead[\fancyplain{}{\leftmark}]{\fancyplain{}{\thepage}}
\fancyfoot{}
\thispagestyle{fancy}%


\addtolength{\headwidth}{0cm}
\unitlength1mm %Define la unidad LE para Figuras
\mathindent0cm %Define la distancia de las formulas al texto,  fleqn las descentra
\marginparwidth0cm
\parindent0cm %Define la distancia de la primera linea de un parrafo a la margen

%Para tablas,  redefine el backschlash en tablas donde se define la posici\'{o}n del texto en las
%casillas (con \centering \raggedright o \raggedleft)
\newcommand{\PreserveBackslash}[1]{\let\temp=\\#1\let\\=\temp}
\let\PBS=\PreserveBackslash

%Espacio entre lineas
\renewcommand{\baselinestretch}{1.1}

%Neuer Befehl f\"{u}r die Tabelle Eigenschaften der Aktivkohlen
\newcommand{\arr}[1]{\raisebox{1.5ex}[0cm][0cm]{#1}}

%Neue Kommandos
\usepackage{Befehle}
%Inicio del documento. Tener en cuenta que hay archivos auxiliares

% Select what to do with todonotes: 
% \usepackage[disable]{todonotes} % notes not showed
\usepackage[draft]{todonotes}   % notes showed

% Select what to do with command \comment:  
% \newcommand{\comment}[1]{}  %comment not showed
% \newcommand{\comment}[1]
% {\par {\bfseries \color{blue} #1 \par}} %comment showed


\begin{document}
\pagenumbering{roman}
%\newpage
%\setcounter{page}{1}
\begin{center}
\begin{figure}
\centering%
\epsfig{file=HojaTitulo/EscudoUN,scale=1}%
\end{figure}
\thispagestyle{empty} \vspace*{2.0cm} \textbf{\huge
Programa de computador para resolver modelos de estructuras tridimensionales de elementos rectos}\\[6.0cm]
\Large\textbf{Cristian Danilo Ramirez Vargas}\\[5.0cm]
\small Universidad Nacional de Colombia\\
Facultad de Ingeniería, Departamento de Ingeniería Civil y Agrícola\\
Bogotá D. C., Colombia\\
\the\year\\
\end{center}

\newpage{\pagestyle{empty}\cleardoublepage}

\newpage
\begin{center}
\thispagestyle{empty} \vspace*{0cm} \textbf{\huge
Programa de computador para resolver modelos de estructuras tridimensionales de elementos rectos}\\[2.0cm]
\Large\textbf{Cristian Danilo Ramirez Vargas}\\[3.0cm]
\small Tesis o trabajo de grado presentada(o) como requisito parcial para optar al
t\'{\i}tulo de:\\
\textbf{Magister en Estructuras}\\[2.5cm]
Director(a):\\
Ph. D. Martín Estrada Mejia\\[2.0cm]
L\'{\i}nea de Investigaci\'{o}n:\\
Análisis de estructuras\\
Grupo de Investigaci\'{o}n:\\
Análisis, diseño y materiales - GIES\\[2.5cm]
Universidad Nacional de Colombia\\
Facultad de Ingeniería, Departamento de Ingeniería Civil y Agrícola\\
Bogotá D. C., Colombia\\
\the\year\\
\end{center}

\newpage{\pagestyle{empty}\cleardoublepage}

\newpage
\thispagestyle{empty} \textbf{}\normalsize
\\\\\\%
% \textbf{(Dedicatoria o un lema)}\\[4.0cm]

\begin{flushright}
\begin{minipage}{10cm}
    \noindent
    \small
    Hombres inteligentes gran pensantes\\
    Hemos creado un monstruo\\
    Las bombas radiactivas y nucleares\\
    Que descompondrán la humanidad\\
    Quien totalmente se autodestruirá\\
    
    Ya creador no hay para volver a comenzar\\
    Como dijo la sagrada maldición\\
    El universo en siete días lo creo\\
    
    \emph{Razas de todos los colores\\
    Tomemos una reacción\\
    Potencias monopolizadoras\\
    Analicen esta situación\\
    Países tercermundistas\\
    De brazos no nos crucemos\\
    Acabemos pronto con esto}\\
    
    Futuro nunca habrá\\
    Futuro nunca ha habido\\
    Este en mundo que esta perdido\\
    Dependiendo de un botón\\
    Y de la decisión\\
    De un idealista cabrón\\
    
    La tercera guerra mundial\\
    Será un estruendo nuclear\\
    Donde historiadores no podrán narrarla\\
    Y los humanos no podremos resistirla\\
    
    \emph{Las invenciones científicas\\
    Lejos de liberar de la ignorancia\\
    Y del trabajo envilecedor\\
    Lo aumentan\\
    Y hacen más refinada la servidumbre}\\
    
    ---\emph{La ciencia de la autodestrucción}, La Pestilencia (1989)\\
\end{minipage}
\end{flushright}

\newpage{\pagestyle{empty}\cleardoublepage}

\newpage
\thispagestyle{empty} \textbf{}\normalsize
\\\\\\%
\textbf{\LARGE Agradecimientos}
\addcontentsline{toc}{chapter}{\numberline{}Agradecimientos}\\

No habría podido hacer este trabajo sin el acompañamiento del profesor Martín Estrada. Su conocimiento del mundo de la programación me ayudó en momentos decisivos durante el desarrollo del código. No sé como hacer para agradecerle por su paciencia.\\

Este trabajo también se debe al curso \emph{Computación Visual} del profesor Jean Pierre Charalambos. Su descripción de la \emph{escena} y como trabajar con ella fue la que me permitió hacer \emph{FEM.js}.\\

Y por si esto no fuera suficiente, apliqué el concepto de \emph{cuaternión} en el problema de la rotación de los ejes de referencia tiempo después de haberlo estudiado en una de sus clases, lo que me permitió implementar el método de análisis matricial de manera innovadora.\\

Finalmente, quiero agradecer a la profesora Maritzabel Molina ya que mi entendimiento del método de análisis matricial proviene de su curso \emph{análisis estructural avanzado}. A ella nos debemos muchos ingenieros estructurales.

\newpage{\pagestyle{empty}\cleardoublepage}

\newpage
% \textbf{\LARGE Resumen}
% \addcontentsline{toc}{chapter}{\numberline{}Resumen}\\\\
% El resumen es una presentaci\'{o}n abreviada y precisa (la NTC 1486 de 2008 recomienda revisar la norma ISO 214 de 1976). Se debe usar una extensi\'{o}n m\'{a}xima de 12 renglones. Se recomienda que este resumen sea anal\'{\i}tico, es decir, que sea completo, con informaci\'{o}n cuantitativa y cualitativa, generalmente incluyendo los siguientes aspectos: objetivos, dise\~{n}o, lugar y circunstancias, pacientes (u objetivo del estudio), intervenci\'{o}n, mediciones y principales resultados, y conclusiones. Al final del resumen se deben usar palabras claves tomadas del texto (m\'{\i}nimo 3 y m\'{a}ximo 7 palabras), las cuales permiten la recuperaci\'{o}n de la informaci\'{o}n.\\

% \textbf{\small Palabras clave: (m\'{a}ximo 10 palabras, preferiblemente seleccionadas de las listas internacionales que permitan el indizado cruzado)}.\\

% A continuaci\'{o}n se presentan algunos ejemplos de tesauros que se pueden consultar para asignar las palabras clave, seg\'{u}n el \'{a}rea tem\'{a}tica:\\

% \textbf{Artes}: AAT: Art y Architecture Thesaurus.

% \textbf{Ciencias agropecuarias}: 1) Agrovoc: Multilingual Agricultural Thesaurus - F.A.O. y 2)GEMET: General Multilingual Environmental Thesaurus.

% \textbf{Ciencias sociales y humanas}: 1) Tesauro de la UNESCO y 2) Population Multilingual Thesaurus.

% \textbf{Ciencia y tecnolog\'{\i}a}: 1) Astronomy Thesaurus Index. 2) Life Sciences Thesaurus, 3) Subject Vocabulary, Chemical Abstracts Service y 4) InterWATER: Tesauro de IRC - Centro Internacional de Agua Potable y Saneamiento.

% \textbf{Tecnolog\'{\i}as y ciencias m\'{e}dicas}: 1) MeSH: Medical Subject Headings (National Library of Medicine's USA) y 2) DECS: Descriptores en ciencias de la Salud (Biblioteca Regional de Medicina BIREME-OPS).

% \textbf{Multidisciplinarias}: 1) LEMB - Listas de Encabezamientos de Materia y 2) LCSH- Library of Congress Subject Headings.\\

% Tambi\'{e}n se pueden encontrar listas de temas y palabras claves, consultando las distintas bases de datos disponibles a trav\'{e}s del Portal del Sistema Nacional de Bibliotecas\footnote{ver: www.sinab.unal.edu.co}, en la secci\'{o}n "Recursos bibliogr\'{a}ficos" opci\'{o}n "Bases de datos".\\

% \textbf{\LARGE Abstract}\\\\
% Es el mismo resumen pero traducido al ingl\'{e}s. Se debe usar una extensi\'{o}n m\'{a}xima de 12 renglones. Al final del Abstract se deben traducir las anteriores palabras claves tomadas del texto (m\'{\i}nimo 3 y m\'{a}ximo 7 palabras), llamadas keywords. Es posible incluir el resumen en otro idioma diferente al espa\~{n}ol o al ingl\'{e}s, si se considera como importante dentro del tema tratado en la investigaci\'{o}n, por ejemplo: un trabajo dedicado a problemas ling\"{u}\'{\i}sticos del mandar\'{\i}n seguramente estar\'{\i}a mejor con un resumen en mandar\'{\i}n.\\[2.0cm]
% \textbf{\small Keywords: palabras clave en ingl\'{e}s(m\'{a}ximo 10 palabras, preferiblemente seleccionadas de las listas internacionales que permitan el indizado cruzado)}\\

\renewcommand{\tablename}{\textbf{Tabla}}
\renewcommand{\figurename}{\textbf{Figura}}
\renewcommand{\listtablename}{Lista de Tablas}
\renewcommand{\listfigurename}{Lista de Figuras}
\renewcommand{\contentsname}{Contenido}

\newcommand{\clearemptydoublepage}{\newpage{\pagestyle{empty}\cleardoublepage}}
\tableofcontents

\cleardoublepage
\addcontentsline{toc}{chapter}{Lista de figuras} % para que aparezca en el indice de contenidos
\listoffigures % indice de figuras

\cleardoublepage
% \addcontentsline{toc}{chapter}{Lista de tablas} % para que aparezca en el indice de contenidos
% \listoftables % indice de tablas

\addcontentsline{toc}{chapter}{Lista de algoritmos}
\lstlistoflistings
\cleardoublepage

% \cleardoublepage
% \include{Tab_Simbolos/TabSimbolosMSc}
%\include{Resumen}%\newcommand{\clearemptydoublepage}{\newpage{\pagestyle{empty}\cleardoublepage}}
\pagenumbering{arabic}

% \usetikzlibrary{arrows, shapes, positioning, shadows, trees}
\tikzset{
  basic/.style  = {draw, text width=2cm, drop shadow, font=\sffamily, rectangle},
  root/.style   = {basic, rounded corners=2pt, thin, align=center, fill=white},
  level 2/.style = {basic, rounded corners=6pt, thin,align=center, fill=white, text width=8em},
  level 3/.style = {basic, thin, align=left, fill=white, text width=6.5em}
}

\chapter{Marco Teórico}
% Lorem ipsum dolor sit amet, consectetur adipiscing elit. Sed vitae libero fermentum, consequat sem a, mattis magna. Mauris neque elit, varius hendrerit neque ut, elementum tempus nisi. Nulla in porttitor augue. Morbi ut turpis lorem. Phasellus porta feugiat dui, ut lacinia nisl. Nulla blandit ornare dolor, vel iaculis velit suscipit non. Morbi egestas ex eu mauris tempus, et fringilla enim tristique. Suspendisse potenti. Integer ut nibh lorem. Integer odio neque, suscipit vel ex commodo, congue aliquet dui.\\

% Maecenas lobortis purus at diam pellentesque, ut mollis sapien cursus. Etiam ultricies porta purus at tincidunt. Proin scelerisque erat sit amet hendrerit condimentum. Aenean quis leo molestie, tristique mauris id, facilisis dolor. Aliquam ornare sollicitudin dolor, nec pellentesque mauris laoreet vitae. In pulvinar commodo convallis. Ut ut venenatis enim, tincidunt elementum ante. Donec ut varius augue, faucibus vehicula massa. Proin placerat, augue porttitor semper laoreet, urna augue euismod leo, sit amet volutpat erat orci non dui. Nam lacinia nec felis quis tempor. Mauris eleifend at turpis nec efficitur.\\

\section{StressUN}
\textit{StressUN} es una libreria que hace uso de \textit{processing}\footnote{véase https://processing.org/} y \textit{frames}\footnote{véase https://github.com/VisualComputing/frames} como librerias. Se podría pensar a \textit{StressUN} como una aplicación específica de la libreria \textit{frames}. \\

La libreria tiene por objetivo ser un \textit{pre procesador} y \textit{pos procesador} en el análisis de estructuras conformadas por elementos tipo \textit{póritco}.\\

La libreria se divide en cuatro grupos de clases, como se muestra en la \ref{fig:modulosStressUN}, que son
\begin{itemize}
    \item \textit{constants},
    \item \textit{core},
    \item \textit{exceptions} y
    \item \textit{primitives}.
\end{itemize}

\begin{figure}[ht]
    \centering
    
    \begin{tikzpicture}[
        level 1/.style={sibling distance=40mm}, edge from parent/.style={->,draw}, >=latex]
        % root of the the initial tree, level 1
        \node[root] {StreesUN}
        % The first level, as children of the initial tree
        child {node[level 2] (c1) {constants}}
        child {node[level 2] (c2) {core}}
        child {node[level 2] (c3) {exceptions}}
        child {node[level 2] (c4) {primitives}};

        % The second level, relatively positioned nodes
        \begin{scope}[every node/.style={level 3}]
            \node [below of = c1, xshift=20pt] (c11) {KeyShortcuts};
            \node [below of = c11] (c12) {Status};

            \node [below of = c2, xshift=20pt] (c21) {CommandLine};
            \node [below of = c21] (c22) {FileManager};
            \node [below of = c22] (c23) {Physics};
            \node [below of = c23] (c24) {Stress};
            \node [below of = c24] (c25) {Style};

            \node [below of = c3, xshift=20pt] (c31) {InvalidScene
            Exception};
            
            \node [below of = c4, xshift=20pt] (c41) {Axis};
            \node [below of = c41] (c42) {Grid};
            \node [below of = c42] (c43) {Grid3D};
            \node [below of = c43] (c44) {Node};
            \node [below of = c44] (c45) {Point};
            \node [below of = c45] (c46) {Portico};
            
        \end{scope}

        % lines from each level 1 node to every one of its "children"
        \foreach \value in {1,2}
            \draw[->] (c1.195) |- (c1\value.west);

        \foreach \value in {1,...,5}
            \draw[->] (c2.195) |- (c2\value.west);

        \foreach \value in {1}
            \draw[->] (c3.195) |- (c3\value.west);
        
        \foreach \value in {1,...,6}
            \draw[->] (c4.195) |- (c4\value.west);
    \end{tikzpicture}
    
    \caption{Clases de la librería \textit{StressUN}.}
    \label{fig:modulosStressUN}
\end{figure}

\subsection{constants}
\subsubsection{KeyShortcuts}
Se almacenan los atajos de teclado que el usuario tiene a disposición.
\subsubsection{Status}
Se almecena el el estado del programa, el cual varia entre el \textit{canvas}, el \textit{command line} y el \textit{menu}.

\subsection{core}
\subsubsection{CommandLine}
Recibe los comandos por teclado
\subsubsection{FileManager}
Administra los archivos que contienen la información del modelo.
\subsubsection{Physics}
Proceso.
\subsubsection{Stress}
Administra las demás clases.
\subsubsection{Style}
Admisnitra los estilos del \textit{pre proceso} y del \textit{pos proceso}.

\subsection{excepetions}
\subsubsection{InvalidSceneException}
Generar un error si no se usa \textit{P3D}.

\subsection{primitives}
\subsubsection{Axis}
Ejes de la estructura.
\subsubsection{Grid}
Conjunto de ejes en un mismo plano.
\subsubsection{Grid3D}
Conjunto de grids.
\subsubsection{Node}
Extremo de un elemento tipo portico.
\subsubsection{Point}
Sirve como snap.
\subsubsection{Portico}
Elemento de la estructura.

% \subsection{Subt\'{\i}tulos nivel 3}
% De la cuarta subdivisi\'{o}n en adelante, cada nueva divisi\'{o}n o \'{\i}tem puede ser se\~{n}alada con vi\~{n}etas, conservando el mismo estilo de \'{e}sta, a lo largo de todo el documento.\\

% Las subdivisiones, las vi\~{n}etas y sus textos acompa\~{n}antes deben presentarse sin sangr\'{\i}a y justificados.\\

% \begin{itemize}
% \item En caso que sea necesario utilizar vi\~{n}etas, use este formato (vi\~{n}etas cuadradas).
% \end{itemize}
\chapter{Introducción}
\label{chap:antecedentes}

\begin{nota}
Los programas de computador comerciales para el análisis y diseño de estructuras que se encuentran vigentes a la fecha cuentan, en general, con un entorno gráfico que le permite al usuario describir el modelo de forma interactiva, procesarlo y visualizar los resultados de manera conveniente.\\
\tcblower
Creo que este primer párrafo debe ser algo más relacionado con la importancia de tener programas de computador para el análisis estructural, y luego, la importancia de que esos programas sean libres y permitan el escrutinio y modificación por parte de los usuarios y desarrolladores del país. Pongo algo de la propuesta que se presentó hace tiempos, pero creo que se puede completar.
\end{nota}

El fundamento teórico de los métodos matriciales para el análisis de estructuras fue propuesto por George A. Maney, quien desarrolló el método de pendiente deflexión en 1915. Por ello algunos lo consideran como el precursor del método matricial por rigideces y se hacen notar que la principal desventaja de este método en su tiempo fue la solución simultanea de múltiples ecuaciones algebraicas \parencite{Kassimali2011}.

Con la llegada de los computadores se inició el desarrollo de programas de análisis estructural en las universidades. Uno de los primeros en ser desarrollados fue \emph{STRESS} (Structural Engineering System Solver) en el MIT (Massachusetts Institute Technology) en 1964, que marcó un hito en el empleo de los computadores. \emph{STRESS} se convirtió en un lenguaje de programación que permitía analizar estructuras al describirlas en tarjetas perforadas mediante la numeración de los nodos y sus respectivas coordenadas en el espacio, la numeración de los elementos tipo pórtico y sus respectivos nudos, la aplicación de cargas en los elementos y las condiciones de apoyo.

Aunque un gran número de ingenieros han desarrollado programas de análisis estructural para aprovechar la capacidad de cálculo de los computadores, una reseña que incluya todos los programas sería casi interminable. En \textcite{escamilla1995microcomputadores} se presenta una buena selección de algunos de estos programas de uso común en América Latina, de la cual se hará un resumen a continuación, incluyendo \emph{ETABS} (Three Dimensional Analysis of Building Systems - Extended Version).\\

\begin{nota}
\emph{ETABS} es un programa de computador creado por Edward Wilson, Jeffery Hollings y Henry Dovey en 1975. Según \textcite{ETABS1975}, este programa de computador fue desarrollado para el análisis estructural lineal de edificios de pórticos y muros a cortante sujetos a cargas estáticas y dinámicas, como las sísmicas. El edificio es idealizado como un sistema de elementos aporticados y muros a cortante independientes interconectado por losas de entrepiso consideradas diafragmas rígidos. Este programa es una extensión de \emph{TABS} (Three Dimensional Analysis of Building Systems), el cual fue concebido para analizar pórticos tridimensionales. Para \textcite{ETABS1972}, una de las razones para desarrollar \emph{TABS} fue darle una retroalimentación a los usuarios de los programas \emph{FRMSTC} (Static Load Analysis of High-Rise Buildings), \emph{FRMDYN} (Dynamic Analysis of Multistory Buildings), \emph{LATERAL} y \emph{SOLID SAP} (Static Analysis Program for Three-Dimensional Solid Structures).\\
\tcblower
Si se puede, agregar las fechas de los programas
\end{nota}

\emph{FRMSTC} permitía analizar edificios simétricos con pórticos y muros a cortante paralelos sujetos a cargas estáticas, con capacidad para evaluar los modos y las frecuencias de vibración. \emph{FRMDYN} era similar, pero las acciones externas consistían en imponer la aceleración del terreno debido a un desplazamiento dependiente del tiempo. Por su lado, \emph{LATERAL} fue una extensión de \emph{FRMSTC} que permitía analizar linealmente pórticos y muros a cortante que no eran necesariamente paralelos, con tres grados de libertad en cada piso. \emph{SOLID SAP} se escribió como un programa general de elementos finitos con una opción particular para considerar la aproximación de diafragma rígido. En este último se podían realizar análisis estáticos y dinámicos de las estructuras.\\

Los programas enunciados anteriormente fueron algunos de los precursores de los programas comerciales actuales para el análisis de estructuras, pero su interacción con los usuarios, exclusivamente mediante archivos de texto y ventanas de comandos, no tuvo un completa aceptación entre los ingenieros. Ello evolucionó, naturalmente, en desarrollos importantes en dos frentes: los programas de análisis y las interfaces de pre- y pos-proceso. 

En la actualidad, \emph{ETABS} se encuentra en la versión 18.1.1 y, según \cite{ETABS2020systemrequirements}, puede hacer análisis estructurales estáticos y dinámicos con gran variedad de estructuras, {\color{red}considerando linealidad o no linealidad material o geométrica --> VERIFICAR}. Además, cuenta con una interfaz gráfica amigable. A través de múltiples cuadros de diálogo, los cuales son accesibles mediante la barra de menús, las barras de herramientas, el explorador del modelo, las vistas del modelo o con atajos de teclado, el usuario puede modelar la estructura que desea analizar al describir los materiales, las secciones transversales, los elementos estructurales, las condiciones de apoyo, los diafragmas y las cargas.\\

% En la figura \ref{fig:etabs_start_page} se presenta la ventana del programa ETABS ejecutándose en un computador con Windows 10.\\

% ser ejecutado en computadores con sistema operativo Windows 7, Windows 8 o Windows 10 con arquitectura de 64 bits que cuenten como mínimo con un procesador Intel Pentium 4 o AMD Athlon 64, una resolución de 1024x768 pixeles con 16 bits por canal, 8 GB de RAM y 6 GB de espacio en el disco duro. 


% \begin{figure}[ht]
%   \centering
%   \begin{annotationimage}{width=0.5\textwidth}{introduction/etabs-startup.png}
%     \draw[annotation left = {{Barra de título} at 0.91}] to (0.15, 0.81);
%     \draw[annotation left = {{Barra de menús} at 0.7}] to (0.15, 0.8);
%     \draw[annotation left = {{Explorador del modelo} at 0.5}] to (0.18, 0.65);
%     \draw[annotation left = {{Barra de herramientas} at 0.3}] to (0.15, 0.4);
%     \draw[annotation left = {{Barra de estado} at 0.125}] to (0.15, 0.225);
%     \draw[annotation right = {{Barra de herramientas} at 0.655}] to (0.84, 0.765);
%     \draw[annotation right = {{Vista del modelo} at 0.47}] to (0.75, 0.67);
%     \draw[annotation right = {{Indicador de actualizaciones} at 0.86}] to (0.85, 0.76);
%   \end{annotationimage}
%   \caption{Ventana del programa ETABS ejecutandose en Windows 10.}
%   \label{fig:etabs_start_page}
% \end{figure}


Según \cite{ETABS2017analysisreferencemanual}, ETABS analiza el modelo usando el motor de análisis \emph{SAPFire}, el cual es común a otros programas de la misma compañia (\emph{SAP2000}, \emph{SAFE} y \emph{CSiBridge}). \emph{SAPFire} es la última versión de la serie de programas \emph{SAP} y ofrece las siguientes herramientas:

\begin{itemize}
\item Análisis estático y dinámico,
\item Análisis lineal y no lineal,
\item Análisis sísmico y análisis incremental no lineal (\emph{pushover}),
\item Análisis de cargas móviles,
\item No linealidad geométrica, incluyendo efectos P-delta y grandes desplazamientos,
\item Etapas constructivas,
\item Fluencia lenta (\emph{creep}), retracción (\emph{shrinkage}) y envejecimiento,
\item Análisis de pandeo,
\item Análisis de densidad espectral de potencia y estado estacionario,
\item Elementos aporticados y laminares, incluyendo el comportamiento de vigas, columnas, cerchas, membranas y placas,
\item Elementos tipo cable y tendón,
\item Elementos bidimensionales planos y elementos sólidos asimétricos,
\item Elementos sólidos tridimensionales,
\item Resortes no lineales y apoyos,
\item Propiedades de los resortes y apoyos dependientes de la frecuencia,
\end{itemize}

% Con los resultados del análisis del modelo, el posprocesador de ETABS puede \emph{diseñar} los elementos estructurales de acuerdo a uno de varios códigos de diseño de diferentes países. ETABS es capaz de diseñar pórticos en acero, pórticos en concreto, vigas compuestas, columnas compuestas, vigas en acero de alma abierta (\emph{steel joist}), muros a cortante y losas de concreto.\\

% Adicionalmente, según \cite{ETABS2019welcome}, ETABS cuenta con la posibilidad de generar dibujos estructurales esquemáticos de las plantas estructurales, de los despieces de vigas, columnas y muros a cortante, y de los detalles de las conexiones de acero.\\

\begin{nota}
En el mercado existen otros programas comerciales de análisis estructural, como \emph{TEKLA}, \emph{MIDAS}, {\color{red}... cuáles???}. En términos generales, todos ellos cuentan con características similares a las descritas para \emph{ETABS}. Actualmente, dichos programas están innovando para permitirle al usuario trabajar con modelos \emph{BIM} (Building Information Modeling).\\
\tcblower
  Creo que vala la pena hacer una tabla con una lista de programas de análisis estructural, que contenga: desarrollador, país, lenguaje, licencia, tipo de análisis (resumido), Interfaz gráfica (Si o No). En ella se deberán incluir los programas comerciales y algunos de los programas gratis, tipo OpenSees u Onsas.
\end{nota}

Adicionalmente, en el mercado existen numerosos programas generales para realizar análisis numéricos con el método de los elementos finitos. Aunque es posible realizar análisis estructurales con algunos de ellos, no están diseñados específicamente para eso, por lo cual no se incluyen en esta reseña.

\section{Problema} % No sé si esta sección deba llamarse Argumento o Justificación 

\begin{nota}
  \tcblower
  Hace falta un buen texto para convencer al lector de que es importante el desarrollo de un programa con las siguientes características:

  \begin{itemize}
    \item gratis
    \item código abierto
    \item lenguaje moderno y amigable 
  \end{itemize}

  Y una interfaz con :

  \begin{itemize}
    \item conexión con el programa de análisis
    \item UI moderna
    \item tecnología 4.0 
    \item trabajo colabrativo --> mejor en web
    \item extandible a una plataforma BIM 
  \end{itemize}

  Hay que decir que obviamente los programas comerciales grandes tienen la parte de la interfaz y el trabajo colaborativo y todo eso, pero que son prohibitivos para la mayoría de ingenieros. No solo en paíces en desarrollo como Colombia sino en todo el mundo. Esto frena el desarrollo de la tecnología y la infraestructura de Colombia y el mundo.
\end{nota}

\section{Objetivo}

Desarrollar un programa de computador a código abierto para el análisis lineal de estructuras aporticadas tridimensionales sometidas a cargas estáticas.\\

Con este trabajo se pretende contribuir al ejercicio libre de la profesión del ingeniero estructural y a la enseñanza del análisis de las estructuras.
\subsection{Objetivos específicos}

\begin{itemize}
\item Desarrollar el módulo de análisis estructural para calcular el desplazamiento de los nudos, el valor de las reacciones y de las fuerzas internas de los elementos de una estructura sometida a cargas estáticas.
\item Desarrollar el ambiente gráfico y la interfaz gráfica de usuario del programa de computador para permitirle al usuario ingresar los datos que describen la estructura, las acciones a las cuales se encuentra sometida y visualizar los resultados del análisis estructural.
\end{itemize}

\section{Metodología}

{\color{red} 
El capítulo de metodología debe ser una cosa diferente. Debe incluir más o menos lo siguiente: 

\begin{itemize}
  \item Esquema general del desarrollo del problema: explicar que se decidió separar el proceso de cálculo de la representación gráfica de la estructura en el pre- y pos-proceso. Por eso se pensó en desarrollar dos programas diferentes, específicamente desarrollados para cumplir propósitos específicos. Así, el programa de representación gráfica no incluye nada respecto al análisis matricial y, por su lado, el de análisis no incluye nada que tenga que ver con la visualización gráfica. Esto significa que es necesario pensar también en una manera de transferir la información, con la cual no se vea comprometida ninguna parte del proceso. La información se debe transferir sin limitaciones de lenguaje de programación ni tipo de datos.

  \item Programa de análisis: 
    \begin{itemize} 
      \item Alcance del programa: lineal elástico, 3D, barras, ... 
      \item Lenguaje de programación: hacer una tabla con lenguajes interpretados y no interpredatos, dinámicos y estáticos, etc. Luego decir por qué se escogió python.
      \item Librerías: hacer una tabla con las librerías de python que se necesitan y poner al frente para qué sirve cada una. Comentar qué problemas traería si no se usan estas librerías y se intenta programas todo con python puro? Tiempo de cálculo? Programación complicada? no sé... 
      \item Paradigma de programación: tabla o lista con los diferentes paradigmas y después explicar por qué se escogió un paradigma de programación orientada a objetos.
    \end{itemize}

  \item Programa de pre- y pos-proceso:
    \begin{itemize}
      \item Alcance del programa: tipo de visualizaciones necesarias, respondiendo al alcance del prorama de análisis. Por ejemplo, no es necesario en esta tesis tener capacidad de visualización dinámica porque tampoco el programa de análisis lo permite. Tipo de resultados que se pueden ver.
      \item Plataformas de programación visual: Explicar que el programa de pre- y pos-proceso requiere dos partes fundamentalmente. La interfaz de usuario (UI) que contiene los botones y elementos con los que el usuario interactua, y por otro lado un espacio para dibujar o representar gráficamente el modelo estructural. Poner una lista de lenguajes comunes para la UI (c++, Qt, Java, JavaScript, etc.). Exponer una tabla con las librerías o plataformas para la otra parte que es la representación gráfica tridimensional del modelo (panda3d, Three.js, processing, etc). Explicar que se escogió Javascript por diferentes razones: la UI (Vue, React, Angular, etc) y la estructura en 3D (Three.js) se puede hacer con este lenguaje por lo que no hay que pensar en compatibilidad, evitar problemas de instalación en diferentes equipos y diferentes sistemas operativos, se pretende que el programa se ejecute desde un servidor para que a futuro se pueda trabajar colaborativamente.
      \item Librerías adicionales: poner una lista de las librerías adicionales y poner para qué sirven.
    \end{itemize}

  \item Sistema de transferencia de datos o información:
    \begin{itemize}
      \item Lenguajes o metodologías más comunes para transferir datos: hacer una lista o tabla (json, xml, CSV, SPSS, txt, etc.) 
      \item Ventajas de Json y por qué se escogió
    \end{itemize}

\end{itemize}
}

Se desarrollaron los programas de computador \emph{pyFEM} y \emph{FEM.js}. El primero para analizar estructuras aporticadas tridimensionales sometidas a cargas estáticas y el segundo para modelarlas. Esto con el fin que FEM.js pueda ser usado junto con otro programa de computador diferente a pyFEM.\\

Durante el desarrollo de estos programas, así como el de este documento, se utilizó \emph{git} como sistema de control de versiones. Según \cite{chacon2014git}, git es un sistema distribuido de control de versiones que registra los cambios realizados a un conjunto de archivos para coordinar el trabajo entre programadores.\\

Una copia de los repositorios de pyFEM y FEM.js se encuentran en la página de internet \emph{GitHub}, la cual permite alojar proyectos utilizando git. pyFEM está alojado en \url{https://github.com/rvcristiand/pyFEM} mientras que FEM.js está alojado en \url{https://github.com/rvcristiand/FEM.js}.\\

\subsection{pyFEM}

pyFEM fue desarrollado en \emph{Python}. Según \cite{lutz2013python}, Python es un lenguaje de programación interpretado orientado a objetos cuya filosofía hace enfasis en la legibilidad de su código. Los archivos revelantes que componen el repositorio de pyFEM son:
\pagebreak

\dirtree{%
  .1 pyFEM/.
  .2 LICENSE.
  .2 README.md.
  .2 example\_1.json.
  .2 example\_2.json.
  .2 example\_3.json.
  .2 pyFEM/.
  .3 classtools.py.
  .3 core.py.
  .3 primitives.py.
  .2 test/.
  .3 space\_frame.py.
  .3 trusses.py.
}
\bigskip
El archivo \verb|LICENCE| contiene la licencia de pyFEM mientras que el archivo \verb|README.md| contiene todas las instrucciones necesarias para ejecutar y usar pyFEM.\\

La carpeta \emph{pyFEM} contiene los archivos \verb|classtools.py|, \verb|core.py| y \verb|primitives.py|, los cuales contienen las instrucciones para analizar los modelos. La extensión \emph{py} se usa para indicar que los archivos son programas de Python.\\

En el archivo \verb|classtools.py| se encuentran las \emph{clases} \verb|UniqueInstances| y \verb|AttrDisplay|, la primera para evitar que se creen \emph{objetos} de una misma clase con la misma información y la segunda para generar una representación conveniente de los objetos.\\

En el archivo \verb|primitives.py| se encuentran varias clases, entre ellas \verb|Material|, \verb|Section|, \verb|Joint|, \verb|Frame|, \verb|Support|, \verb|LoadPattern|, etc., las cuales permiten describir los diferentes atributos del modelo.\\

En el archivo \verb|core.py| se encuentra la clase \verb|Structure| la cual permite describir estructuras para ser analizados. Para crear objetos de esta clase se debe llamar la clase indicando los grados de libertad a tener en cuenta en el análisis. A partir de un objeto de esta clase es posible describir el modelo de la estructura al agregar materiales, secciones transversales, nodos, elemenetos aporticados, apoyos, patrones de carga, cargas en los nodos y cargas distribuidas en los elementos aporticados.\\

Los archivos \verb|example_1.json|, \verb|example_2.json| y \verb|example_3.json| almacenan los modelos de tres de los ejemplos presentados en \cite{escamilla1995microcomputadores} que han sido analizados con pyFEM. La extensión \emph{json} se usa para indicar que los archivos tienen formato \emph{JSON} (de sus siglas en inglés \emph{JavaScript Object Notation}), el cual es un formato sencillo para el intercambio de datos. El modelo es descrito de tal manera que puede ser interpretado por FEM.js para generar su representación en una escena tridimensional.\\

En el ejemplo \ref{ej:cercha-plana} se presenta la solución a un ejercicio de \cite{escamilla1995microcomputadores} usando pyFEM. En el capítulo \ref{cha:pyFEM} se presentan las rutinas que ejecuta pyFEM para solucionar los modelos estructurales.\\

\begin{ejemplo}
  \label{ej:cercha-plana}
  Resuelva completamente la cercha mostrada por el método matricial de los desplazamientos. El material es acero estructural con $ E = 2040\; t / cm^2 $. Las áreas están dadas entre paréntesis en $ cm^2 $.\\  
  \begin{center}
    \begin{tikzpicture}
      % Points
      \point{1}{0}{0};
      \point{2}{8}{0};
      \point{3}{4}{3};
      \point{4}{4}{0};
      
      \point{p}{4}{-1.3};
      \point{q}{5.05}{3.7875};
      
      \point{1-3}{2}{1.5};
      \point{1-4}{2}{0};
      \point{3-2}{6}{1.5};
      \point{4-2}{6}{0};
      \point{4-3}{4}{1.5};
      
      % Beams
      \beam{2}{1}{3}[0][0];
      \beam{2}{1}{4}[1][1];
      \beam{2}{3}{2}[1][1];
      \beam{2}{4}{2}[1][1];
      \beam{2}{4}{3}[1][1];
      
      % Supports
      \support{1}{1};
      \support{2}{2};
      
      % Joints
      \hinge{1}{1};
      \hinge{1}{2};
      \hinge{1}{3};
      \hinge{1}{4};
      
      % Loads
      \load{1}{q}[216.87];
      \load{1}{p}[90];
      
      % notation
      \notation{1}{1}{1}[above left];
      \notation{1}{2}{2}[above right];
      \notation{1}{3}{3}[above left];
      \notation{1}{4}{4}[below left];
      
      \notation{1}{3}{$ \SI{5}{\tonne} $}[above right=10mm];
      \notation{1}{4}{$ \SI{20}{\tonne} $}[below right=5mm and 0mm];
      
      \notation{5}{1}{3}[$ (100) $ ][.5][above][0.5];
      \notation{5}{1}{4}[$ (40) $ ][.5][above][0.5];
      \notation{5}{3}{2}[$ (150) $ ][.5][above][0.5];
      \notation{5}{4}{2}[$ (40) $ ][.5][above][0.5];
      \notation{5}{4}{3}[$ (30) $ ][.5][right][1];
            
      % Dimensions
      \dimensioning{1}{1}{4}{-1.5}[$ \SI{4}{m} $];
      \dimensioning{1}{4}{2}{-1.5}[$ \SI{4}{m} $];
      \dimensioning{2}{2}{3}{9}[$ \SI{3}{m} $];
    \end{tikzpicture}
    \captionof{figure}{Cercha simple plana del \textit{Ejemplo 7.1} de \cite{escamilla1995microcomputadores}.}
  \end{center}

  \emph{\textbf{Solución -}} En el algoritmo \ref{alg:cercha_plana} se presenta un programa de Python para analizar el modelo de la estructura usando pyFEM. Las instrucciones consisten en crear un nuevo objeto tipo \verb|Structure|, al cual se le ha dado el nombre \emph{model}, agregarle
  \begin{inparaenum}[$ (a) $]
  \item materiales,
  \item secciones transversales,
  \item nodos,
  \item elementos aporticados,
  \item apoyos, patrones de carga y
  \item cargas en los nodos,
  \end{inparaenum}
  analizar el modelo y exportarlo a formato JSON.\\

  Cuando se ejecuta la instrucción \verb|model.solve()| pyFEM comienza a solucionar el modelo de la estructura. Los pasos que efectúa para solucionar el modelo son:
  \begin{inparaenum}[$ (1) $]
  \item asignar los grados de libertad de los nodos, 
  \item ensamblar la matriz de rigidez del modelo de la estructura, 
  \item imponer las condiciones de apoyo en la matriz de rigidez del modelo, 
  \item ensamblar el vector de fuerzas en los nodos para cada uno de los patrones de carga, 
  \item imponer las condiciones de apoyo en el vector de fuerzas en los nodos para cada caso de carga, 
  \item encontrar los desplazamientos de los nodos para cada patrón de carga, 
  \item encontrar las reacciones en los apoyos para cada patrón de carga y
  \item guardar la solución en los nodos y en los apoyos para cada patrón de carga.
  \end{inparaenum}
  
  \begin{lstlisting}[language=Python,caption=Ingreso de los datos del modelo de la estructura a \textit{pyFEM}.,label=alg:cercha_plana, frame=single]
# create the model
model = Structure(ux=True, uy=True)

# add materials
model.add_material(key='1', modulus_elasticity=2040e4)

# add sections
model.add_section(key='1', area=030e-4)
model.add_section('2', area=040e-4)
model.add_section('3', area=100e-4)
model.add_section('4', area=150e-4)

# add joints
model.add_joint(key=1, x=0, y=0)
model.add_joint(2, 8, 0)
model.add_joint(3, 4, 3)
model.add_joint(4, 4, 0)

# add frames
model.add_frame(key='1-3', key_joint_j=1, key_joint_k=3, key_material='1', key_section='3')
model.add_frame('1-4', 1, 4, '1', '2')
model.add_frame('3-2', 3, 2, '1', '4')
model.add_frame('4-2', 4, 2, '1', '2')
model.add_frame('4-3', 4, 3, '1', '1')

# add supports
model.add_support(key_joint=1, ux=True, uy=True)
model.add_support(2, ux=False, uy=True)

# add load patterns
model.add_load_pattern(key='point loads')

# add point loads
model.add_load_at_joint(key_load_pattern='point loads', key_joint=3, fx=5 * 0.8, fy=5 * 0.6)
model.add_load_at_joint('point loads' , 4, fy=-20)

# solve the problem
model.solve()
print(model)

# export the model
model.export('example_1.json')
  \end{lstlisting}
% \end{minipage}

  Para realizar el ensamblaje de la matriz de rigidez del modelo de la estructura y del vector de fuerzas de los nodos, pyFEM asigna números a los grados de libertad de los nodos de la estructura en el orden en que fueron ingresados; al nodo 1 se le han asignado los grados de libertad \emph{0} y \emph{1}, al nodo 2 los grados de libertad \emph{2} y \emph{3}, y así sucesivamente.\\

  Una vez se establecen los grados de libertad de los nodos se ensambla la matriz de rigidez del modelo de la estructura. Este proceso consiste en calcular una a una las matrices de rigidez de los elementos ensamblandolas en la matriz de rigidez del modelo.\\

  El usuario puede consultar las matrices de rigidez de cada uno de los elementos del modelo de la estructura. En \eqref{eq:matriz-rigidez-elemento-portico-coordenadas-locales} se presenta la matriz de rigidez en coordenadas locales del elemento 1-3, la cual se obtiene mediante la instrucción \verb|model.frames['1-3'].get_local_stiffness_matrix()|.\\

  \begin{equation}
    \label{eq:matriz-rigidez-elemento-portico-coordenadas-locales}
    \begin{bNiceArray}{CCCCC}
      40800 & 0 & -40800 & 0\\
      0 & 0 &      0 & 0\\
      -40800 & 0 & 40800 & 0\\
      0 & 0 &      0 & 0\\
    \end{bNiceArray}
    \si[per-mode=symbol]{\tonne\per\meter}
  \end{equation}

  Así mismo, el usuario puede consultar las matrices de rotación de cada uno de los elementos del modelo. En \eqref{eq:matriz-rotacion} se presenta la matriz de rotación del elemento 1-3, la cual se obtiene mendiante la instrucción \verb|model.frames['1-3'].get_rotation_matrix()|.\\

  \begin{equation}
    \label{eq:matriz-rotacion}
    \begin{bNiceArray}{CCCCC}
      0.8 & -0.6 &   0 &    0\\
      0.6 &  0.8 &   0 &    0\\
      0 &    0 & 0.8 & -0.6\\
      0 &    0 & 0.6 &  0.8\\
    \end{bNiceArray}
  \end{equation}

  El usuario tambien puede consultar las matrices de rigidez en coordenadas globales de cada uno de los elementos del modelo de la estructura. En \eqref{eq:matriz-rigidez-elemento-portico-coordenadas-globales} se presenta la matriz de rigidez en coordenadas globales del elemento 1-3, con sus respectivos grados de libertad, la cual se obtiene mediante la instrucción \verb|model.frames['1-3'].get_global_stiffness_matrix()|.

  \begin{equation}
    \label{eq:matriz-rigidez-elemento-portico-coordenadas-globales}
    \begin{bNiceArray}{CCCCC}[
        first-row,
        first-col,
      ]
      & \emph{0} & \emph{1} & \emph{4} & \emph{5}\\
      \emph{0} & 26112 & 19584 & -261112 & -19584\\
      \emph{1} & 19584 & 14688 & -19584 & -14688\\
      \emph{4} & -26112 & -19584 & 26112 & 19584\\
      \emph{5} & -19584 & -14688 & 19584 & 14688\\
    \end{bNiceArray}
    \si[per-mode=symbol]{\tonne\per\meter}
  \end{equation}

  En \eqref{eq:k_1_4}, \eqref{eq:k_3_2} y \eqref{eq:k_4_3} se presentan las matrices de rigidez en coordenadas globales de los elementos 1-4, 3-2 y 4-3 de la estructura las cuales se obtienen con instrucciones similares a la anterior.\\

  \begin{equation}
    \begin{bNiceArray}{CCCCC}[
        first-row,
        first-col,
      ]
      \label{eq:k_1_4}
      & \emph{0} & \emph{1} & \emph{6} & \emph{7}\\
      \emph{0} & 20400 & 0 & -20400 & 0\\
      \emph{1} & 0 & 0 & 0 & 0\\
      \emph{6} & -20400 & 0 & 20400 & 0\\
      \emph{7} & 0 & 0 & 0 & 0 \\
    \end{bNiceArray}
    \si[per-mode=symbol]{\tonne\per\meter}
  \end{equation}

  \begin{equation}
    \begin{bNiceArray}{CCCCC}[
        first-row,
        first-col,
      ]
      \label{eq:k_3_2}    
      & \emph{4} & \emph{5} & \emph{2} & \emph{3}\\
      \emph{4} & 39168 & -29376 & -39168 & 29376\\
      \emph{5} & -29376 & 22032 & 29376 & -22032\\
      \emph{2} & -39168 & 29376 & 39168 & -29376\\
      \emph{3} & 29376 & -22032 & -29376 & 22032\\
    \end{bNiceArray}
    \si[per-mode=symbol]{\tonne\per\meter}
  \end{equation}

  \begin{equation}
    \begin{bNiceArray}{CCCCC}[
        first-row,
        first-col,
      ]
      \label{eq:k_4_3}
      & \emph{6} & \emph{7} & {4} & \emph{5}\\
      \emph{6} & 0 & 0 & 0 & 0\\
      \emph{7} & 0 & 20400 & 0 & -20400\\
      \emph{4} & 0 & 0 & 0 & 0\\
      \emph{5} & 0 & -20400 & 0 & 20400\\
    \end{bNiceArray}
    \si[per-mode=symbol]{\tonne\per\meter}
  \end{equation}

  El usuario también puede consultar la matriz de rigidez del modelo de la estructura mediante la instrucción \verb|structure.get_stiffness_matrix()|. En \eqref{eq:k_global} se presenta la matriz de rigidez de la estructura.\\

  \begin{equation}
    \begin{bNiceArray}{CCCCCCCC}[
        first-row,
        first-col,
      ]
      & \emph{0} & \emph{1} & \emph{2} & \emph{3} & \emph{4} & \emph{5} & \emph{6} & \emph{7}\\
      \emph{0} &  46512 &  19584 &      0 &      0 & -26112 & -19584 & -20400 &      0\\
      \emph{1} &  19584 &  14688 &      0 &      0 & -19584 & -14688 &      0 &      0\\
      \emph{2} &      0 &      0 &  59568 & -29376 & -39168 &  29376 & -20400 &      0\\
      \emph{3} &      0 &      0 & -29376 &  22032 &  29376 & -22032 &      0 &      0\\
      \emph{4} & -26112 & -19584 & -39168 &  29376 &  65280 &  -9792 &      0 &      0\\
      \emph{5} & -19584 & -14688 &  29376 & -22032 &  -9792 &  57120 &      0 & -20400\\
      \emph{6} & -20400 &      0 & -20400 &      0 &      0 &      0 &  40800 &      0\\
      \emph{7} &      0 &      0 &      0 &      0 &      0 & -20400 &      0 &  20400
      \label{eq:k_global}
    \end{bNiceArray}
    \si[per-mode=symbol]{\tonne\per\meter}
  \end{equation}

  Una vez se ensambla la matriz de rigidez del modelo se modifica para tener en cuenta las condiciones de apoyo. Este proceso consiste en modificar las filas y las columnas asociadas a los grados de libertad restringidos. En \eqref{eq:k_global_support_applied} se presenta la matriz de rigidez sujeta a las condiciones de apoyo del modelo de la estructura la cual se obtiene mediante la instrucción \verb|model.get_stiffness_matrix_with_support()|.\\

  \begin{equation}
    \label{eq:k_global_support_applied}
    \begin{bNiceArray}{CCCCCCCC}[
        first-row,
        first-col,
      ]
      & \emph{0} & \emph{1} & \emph{2} & \emph{3} & \emph{4} & \emph{5} & \emph{6} & \emph{7}\\
      \emph{0} & 1 & 0 &      0 & 0 &      0 &      0 &      0 &      0\\
      \emph{1} & 0 & 1 &      0 & 0 &      0 &      0 &      0 &      0\\
      \emph{2} & 0 & 0 &  59568 & 0 & -39168 &  29376 & -20400 &      0\\
      \emph{3} & 0 & 0 &      0 & 1 &      0 &      0 &      0 &      0\\
      \emph{4} & 0 & 0 & -39168 & 0 &  65280 &  -9792 &      0 &      0\\
      \emph{5} & 0 & 0 &  29376 & 0 &  -9792 &  57120 &      0 & -20400\\
      \emph{6} & 0 & 0 & -20400 & 0 &      0 &      0 &  40800 &      0\\
      \emph{7} & 0 & 0 &      0 & 0 &      0 & -20400 &      0 &  20400
    \end{bNiceArray}
    \si[per-mode=symbol]{\tonne\per\meter}
  \end{equation}

  Una vez se obtiene la matriz de rigidez modificada del modelo de la estructura se resuelve para cada uno de los patrones de carga.\\

  Así como se deben encontrar las matrices de rigidez de cada uno de los elementos del modelo de la estructura para posteriormente ensamblarlas, se debe encontrar la acción en los nodos de cada carga. En \eqref{eq:f_global} se presenta el vector de fuerzas nodales del modelo para el patrón de carga \emph{point loads} mediante la instrucción \verb|structure.load_patterns['point loads'].get_f()|.\\

  \begin{equation}
    \label{eq:f_global}
    \begin{BNiceArray}{CC}[
        first-col,
      ]
      \emph{0} & 0\\
      \emph{1} & 0\\
      \emph{2} & 0\\
      \emph{3} & 0\\
      \emph{4} & 4\\
      \emph{5} & 3\\
      \emph{6} & 0\\
      \emph{7} & -20
    \end{BNiceArray}
    \si{\tonne}
  \end{equation}

  \bigskip
  Obtenido el vector de fuerzas para dicho patrón de carga se imponen las condiciones de apoyo del modelo de la estructura. Debido a que los desplazamiento en los apoyos son iguales a cero el vector de fuerzas en los nodos no varia.\\

  Al contar con la matriz de rigidez y el vector de fuerzas en los nodos, ambos modificados por las condiciones de apoyo, se calculan los desplazamientos y las reacciones. En \eqref{eq:u_point_loads} y \eqref{eq:f_point_loads_solution} se presentan el vector de desplazamientos y el vector de fuerzas en los nodos del modelo de la estructura para el patrón de carga \textit{point loads}, los cuales son iguales a los presentados en \cite{escamilla1995microcomputadores}.

  \begin{equation}
    \label{eq:u_point_loads}
    \begin{BNiceArray}{CC}[
        first-col,
      ]
      \emph{0} & 0\\
      \emph{1} & 0\\
      \emph{2} & 1.307\\
      \emph{3} & 0\\
      \emph{4} & 0.645\\
      \emph{5} & -1.337\\
      \emph{6} & 0.654\\
      \emph{7} & -2.317\\
    \end{BNiceArray}
    \SI{1e-3}{\meter}
  \end{equation}

  \begin{equation}
    \label{eq:f_point_loads_solution}
    \begin{BNiceArray}{CC}[
        first-col,
      ]
      \emph{0} & -4\\
      \emph{1} & 7\\
      \emph{2} & 0\\
      \emph{3} & 10\\
      \emph{4} & 4\\
      \emph{5} & 3\\
      \emph{6} & 0\\
      \emph{7} & -20
    \end{BNiceArray} 
    \si{\tonne} 
  \end{equation}

  Cuando se ejecuta la instrucción \verb|print(model)| pyFEM genera un informe del análisis. A continuación se presenta el informe generado para el objecto \verb|model|.\\

  \begin{lstlisting}[language={}, frame=single]
Flag joint displacements
------------------------
ux: True
uy: True
uz: False
rx: False
ry: False
rz: False

Materials
---------
label   E                       G
1               20400000.0,     0

Sections
--------
label   A               Ix      Iy      Iz
1               0.003,  0,      0,      0
2               0.004,  0,      0,      0
3               0.01,   0,      0,      0
4               0.015,  0,      0,      0

Joints
------
label   x       y       z
1               0,      0,      0
2               8,      0,      0
3               4,      3,      0
4               4,      0,      0

Frames
------
label   Joint j Joint k material        section
1-3             1               3               1                       3
1-4             1               4               1                       2
3-2             3               2               1                       4
4-2             4               2               1                       2
4-3             4               3               1                       1

Supports
--------
label   ux              uy              uz              rx              ry              rz
1               True,   True,   False,  False,  False,  False
2               False,  True,   False,  False,  False,  False

Load patterns
-------------
point loads:
label        fx        fy        fz        mx        my        mz
3              4.0       3.0         0         0         0         0
4                0       -20         0         0         0         0

Displacements
-------------
point loads:
label         ux         uy         uz         rx         ry         rz
1               +0.00000,       +0.00000,       +0.00000,       +0.00000,       +0.00000,       +0.00000
2               +0.00131,       +0.00000,       +0.00000,       +0.00000,       +0.00000,       +0.00000
3               +0.00065,       -0.00134,       +0.00000,       +0.00000,       +0.00000,       +0.00000
4               +0.00065,       -0.00232,       +0.00000,       +0.00000,       +0.00000,       +0.00000

Reactions
---------
point loads:
label         fx         fy         fz         mx         my         mz
1               -4.00000,       +7.00000,       +0.00000,       +0.00000,       +0.00000,       +0.00000
2               +0.00000,       +10.00000,      +0.00000,       +0.00000,       +0.00000,       +0.00000
  \end{lstlisting}

  Cuando se ejecuta la instrucción \verb|model.export('example_1.json')| pyFEM genera un archivo que contiene toda la información del modelo en formato JSON para ser leído por FEM.js. A continuación se presenta la información del modelo en formato JSON.\\
  \begin{lstlisting}[language={}, frame=single]
{
    "materials": {
        "1": {
            "E": 20400000.0,
            "G": 0
        }
    },
    "sections": {
        "1": {
            "area": 0.003,
            "Ix": 0,
            "Iy": 0,
            "Iz": 0,
            "type": "Section"
        },
        "2": {
            "area": 0.004,
            "Ix": 0,
            "Iy": 0,
            "Iz": 0,
            "type": "Section"
        },
        "3": {
            "area": 0.01,
            "Ix": 0,
            "Iy": 0,
            "Iz": 0,
            "type": "Section"
        },
        "4": {
            "area": 0.015,
            "Ix": 0,
            "Iy": 0,
            "Iz": 0,
            "type": "Section"
        }
    },
    "joints": {
        "1": {
            "x": 0,
            "y": 0,
            "z": 0
        },
        "2": {
            "x": 8,
            "y": 0,
            "z": 0
        },
        "3": {
            "x": 4,
            "y": 3,
            "z": 0
        },
        "4": {
            "x": 4,
            "y": 0,
            "z": 0
        }
    },
    "frames": {
        "1-3": {
            "j": 1,
            "k": 3,
            "material": "1",
            "section": "3"
        },
        "1-4": {
            "j": 1,
            "k": 4,
            "material": "1",
            "section": "2"
        },
        "3-2": {
            "j": 3,
            "k": 2,
            "material": "1",
            "section": "4"
        },
        "4-2": {
            "j": 4,
            "k": 2,
            "material": "1",
            "section": "2"
        },
        "4-3": {
            "j": 4,
            "k": 3,
            "material": "1",
            "section": "1"
        }
    },
    "supports": {
        "1": {
            "ux": true,
            "uy": true,
            "uz": false,
            "rx": false,
            "ry": false,
            "rz": false
        },
        "2": {
            "ux": false,
            "uy": true,
            "uz": false,
            "rx": false,
            "ry": false,
            "rz": false
        }
    },
    "load_patterns": {
        "point loads": {
            "joints": {
                "3": [
                    {
                        "fx": 4.0,
                        "fy": 3.0,
                        "fz": 3.0,
                        "mx": 0,
                        "my": 0,
                        "mz": 0
                    }
                ],
                "4": [
                    {
                        "fx": 0,
                        "fy": -20,
                        "fz": -20,
                        "mx": 0,
                        "my": 0,
                        "mz": 0
                    }
                ]
            }
        }
    }
}
  \end{lstlisting}    
\end{ejemplo}

\subsection{FEM.js}

FEM.js fue desarrollado en \emph{Three.js}. Según \cite{dirksen2015threejs}, Three.js es un \emph{API} (de sus siglas en inglés \emph{application programming interface}) programada en \emph{JavaScript} para \emph{WebGL} que permite crear escenas tridimensionales en el navegador de internet. Los archivos revelantes que componen el repositorio de FEM.js son:
\bigskip

\dirtree{%
  .1 FEM.js/.
  .2 LICENSE.
  .2 README.md.
  .2 css/.
  .3 style.css.
  .2 example\_1.json.
  .2 example\_2.json.
  .2 example\_3.json.
  .2 index.html.
  .2 libs/.
  .3 CSS2DRenderer.js.
  .3 OrbitControls.js.
  .3 Projector.js.
  .3 dat.gui.min.js.
  .3 stats.js.
  .3 three.js.
  .2 main.js.
  .2 modules/.
  .3 FEM.js.
  .3 terminal.js.
}
  
\bigskip

El archivo \verb|LICENCE| contiene la licencia de FEM.js mientras que el archivo \verb|README.md| contiene todas las instrucciones necesarias para ejecutar y usar FEM.js.\\

El archivo \verb|index.html| define la estructura de la página web de FEM.js. En la \emph{etiqueta} \verb|head| se define la ubicación los archivos \verb|three.js|, \verb|CCS2Renderer.js|, \verb|OrbitControls.js|, \verb|dat.gui.min.js|, \verb|stats.js|, el estilo de la página según el archivo \verb|style.css| y el \emph{módulo} \verb|main.js|. En la etiqueta \verb|body| se definen las secciones \verb|renderer-output| y \verb|console|, para mostrar la escena tridimensional y recibir las instrucciones del usuario respectivamente (véase la figura \ref{fig:FEM.js-running}).\\

Los archivos \verb|three.js|, \verb|CCS2Renderer.js| y \verb|OrbitControls.js| son necesarios para renderizar gráficos con WebGL, asociar objetos de la escena con etiquetas html y manipular la cámara. Estos archivos hacen parte del repositorio del proyecto Three.js alojado en GitHub (\url{https://github.com/mrdoob/three.js/}).\\

Los archivos \verb|dat.gui.min.js| y \verb|stats.js| permiten crear interfaces gráficas de usuario que cambian el valor de las variables y un monitor del desemepeño del código respectivamente. Estos archivos hacen parte de los repositorios \emph{dat.gui} y \emph{stats.js} alojados en GitHub (\url{https://github.com/dataarts/dat.gui} y \url{https://github.com/mrdoob/stats.js/}).\\

El archivo \verb|style.css| define la presentación de las etiquetas html de la página web.\\

El archivo \verb|main.js| define las funciones del archivo \verb|FEM.js| que ejecuta \verb|terminal.js| y algunos \emph{eventos} para que todos los elementos de la página funcionen adecuadamente. El archivo \verb|terminal.js| define una serie de funciones para interpretar y ejecutar las instrucciones que ingrese el usuario.\\

El archivo \verb|FEM.js| contiene la configuración por defecto del programa, la descripción del panel lateral derecho y todas las funciones que hacen posible que el usuario pueda interactuar con el modelo.\\

Los archivos \verb|example_1.json|, \verb|example_2.json| y \verb|example_3.json| almacenan los modelos de tres de los ejemplos presentados en \cite{escamilla1995microcomputadores} que han sido generados con pyFEM.\\

FEM.js puede representar cualquiera de estos archivos al ejecutar la función \verb|open()| con el nombre del archivo entre comillas dobles o sencillas como dato de entrada. En la figura \ref{fig:FEM.js-running} se presenta FEM.js con el archivo \verb|example_2.json| abierto ejecutándose en el navegador de internet Firefox.\\

\begin{figure}[ht]
  \centering
  \begin{annotationimage}{width=0.5\textwidth}{introduction/FEM.js-running.png}
    \imagelabelset{outer dist = 0.5 cm}
    \draw[annotation left = {{renderer-output} at 0.81}] to (0.15, 0.71);
    \draw[annotation left = {{console} at 0.15}] to (0.15, 0.05);
    \draw[annotation right = {{dat.GUI} at 0.665}] to (0.84, 0.765);
    
  \end{annotationimage}
  \caption{FEM.js ejecutándose en Firefox}
  \label{fig:FEM.js-running}
\end{figure}

Así mismo, es capaz de ejecutar todas las funciones que se definan con \verb|add_funcion()| del archivo \verb|terminal.js|. Esta función recibe como parámetros el nombre de la función y un objecto el cual debe definir la propiedad \verb|func|. El nombre de la función se usa para llamar al parámetro \verb|func| con los valores seprados por comas ingresados por el usuario.\\

A continuación se presentan una lista de las funciones definidas en \verb|main.js|.\\

\begin{multicols}{2}
  \setlength{\columnseprule}{0pt}
  \begin{itemize}
  \item addFrame,
  \item removeFrame,
  \item addSection,
  \item addRectangularSection,
  \item removeSection,
  \item addMaterial,
  \item removeMaterial,
  \item addJoint,
  \item removeJoint,
  \item setFrameView,
  \item showJointsLabel,
  \item hideJointsLabel,
  \item showFramesLabel,
  \item hideFramesLabel,
  \item setUpwardsAxis,
  \item setView,
  \item open,
  \item getStructure,
  \item getLoadPatterns.
  \end{itemize}
\end{multicols}

Aunque el nombre de la función no tenga que ser necesariamente igual al del parámetro \verb|func|, todos los nombres de la lista coínciden con funciones definidas en el archivo \verb|FEM.js| (aun cuando no es necesario que estén definidas ahí).\\

La descripción de los parámetros de entrada de cada una de estás funciones se encuentran en el archivo \verb|README.md|. A partir de dichas instrucciones es posible generar el modelo tridimensional de la estructura. Por ejemplo, para generar el modelo de la estructura del ejemplo \ref{ej:cercha-plana} se deben ingresar las siguientes instrucciones\\
\begin{lstlisting}[language={}, frame=single]
addMaterial(1, 2040e4)

addSection(1)
addSection(2)
addSection(3)
addSection(4)

addJoint(1, 0, 0, 0)
addJoint(2, 8, 0, 0)
addJoint(3, 4, 3, 0)
addJoint(4, 4, 0, 0)

addFrame(1-3, 1, 3, 1, 3)
addFrame(1-4, 1, 4, 1, 2)
addFrame(3-2, 3, 2, 1, 4)
addFrame(4-2, 4, 2, 1, 2)
addFrame(4-3, 4, 3, 1, 1)

addSupport(1, true, true)
addSupport(2, false, true)

addLoadPattern('point loads')

addLoadAtJoint('point loads', 3, 4, 3)
addLoadAtJoint('point loads', 4, 0, -20)
\end{lstlisting}

En la figura \ref{fig:FEM.js-cercha-plana} se presenta FEM.js después de ejecutar los anteriores comandos.

\begin{figure}[ht]
  \centering
  \includegraphics[width=0.8\textwidth]{introduction/FEM.js-cercha-plana.png}
  \caption{Ejemplo \ref{ej:cercha-plana} modelado en FEM.js.}
  \label{fig:FEM.js-cercha-plana}
\end{figure}

\subsubsection{dat.gui}
El panel lateral derecho de FEM.js fue desarrollado con \emph{dat.GUI} para que el usuario pueda personalizar la escena. Este panel está agrupado en las siguientes categorias:

\begin{multicols}{2}
  \begin{itemize}
  \item model,
  \item background,
  \item camera,
  \item controls,
  \item ground,
  \item axes,
  \item joint,
  \item frame,
  \item support,
  \item load.
  \end{itemize}
\end{multicols}

En la sección \verb|model| se establece la orientación del modelo al definir uno de los ejes principales del modelo que apunta hacía la parte superior de la pantalla y una subsección llamada \emph{axes}. En esta sección se define el tamaño y visibilidad de los ejes principales del modelo y dos subsecciones llamadas \emph{head} y \emph{shaft}. En estas susecciones se define la geometría de la cabeza y la cola de los vectores de los ejes principales del modelo.\\

% En la figura \ref{fig:dat.gui-model} se presenta esta sección expandida.

% \begin{figure}[ht]
%   \centering
%   \includegraphics[width=0.8\textwidth]{introduction/dat-gui-model.png}
%   \caption{Sección \emph{model} del panel lateral de FEM.js.}
%   \label{fig:dat.gui-model}
% \end{figure}

En la sección \verb|background| se establen dos colores para generar el fondo de la escena en gradiente. El color \emph{top} define el color para la parte superior del fondo de la escena mientras que el color \emph{bottom} define el color para la parte inferior.\\

% En la figura \ref{fig:dat.gui-background} se presenta esta sección expandida.\\

% \begin{figure}[ht]
%   \centering
%   \includegraphics[width=0.8\textwidth]{introduction/dat-gui-background.png}
%   \caption{Sección \emph{background} del panel lateral de FEM.js.}
%   \label{fig:dat.gui-background}
% \end{figure}

En la sección \verb|camera| se establece el tipo de proyección de la cámara pudiéndose elegir entre perspectiva y ortogonal. En la figura \ref{fig:dat.gui-camera} se presenta el modelo del archivo \verb|example_2.json| en proyección ortogonal.\\

\begin{figure}[ht]
  \centering
  \includegraphics[width=0.8\textwidth]{introduction/dat-gui-camera.png}
  \caption{FEM.js en proyección ortogonal.}
  \label{fig:dat.gui-camera}
\end{figure}

En la sección \verb|controls| se establece el comportamiento de los controles de FEM.js. Ahí se define la velocidad con la que estos hacen rotar, hacen \emph{zoom}, desplazan la escena, si se desplaza la escena paralelo al plano del modelo o al plano de la proyección y una subsección llamada \verb|damping|. En esta subsección se define si se adiciona un \emph{amortiguamiento} a la rotación y la intensidad del mismo.\\

En la sección \verb|ground| se define la visibilidad y el tamaño del conjunto de elementos \emph{plano} y \emph{grilla} así como dos secciones llamadas \verb|plane| y \verb|grid|. En la sección \verb|plane| se define la visibilidad, el color, la transparencia y la opacidad del plano del modelo mientras que en la sección \verb|grid| se define la visibilidad, el número de divisiones y los colores de las divisiones mayores y menores de la grilla.\\

En la sección \verb|axes| se definen tres colores los cuales se asocian a los ejes $ x $, $ y $ y $ z $. Estos colores establecen los colores de los ejes globales y locales, los apoyos y las cargas. En la figura \ref{fig:dat.gui-axes} se presenta el modelo del archivo \verb|example_3.json| con una definición alternativa de dichos colores.\\

\begin{figure}[ht]
  \centering
  \includegraphics[width=0.8\textwidth]{introduction/dat-gui-axes.png}
  \caption{Colores alternativos para los elementos asociados a los ejes $ x $, $ y $ y $ z $.}
  \label{fig:dat.gui-axes}
\end{figure}

En la sección \verb|joint| se define la visibilidad, el tamaño, el color, la transparencia y la opacidad de los nodos del modelo. Así mismo se define la visibilidad de los \emph{labels} de los nodos.\\

En la sección \verb|frame| se define la visibilidad, la vista (\emph{extruida} o en \emph{palillo}), el tamaño, el color, la transparencia y la opacidad de los elementos aporticados del modelo. Así mismo se define la visibilidad de los \emph{labels} de estos elementos y una sección llamada \verb|axes|, similar a la que se encuentra en la sección \verb|model|, con la diferencia que esta establece la visibilidad y el tamaño de los ejes locales de los elementos aporticados. En la figura \ref{fig:dat.gui-frame} se presenta el modelo del archivo \verb|example_3.json| en \emph{estructura de palillo}.\\

\begin{figure}[ht]
  \centering
  \includegraphics[width=0.8\textwidth]{introduction/dat-gui-frame.png}
  \caption{Vista del modelo como \emph{estructura de palillo}.}
  \label{fig:dat.gui-frame}
\end{figure}

En la sección \verb|support| se define la visibilidad, el \emph{modo} de los apoyos del modelo y dos secciones llamadas \verb|analytical| y \verb|space|. Los modos de los apoyos pueden ser \emph{space} o \emph{analytical}, en donde estos se representan con alguna de las analogías usadas en la literatura para representar apoyos o mediante vectores con un disco inclinado en la mitad de las colas.\\

En la sección \verb|analytical| se definen tres secciones llamadas \verb|head|, \verb|shaft| y \verb|restraint|, con las cuales se puede definir la geometría de los vectores con colas rectas o curvas (para representar restricciones a la traslación o a la rotación respectivamente) que tienen un un disco inclinado en la mitad de la cola.\\

En la sección \verb|space| se definen tres secciones llamadas \verb|foundation|, \verb|pedestal| y \verb|pin|, con las cuales se puede definir la geometría de los elementos \emph{fundación}, \emph{pedestal} o \emph{rótula}, usados para representar los apoyos como elementos espaciales. Cuando se restringen todas las traslaciones y rotaciones el apoyo se resepresenta mediante un pedestal y una fundación, mientras que si se restrigen solo las translaciones el apoyo se representa por un pedestal y una pirámide con base cuadrada. Estos apoyos toman los colores definidos en la sección \verb|axes| de manera conveniente.\\

En la figura \ref{fig:dat.gui-support} se presenta el modelo del archivo \verb|example_3.json| con los apoyos en modo \emph{analytical}.\\

\begin{figure}[ht]
  \centering
  \includegraphics[width=0.8\textwidth]{introduction/dat-gui-support.png}
  \caption{Apoyos del modelo en modo \emph{analytical}.}
  \label{fig:dat.gui-support}
\end{figure}

En la sección \verb|load| se define la visibilidad, el patrón de cargas, el sistema de coordenadas de referencia y como se representan las cargas del modelo, así como cuatro secciones con los nombres \verb|force|, \verb|torque|, \verb|head| y \verb|shaft|. En el momento únicamente se cuenta con el sistemda de refencia global para representar las cargas mientras que se pueden represetar como resultantes o componentes, aunque esta última opción actualmente sólo esta disponible para las cargas puntuales.\\

En las secciones \verb|force|, \verb|torque|, \verb|head| y \verb|shaft| se definen las dimensiones y el color de los elementos que representan las cargas. El tamaño de los diferentes elementos para representar las cargas se escalan en función de valor que estas representen.\\

\chapter{pyFEM}
\label{cha:pyFEM}

pyFEM es un programa de computador desarrollado en Python para analizar linealmente estructuras aporticadas tridimensionales sometidas a cargas estáticas. Una copia del programa se encuentra alojada en la página web de GitHub \url{https://github.com/rvcristiand/pyFEM}.\\

Los principales archivos del programa son:\\

\dirtree{%
  .1 pyFEM/.
  .2 LICENSE.
  .2 README.md.
  .2 example\_1.json.
  .2 example\_2.json.
  .2 example\_3.json.
  .2 pyFEM/.
  .3 classtools.py.
  .3 core.py.
  .3 primitives.py.
  .2 test/.
  .3 space\_frame.py.
  .3 trusses.py.
}
\bigskip
El usuario puede generar objetos de la clase \verb|Structure|, definida en el archivo \verb|core.py|, para describir y analizar linealmente modelos de estructuras aporticadas tridimensionales sometidas a fuerzas estáticas. En la figura \ref{fig:pyFEM-Structure} se presentan los métodos y atributos de esta clase.\\

\begin{figure}[ht]
  \centering
  \begin{tikzpicture}
    \node (clase) [class] {\textbf{Structure}};

    % attrs
    \node (atributos) [left=of clase, xshift=-1cm]{};

    % grados de libertad
    \matrix [
    matrix of nodes,
    below=of atributos,
    nodes={
      anchor=center
    }] (gradosLibertad) {
      \node {ux,}; & \node {uy,}; & \node {uz,};\\
      \node {rx,}; & \node {ry,}; & \node {rz,};\\
    };

    % diccionarios
    \matrix [
    matrix of nodes,
    below=of gradosLibertad,
    yshift=1cm,
    nodes={
      anchor=center
    }] (diccionarios) {      
      \node {materials,};\\
      \node {sections,};\\
      \node {joints,};\\      
      \node {frames,};\\
      \node {supports,};\\      
      \node {load\_patterns,};\\
    };

    % resultados
    \matrix [
    matrix of nodes,
    below=of diccionarios,
    yshift=1cm,
    nodes={
      anchor=center
    }] (resultados) {      
      \node {displacements,};\\
      \node {reactions};\\
    };
    
    % methods
    \node (aux2) [right=of clase, xshift=1cm]{};
    \matrix [
    matrix of nodes,
    below=of aux2,
    nodes={
      anchor=center
    }] (metodos1) {
      \node {add\_material(),}; & \node {add\_section(),};\\
      \node {add\_rectangular\_section(),}; & \node {add\_joint(),};\\
      \node {add\_frame(),}; & \node {add\_support(),};\\
      \node {add\_load\_pattern(),}; & \node {add\_load\_at\_joint(),};\\
    };

    \matrix [
    matrix of nodes,
    below=of metodos1,
    yshift=1cm,
    nodes={
      anchor=center
    }] (metodos2) {
      \node {add\_distributed\_load(),};\\
      \node {get\_flag\_active\_joint\_displacements(),};\\
      \node {get\_number\_active\_joint\_displacements(),};\\
    };
    
    \matrix [
    matrix of nodes,
    below=of metodos2,
    yshift=1cm,
    nodes={
      anchor=center
    }] (metodos3) {
      \node {get\_number\_joints(),}; & \node {get\_number\_frames(),};\\
      \node {set\_indexes(),}; & \node {get\_stiffness\_matrix(),};\\
    };

    \matrix [
    matrix of nodes,
    below=of metodos3,
    yshift=1cm,
    nodes={
      anchor=center
    }] (metodos4) {
      \node {get\_stiffness\_matrix\_with\_support(),};\\
      \node {solve\_load\_pattern(),};\\
      \node {set\_load\_pattern\_displacements(),};\\
    };

    \matrix[
    matrix of nodes,
    below=of metodos4,
    yshift=1cm,
    nodes={
      anchor=center
    }] (metodos5) {
      \node {set\_load\_pattern\_reactions(),}; & \node {solve(),};\\
    };
    
    \node [below=of metodos5, yshift=1cm] (metodos6) {export()};
    
    \node [atributos, fit=(gradosLibertad) (diccionarios) (resultados)] {};
    \node [metodos, fit=(metodos1) (metodos2) (metodos3) (metodos4) (metodos5) (metodos6)] {};

    \draw[myarrow] (gradosLibertad.north) -- ++(0,0) |- ( clase.west);
    \draw[myarrow] (metodos1.north) -- ++(0,0) |- (clase.east);
  \end{tikzpicture}
  \caption{Métodos y atributos de la clase \texttt{Structure}.}
  \label{fig:pyFEM-Structure}
\end{figure}

El constructor de la clase \verb|Structure| recibe seis argumentos de entrada opcionales, uno para cada grado de libertad (tres translaciones y tres rotaciones), los cuales tienen \verb|False| como valor por defecto. Cuando el usuario crea un objeto de esta clase debe indicar qué grados de libertad quiere tener en cuenta para analizar el modelo.\\

En el algoritmo \ref{alg:Structure-init} se presenta el constructor de la clase \verb|Structure|. El constructor de la clase le asigna a los atributos \verb|ux|, \verb|uy|, \verb|uz|, \verb|rx|, \verb|ry| y \verb|rz| los respectivos valores de los argumentos de entrada. A los demás atribututos les asigna un diccionario vacío.\\
\pagebreak
\begin{lstlisting}[language=Python,caption=Constructor de la clase \texttt{Structure}.,label=alg:Structure-init, frame=single]
def __init__(self,ux=False,uy=False,uz=False,rx=False,ry=False,rz=False):
    """
    Instantiate a Structure object

    Parameters
    ----------
    ux : bool
        Flag translaction along 'x' axis activate.
    uy : bool
        Flag translaction along 'y' axis activate.
    uz : bool
        Flag translaction along 'z' axis activate.
    rx : bool
        Flag rotation around 'x' axis activate.
    ry : bool
        Flag rotation around 'y' axis activate.
    rz : bool
        Flag rotation around 'z' axis activate.
    """
    # flag active joint displacements
    self.ux = ux
    self.uy = uy
    self.uz = uz
    self.rx = rx
    self.ry = ry
    self.rz = rz

    # dict materials and sections
    self.materials = {}
    self.sections = {}

    # dict joints and frames
    self.joints = {}
    self.frames = {}

    # dict supports
    self.supports = {}

    # dict load patterns
    self.load_patterns = {}

    # dict displacements
    self.displacements = {}

    # dict reactions
    self.reactions = {}
\end{lstlisting}
\bigskip
El usuario puede describir el modelo con los objetos tipo \verb|Structure| agregándole objetos que representan materiales, secciones transversales, nodos, elementos aporticados, apoyos, patrones de carga, fuerzas aplicadas en los nodos y cargas distribuidas en los elementos aporticados, mediante los métodos que comienzan con \emph{add}.\\

Estos métodos reciben uno o varios argumentos de entrada obligatorios y una serie de argumentos de entrada opcionales para crear los objectos y almacenarlos en los respectivos diccionarios del objeto tipo \verb|Structure|.\\

Dichos métodos son similares entre sí, solo cambia el diccionario al cual se le está agregando la nueva entrada y el objecto que se está creando. Por ejemplo, en el algoritmo \ref{alg:Structure-add_material} se presenta el método \verb|add_material()|. Los argumentos de entrada 
% opcionales
\verb|*args| y \verb|**kwargs| son pasados al constructor de la clase \verb|Material| para crear un objeto tipo \verb|Material|, mientras que el argumento \verb|key| es usado como llave para almacenar dicho objeto en el diccionario \verb|materials|.\\

\begin{lstlisting}[language=Python,caption=Método \texttt{add\_material()} de la clase \texttt{Structure}.,label=alg:Structure-add_material, frame=single]
def add_material(self, key, *args, **kwargs):
    """
    Add a material

    Parameters
    ----------
    key : immutable
        Material's key.
    """
    self.materials[key] = Material(*args, **kwargs)
\end{lstlisting}
\bigskip
En el caso de los métodos \verb|add_section()|, \verb|add_rectangular_section()|, \verb|add_joint()| y \verb|add_load_pattern()|, el diccionario ya no es \verb|materials| sino \verb|sections|, \verb|joints| o \verb|load_| \verb|patterns|, según corresponda, y el objeto a crear ya no es de tipo \verb|Material| sino de tipo \verb|Section|, \verb|RectangularSection|, \verb|Joint| o \verb|LoadPattern|, respectivamente.\\

El método \verb|add_frame()| permite agregar objetos tipo \verb|Frame| de manera similar a como lo hace el método \verb|add_material()|, con la diferencia que este método recibe como argumentos de entrada las llaves con las que fueron creados el nodo cercano, el nodo lejano, el material y la sección transversal.\\

En el algoritmo \ref{alg:Structure-add_frame} se presenta el método \verb|add_frame()|. Las llaves del material, de la sección transversal y de los nodos se utilizan para recuperan los objetos relacionados en los diferentes diccionarios del objeto tipo \verb|Structure| para crea el objeto tipo \verb|Frame|.\\

\begin{lstlisting}[language=Python,caption=Método \texttt{add\_frame()} de la clase \texttt{Structure}.,label=alg:Structure-add_frame, frame=single]
def add_frame(self, key, key_joint_j, key_joint_k, key_material, key_section):
    """
    Add a frame

    Parameters
    ----------
    key : immutable
        Frame's key.
    key_joint_j : immutable
        Joint j's key.
    key_joint_k : immutable
        Joint k's key.
    key_material : immutable
        Material's key.
    key_section : immutable
        Section's key.
    """
    self.frames[key] = Frame(self.joints[key_joint_j], self.joints[key_joint_k], self.materials[key_material], self.sections[key_section])
\end{lstlisting}
\bigskip
El método \verb|add_support()| es similar a los anteriores, con la diferencia que los objeto tipo \verb|Joint| son usados como llaves para almacenar los objeto tipo \verb|Support|, tal como se presenta en el algoritmo \ref{alg:Structure-add_support}.\\

\begin{lstlisting}[language=Python,caption=Método \texttt{add\_support()} de la clase \texttt{Structure}.,label=alg:Structure-add_support, frame=single]
def add_support(self, key_joint, *args, **kwargs):
    """
    Add a support

    Parameters
    ----------
    key_joint : immutable
        Joint's key.
    """
    self.supports[self.joints[key_joint]] = Support(*args, **kwargs)
\end{lstlisting}
\bigskip
Por su parte, los métodos \verb|add_load_at_joint()| y \verb|add_distributed_load()| reciben dos argumentos de entrada obligatorios y una serie de argumentos de entrada  opcionales. El primer argumento de entrada obligatorio es la llave con la que se creó el objeto tipo \verb|LoadPattern| y el segundo es la llave con el que se creó el objeto tipo \verb|Joint| o el objeto tipo \verb|Frame|, respectivamente.\\

En el algoritmo \ref{alg:Structure-add_load_at_joint} se presenta el método \verb|add_load_at_joint()|. Con la llave del patrón de carga se recupera el objeto tipo \verb|LoadPattern| mientras que con la llave del nodo se recupera el objeto tipo \verb|Joint|. El objeto tipo \verb|Joint| y los demás argumentos de entrada opcionales son pasados al método \verb|add_point_load_at_joint()|.\\

\begin{lstlisting}[language=Python,caption=Método \texttt{add\_load\_at\_joint()} de la clase \texttt{Structure}.,label=alg:Structure-add_load_at_joint, frame=single]
def add_load_at_joint(self, key_load_pattern, key_joint, *args, **kwargs):
    """
    Add a point load at joint

    Parameters
    ----------
    key_load_pattern : immutable
        Load pattern's key.
    key_joint : immutable
        Joint's key,
    """
    self.load_patterns[key_load_pattern].add_point_load_at_joint(self.joints[key_joint], *args, **kwargs)
\end{lstlisting}
\bigskip
En el caso del método \verb|add_distributed_loads()|, el objeto tipo \verb|Frame| y los demás argumentos de entrada opcionales son pasados al método \verb|add_distributed_load()| del objeto tipo \verb|LoadPattern|.\\

Cuando el usuario termina de describir el modelo puede ejecutar el método \verb|solve()| de la clase \verb|Structure| para analizarlo. pyFEM soluciona la estructura sometida a los diferentes patrones de carga, almacenando los resultados de los vectores de desplazamientos y fuerzas en los nodos en los atributos \verb|displacements| y \verb|reactions|, respectivamente.\\

% Los métodos \verb|set_load_pattern_displacements()| y \verb|set_load_pattern_reactions()| reciben como argumentos de entrada el objeto tipo \verb|LoadPattern|, el diccionario generado por el método \verb|set_indexes()| y los resultados del análisis del modelo sometido a las cargas definidas en el objeto tipo \verb|LoadPattern|.\\

% En el algoritmo \ref{alg:Structure-set_load_pattern_displacements} se presenta el método \verb|set_load_pattern_displacements()|. El método genera una nueva entrada en el diccionario \verb|displacements|, usando como llave el objeto tipo \verb|LoadPattern|, para almacenar el diccionario \verb|load_pattern_displacements|.\\

% Para cada nodo pyFEM genera una entrada en el diccionario \verb|load_pattern_displacements|, donde las llaves son los objeto tipo \verb|Joint| y los valores son objetos tipo \verb|Displacement|. Los objetos tipo \verb|Displacement| son generados con los desplazamientos del respectivo nodo, al recuperar sus grados de libertad del diccionario \verb|indexes| y con ellos extraer los respectivos desplazamientos del \emph{array} \verb|u|.\\

% \begin{lstlisting}[language=Python,caption=Método \texttt{set\_load\_pattern\_displacements()} de la clase \texttt{Structure}.,label=alg:Structure-set_load_pattern_displacements, frame=single]
% def set_load_pattern_displacements(self, load_pattern, indexes, u):
%     """
%     Set load pattern displacements

%     Parameters
%     ----------
%     load_pattern : LoadPatterns
%         Load pattern.
%     indexes : dict
%         Key value pairs joints and indexes.
%     u : ndarray
%         Displacements.
%     """
%     flag_joint_displacements = self.get_flag_active_joint_displacements()

%     load_pattern_displacements = {}

%     for joint in self.joints.values():
%         joint_indexes = indexes[joint]
%         displacements = flag_joint_displacements.astype(float)
%         displacements[flag_joint_displacements] = u[joint_indexes, 0]
%         load_pattern_displacements[joint] = Displacement(*displacements)

%     self.displacements[load_pattern] = load_pattern_displacements
% \end{lstlisting}
% \bigskip
% En el caso del método \verb|set_load_pattern_reactions()|, en vez de generarse una nueva entrada en el diccionario \verb|displacements|, ésta se genera en el diccionario \verb|reactions|. Esta nueva entrada es un diccionario que relaciona los objetos tipo \verb|Joint| con las reacciones en los apoyos de la estructura, mediante objetos tipo \verb|Reaction|.\\

En las siguientes secciones se presenta la implementación de las clases con las que el usuario puede describir el modelo (\verb|Material|, \verb|Section|, \verb|Joint|, etc.) y después la implementación de los demás métodos de la clase \verb|Structure|.\\

\section{Clases}

Además de la clase \verb|Structure|, definida en el archivo \verb|core.py|, pyFEM define otras clases en los archivos \verb|primitives.py| y \verb|classtools.py|.\\

En el archivo \verb|primitives.py| se definen todas las clases que permiten describir el modelo, es decir:

\begin{multicols}{2}
  \setlength{\columnseprule}{0pt}
  \begin{itemize}
  \item \verb|Material|,
  \item \verb|Section|,
  \item \verb|RectangularSection|,
  \item \verb|Joint|,
  \item \verb|Frame|,
  \item \verb|Support|,
  \item \verb|LoadPattern|,
  \item \verb|PointLoad|,
  \item \verb|DistributedLoad|,
  \item \verb|Displacement|,
  \item \verb|Reaction|.
  \end{itemize}
\end{multicols}

En el archivo \verb|classtools.py| se define la clase \verb|AttrDisplay| y la \emph{metaclase} \verb|UniqueInstance|. La clase \verb|AttrDisplay| implementa una representación de los objetos más cómoda para los usuarios, mientras que la metaclase \verb|UniqueInstance| no permite crear objetos con los mismos atributos de otros objetos de la misma clase.\\

A continuación se presenta la implementación de todas las clases de pyFEM.

\subsection{Material}
La clase \verb|Material| representa un material líneal elástico al definir los valores del módulo de Young y del módulo a cortante.\\

En el algoritmo \ref{alg:Material} se presenta la implementación de la clase \verb|Material|. Se asigna la tupla \verb|('E', 'G')| al atributo \verb|__slots__| de la clase para indicarle a Python que limite la cantidad de atributos que puede tener una instancia. Esto como mecánismo de optimización.\\

Según \cite{lutz2013python}, asignar un diccionario en el espacio de nombres para cada objeto instanciado puede ser costoso, en términos de memoria, si muchas instancias son creadas y solo se requiere un par de atributos. Para ahorrar memoria, en lugar de asignar un diccionario por instancia, Python reserva el espacio suficiente en cada instancia para guardar un valor por cada atributo del \emph{slot}.\\

El constructor de la clase recibe los argumentos de entrada \verb|modulus_elasticity| y \verb|shearing_| \verb|modulus_elasticity|, los cuales tienen $ 0 $ como valor por defecto. Estos valores son pasados a los atributos \verb|E| y \verb|G| del objeto tipo \verb|Material| respectivamente.\\

\begin{lstlisting}[language=Python,caption=Clase \texttt{Material} implementada en el archivo \texttt{primitives.py}.,label=alg:Material, frame=single]
class Material(AttrDisplay):
    """
    Linear elasic material

    Attributes
    ----------
    E : float
        Young's modulus.
    G : float
        Shear modulus.
    """
    __slots__ = ('E', 'G')

    def __init__(self, modulus_elasticity=0, shearing_modulus_elasticity=0):
        """
        Instantiate a Material object

        Parameters
        ----------
        modulus_elasticity : float
            Young's modulus.
        shearing_modulus_elasticity : float
            Shear modulus.
        """
        self.E = modulus_elasticity
        self.G = shearing_modulus_elasticity  
\end{lstlisting}
\bigskip
\subsection{Section}
La clase \verb|Section| representa la sección transversal de los elementos aporticados de manera general, al definir los valores del área transversal, de la constante de torsión y de las inercias principales alrededor de los ejes principales.\\

En el algoritmo \ref{alg:Section} se presenta la implementación de la clase \verb|Section|. Así como se asignó una tupla al atributo \verb|__slots__| de la clase \verb|Material|, como mecánismo de optimización, se asigna una tupla al atributo \verb|__slots__| de la clase \verb|Section| con los elementos \verb|'A'|, \verb|'Ix'|, \verb|'Iy'| y \verb|'Iz'|.\\

El constructor de la clase recibe los argumentos de entrada \verb|area|, \verb|torsion_constant|, \verb|moment_| \verb|inertia_y| y \verb|moment_inertia_z|, los cuales tienen $ 0 $ como valor por defecto. Estos valores son pasados a los atributos \verb|A|, \verb|Ix|, \verb|Iy| y \verb|Iz|, respectivamente.\\

\begin{lstlisting}[language=Python,caption=Clase \texttt{Section} implementada en el archivo \texttt{primitives.py}.,label=alg:Section, frame=single]
class Section(AttrDisplay):
    """
    Cross-sectional area

    Attributes
    ----------
    A : float
        Cross-sectional area.
    Ix : float
        Inertia around axis x-x.
    Iy : float
        Inertia around axis y-y.
    Iz : float
        Inertia around axis z-z.
    """
    __slots__ = ('A', 'Iy', 'Iz', 'Ix')

    def __init__(self, area=0, torsion_constant=0, moment_inertia_y=0, moment_inertia_z=0):
        """
        Instantiate a Section object

        Parameters
        ----------
        area : float
            Cross-sectional area.
        torsion_constant : float
            Inertia around axis x-x.
        moment_inertia_y : float
            Inertia around axis y-y.
        moment_inertia_z : float
            Inertia around axis z-z.
        """
        self.A = area
        self.Ix = torsion_constant
        self.Iy = moment_inertia_y
        self.Iz = moment_inertia_z
\end{lstlisting}

\subsection{RectangularSection}
La clase \verb|RectangularSection| representa la sección transversal de forma rectángular de los elementos aporticados, al definir los valores de la base y del alto de la figura.\\

En el algoritmo \ref{alg:RectangularSection} se presenta la implementación de la clase \verb|RectangularSection|. Esta clase hereda todos los métodos y atributos de la clase \verb|Section|, para evitar duplicar el código a lo largo del programa, al pasar dicha clase como argumento de entrada cuando se construye la clase \verb|RectangularSection|.\\

Al atributo \verb|__slots__| de la clase \verb|RectangularSection| se le asigna una tupla con las entradas \verb|'width'| y \verb|'height'|. Python no solo limita las instancias de esta clase a los atributos \verb|width| y \verb|height|, sino que se extiende a los elementos definidos en el atributo \verb|__slots__| de la clase \verb|Section|.\\

Según \cite{lutz2013python}, como las variables \verb|__slots__| son atributos a nivel de clases, los objetos instanciados adquieren la unión de todas las entradas en todos los atributos \verb|__slots__| de la clase y sus \emph{super clases}.\\

El constructor de la clase recibe los argumentos de entrada \verb|width| y \verb|height|, los cuales no tiene ningún valor por defecto (a diferencia de los argumentos de entrada del constructor de la clase \verb|Section|). Los valores de los argumentos de entrada son asignados a los respectivos atributos de los objeto tipo \verb|RectangularSection|.\\

Se asume que el valor del parámetro \verb|width| corresponde a la dimensión del elemento aporticado de sección transversal rectangular a lo largo del eje $ y $ del sistema de coordenadas local, mientras que el parámetro \verb|heigth| corresponde a la dimensión del elemento aporticado a lo largo del eje $ z $.\\

Teniendo en cuenta esto se cálcula el área, la constante de torsión y los momentos de inercia alrededor de los ejes $ y $ y $ z $. Para calcular la constante de torsión se utiliza la expresión \eqref{eq:torsion-rectangulo}, la misma que se presenta en \cite{escamilla1995microcomputadores},\\

\begin{equation}
  \label{eq:torsion-rectangulo}
  I_{xx} = \left(\frac{1}{3} - 0.21 \frac{a}{b} \left(1 - (1 / 12) (a / b)^4\right)\right) b a ^ 3
\end{equation}
donde \verb|a| es la dimensión menor del rectángulo y \verb|b| su dimensión mayor.\\

Finalmente, las propiedades de la sección transversal son pasadas al constructor de la clase \verb|Section| para asignárselas a los atributos del objeto tipo \verb|RectangularSection|. Esto se hace mediante la función \verb|super()| que trae Python por defecto, la cual genera una referencia a la \emph{clase padre}, en este caso, la clase \verb|Section|.\\

\begin{lstlisting}[language=Python,caption=Clase \texttt{RectangularSection} implementada en el archivo \texttt{primitives.py}.,label=alg:RectangularSection, frame=single]
class RectangularSection(Section):
    """
    Rectangular cross-section

    Attributes
    ----------
    width : float
        Width rectangular cross section.
    height : float
        Height rectangular cross section.
    A : float
        Cross-sectional area.
    Ix : float
        Inertia around axis x-x.
    Iy : float
        Inertia around axis y-y.
    Iz : float
        Inertia around axis z-z.
    """
    __slots__ = ('width', 'height')

    def __init__(self, width, height):
        """
        Instantiate a rectangular section object

        Parameters
        ----------
        width : float
            Width rectangular cross section.
        height : float
            Height rectangular cross section.
        """
        self.width = width
        self.height = height

        a = min(width, height)
        b = max(width, height)
        area = width * height
        torsion_constant = (1/3 - 0.21 * (a / b) * (1 - (1/12) * (a/b)**4)) * b * a ** 3
        moment_inertia_y = (1 / 12) * width * height ** 3
        moment_inertia_z = (1 / 12) * height * width ** 3

        super().__init__(area, torsion_constant, moment_inertia_y, moment_inertia_z)
\end{lstlisting}

\subsection{Joint}
La clase \verb|Joint| representa nodos de la estructura, al definir sus coordenadas en el sistema de coordenadas global.\\

En el algoritmo \ref{alg:Joint} se presenta la implementación de la clase \verb|Joint|. Como mecánismo de optimización, se asigna una tupla con los elementos \verb|x|, \verb|y| y \verb|z| al atributo \verb|__slots__| de la clase para indicarle a Python que limite la cantidad de atributos que puede tener una instancia.\\

El constructor de la clase recibe tres argumentos de entrada opcionales, uno para cada una de las coordenadas, los cuales tienen $ 0 $ como valor por defecto. Estos valores son pasados a los atributos \verb|x|, \verb|y| y \verb|z| del objeto respectivamente.\\

Finalmente, la clase \verb|Joint| implementa el método \verb|get_coordinate()| que generar una \emph{array} con las coordenadas del objeto.\\

\begin{lstlisting}[language=Python,caption=Clase \texttt{Joint} implementada en el archivo \texttt{primitives.py}.,label=alg:Joint, frame=single]
class Joint(AttrDisplay, metaclass=UniqueInstances):
    """
    End of frames

    Attributes
    ----------
    x : float
        X coordinate.
    y : float
        Y coordinate.
    z : float
        Z coordinate.

    Methods
    -------
    get_coordinate()
        Return joint's coordinates.
    """
    __slots__ = ('x', 'y', 'z')

    def __init__(self, x=0, y=0, z=0):
        """
        Instantiate a Joint object

        Parameters
        ----------
        x : float
            X coordinate.
        y : float
            Y coordinate.
        z : float
            Z coordinate.
        """
        self.x = x
        self.y = y
        self.z = z

    def get_coordinate(self):
        """Get coordinates"""
        return np.array([self.x, self.y, self.z])
\end{lstlisting}

\subsection{Frame}
La clase \verb|Frame| representa los elementos aporticados de la estructura, al definir sus nodos, material y sección tranversal. En la figura \ref{fig:pyFEM-Frame} se presentan los métodos y atributos de esta clase.\\

\begin{figure}[ht]
  \centering
  \begin{tikzpicture}
    \node (clase) [class] {\textbf{Frame}};

    % attrs
    \node (atrrs) [left=of clase, xshift=-1cm]{};

    % atributos
    \matrix [
    matrix of nodes,
    below=of atrrs,
    nodes={
      anchor=center
    }] (atributos) {
      \node {joint\_j,};\\
      \node {joint\_k,};\\
      \node {material,};\\
      \node {section};\\
    };
    
    % methods
    \node (meths) [right=of clase, xshift=1cm]{};
    \matrix [
    matrix of nodes,
    below=of meths,
    nodes={
      anchor=center
    }] (metodos) {
      \node {get\_length(),}; & \node {get\_direction\_cosines(),};\\
      \node {get\_rotation(),}; & \node {get\_rotation\_matrix(),};\\
      \node {get\_local\_stiffness\_matrix(),}; & \node {get\_global\_stiffness\_mtrix()};\\
    };
    
    \node [atributos, fit=(atributos)] {};
    \node [metodos, fit=(metodos)] {};

    \draw[myarrow] (atributos.north) -- ++(0,0) |- ( clase.west);
    \draw[myarrow] (metodos.north) -- ++(0,0) |- (clase.east);
  \end{tikzpicture}
  \caption{Métodos y atributos de la clase \texttt{Frame}.}
  \label{fig:pyFEM-Frame}
\end{figure}

En la figura \ref{fig:space-frame} se presenta un elemento aporticado \emph{i} con sus nodos \emph{j} y \emph{k} empotrados. El sistema de coordenadas local del elemento tiene como origen el nodo \emph{j}; el eje $ x $ coincide con el eje centroidal del elemento y es positivo en el sentido del nodo \emph{j} al nodo \emph{k}.\\

Los ejes $ y $ y $ z $ son los ejes principales del elemento de manera que los planos $ xy $ y $ zx $ son los planos principales de flexión. Se asume que el centro de cortante y el centroide del elemento coinciden de tal forma que la flexión y la torsión se presentan una independiente de la otra.\\

Los grados de libertad se numeran del 1 al 12, empezando por las translaciones y las rotaciones del nodo \emph{j}, tomados en orden $ x $, $ y $, $ z $ respectivamente.\\

\begin{figure}[ht]
  \centering
  \begin{tikzpicture}[coords]
    % points
    \dpoint{a}{0}{0}{0};
    \dpoint{b}{4}{0}{0};

    \dpoint{alabel}{0}{0}{0.25};
    \dpoint{blabel}{4}{0}{0.25};

    \dpoint{beamlabel}{2}{0}{0.25};

    \dpoint{jux}{-1}{0}{0};
    \dpoint{juy}{0}{-1.5}{0.25};   
    \dpoint{juz}{0}{0}{-1.25};

    \dpoint{jrx}{-2}{0}{0};
    \dpoint{jry}{0}{-2.75}{0.25};   
    \dpoint{jrz}{0}{0}{-2.5};

    \dpoint{kux}{4.5}{0}{0};
    \dpoint{kuy}{4}{-1.5}{0.25};   
    \dpoint{kuz}{4}{0}{-1.25};

    \dpoint{krx}{5.25}{0}{0};
    \dpoint{kry}{4}{-2.75}{0.35};   
    \dpoint{krz}{4}{0}{-2.5};   
   
    % beams
    \dbeam{1}{a}{b};

    \dnotation{6}{beamlabel}{\emph{i}};

    % supports
    \dsupport{2}{a}[yz];
    \dsupport{2}{b}[yz];

    \dnotation{1}{alabel}{\emph{j}};
    \dnotation{1}{blabel}{\emph{k}};
    % restrains
    % j
    \dload{1}{a}[270][0][1];
    \dload{1}{a}[270][90][1][0.5];   
    \dload{1}{a}[180][180][1][0.5];

    \dload{3}{a}[270][0][1][1.25];
    \dload{3}{a}[270][90][1][1.75];   
    \dload{3}{a}[180][180][1][1.75];

    \dnotation{1}{jux}{1};
    \dnotation{1}{juy}{2};
    \dnotation{1}{juz}{3};

    \dnotation{1}{jrx}{4};
    \dnotation{1}{jry}{5};
    \dnotation{1}{jrz}{6};   
    
    % k
    \dload{2}{b}[90][0][1];
    \dload{1}{b}[270][90][1][0.5];   
    \dload{1}{b}[180][180][1][0.5];

    \dload{4}{b}[90][0][1][1.25];
    \dload{3}{b}[270][90][1][1.75];   
    \dload{3}{b}[180][180][1][1.75];

    \dnotation{1}{kux}{7};
    \dnotation{1}{kuy}{8};
    \dnotation{1}{kuz}{9};

    \dnotation{1}{krx}{10};
    \dnotation{1}{kry}{11};
    \dnotation{1}{krz}{12};
    
    % axes
    \dscaling{3}{1};
    \daxis{1}{0, 0, 0};
  \end{tikzpicture}
  \caption{Elemento aporticado y su sistema de coordenadas local.}
  \label{fig:space-frame}
\end{figure}

Como mecánismo de optimización, se asigna una tupla con los elementos \verb|joint_j|, \verb|joint_k|, \verb|material| y \verb|section| al atributo \verb|__slots__| de la clase para indicarle a Python que limite la cantidad de atributos que puede tener una instancia.\\

En el algoritmo \ref{alg:Frame-init} se presenta el constructor de la clase \verb|Frame|. El constructor de la clase \verb|Frame| recibe cuatro argumentos de entrada; para el nodo cercano, el nodo lejano, el material y la sección, los cuales tiene \verb|None| como valor por defecto. Los argumentos de entrada son pasados a los atributos \verb|joint_j|, \verb|joint_k|, \verb|material| y \verb|section| respectivamente.\\\

\begin{lstlisting}[language=Python,caption=Constructor de la clase \texttt{Frame}.,label=alg:Frame-init, frame=single]
def __init__(self, joint_j=None, joint_k=None, material=None, section=None):
    """
    Instantiate a Frame object

    Parameters
    ----------
    joint_j : Joint
        Near Joint object.
    joint_k : Joint
        Far Joint object.
    material : Material
        Material object.
    section : Section
        Section object.
    """
    self.joint_j = joint_j
    self.joint_k = joint_k
    self.material = material
    self.section = section
\end{lstlisting}
\bigskip
A continuación se presentan los métodos de la clase \verb|Frame|, con los cuales se puede, entre otras cosas, calcular la matriz de rigidez de los elementos tipo \verb|Frame|.\\

\subsubsection{get\_length()}
El método \verb|get_length()| de la clase \verb|Frame| permite calcular la longitud de los elementos aporticados representado por objetos tipo \verb|Frame|.\\

En el algoritmo \ref{alg:Frame-get_length} se presenta la implementación del método \verb|get_length()|. El método calcula la distancia que hay entre las coordenadas de los nodos del elemento aporticado. Para esto llama la función \verb|distance.euclidean()| con las coordenadas de los objeto tipo \verb|Joint|, las cuales obtiene con el método \verb|get_coordinate()| (véase el algoritmo \ref{alg:Joint}).\\

Según \cite{2020SciPy-NMeth}, esta función calcula la distancia euclidiana entre dos \emph{arrays} $ u $ y $ v $ de una dimensión como
\begin{equation}
  {||u-v||}_2 = \left(\sum{w_i |(u_i - v_i)|^2}\right)^{1/2}
\end{equation}

donde \verb|w| es un \emph{array} que toma para cada entrada un peso de $ 1 $ por defecto.\\ 
\pagebreak

\begin{lstlisting}[language=Python,caption=Método \texttt{get\_length()} de la clase \texttt{Frame}.,label=alg:Frame-get_length, frame=single]
def get_length(self):
    """Get length"""
    return distance.euclidean(self.joint_j.get_coordinate(), self.joint_k.get_coordinate())
\end{lstlisting}

\subsubsection{get\_direction\_cosines()}
El método \verb|get_direction_cosines()| de la clase \verb|Frame| permite calcular los cosenos directores del eje $ x $ del sistema de coordenadas local de los elementos aporticados, representados por objetos tipo \verb|Frame|, en el sistema de coordenadas global.\\

En el algoritmo \ref{alg:Frame-get_direction_cosines} se presenta la implementación del método \verb|get_direction_cosines()|. El método resta las coordenadas de los nodos del elemento aporticado y almacena el resultado en la variable \verb|vector|. Después divide cada uno de los elementos de \verb|vector| por la norma de dicha variable, calculada mediante la función \verb|linalg.norm()|.\\

Según \cite{harris2020array}, esta función calcula la norma de un vector como

\begin{equation}
  {||A||}_F = \left[\sum_{i,j} abs(a_{i,j})^2\right]^{1/2}
\end{equation}

donde $ a_{i,j} $ es el elemento del vector en la posición $ i, j $.\\

\begin{lstlisting}[language=Python,caption=Método \texttt{get\_direction\_cosines()} de la clase \texttt{Frame}.,label=alg:Frame-get_direction_cosines, frame=single]
def get_direction_cosines(self):
    """Get direction cosines"""
    vector = self.joint_k.get_coordinate() - self.joint_j.get_coordinate()

    return vector / linalg.norm(vector)
\end{lstlisting}

\subsubsection{get\_rotation()}
El método \verb|get_rotation()| de la clase \verb|Frame| permite calcular la rotación de los elementos aporticados, representados por objetos tipo \verb|Frame|, con respecto al sistema de coordenadas global.\\

Según el \emph{teorema de rotación de Euler} (véase \cite{euler_rotations}), siempre es posible encontrar un diámetro de una esfera cuya posición es la misma después de rotar la esfera alredor de su centro, por lo que cualquier secuencia de rotaciones de un sistema coordenado tridimensional es equivalente a una única rotación alrededor de un eje que pase por el origen.\\

El ángulo $ \theta $ y el vector $ n $  que definen la rotación del eje $ x $ del sistema de coordenadas global hacia el eje $ x_m $ del sistema de coordenadas local de un elemento aporticado se puede calcular como
\begin{equation}
  \label{eq:quaternion_from_axis}
  \begin{aligned}
    \mathbf{n} &= (1, 0, 0) \times \mathbf{x_m} \\
    \theta &= \arccos((1, 0, 0) \cdot \mathbf{x_m})
  \end{aligned}
\end{equation}

Según \cite{dunn20023d}, la rotación de un sistema de coordenadas tridimensionales alrededor del eje $ \mathbf{n} $ una cantidad $ \theta $ se puede describir mediante un \emph{cuaternión} como

\begin{equation}
  \label{eq:quaternion}
  \mathbf{q} =
  \begin{bmatrix}
    cos(\theta/2) & sin(\theta/2) \mathbf{n}
  \end{bmatrix}
\end{equation}

y se puede obtener la matriz de rotación a partir de un cuaternión de la siguiente manera
\begin{equation}
  \label{eq:matriz-rotacion}
  \mathbf{R} =
  \begin{bNiceMatrix}
    1 - 2y^2-2z^2 & 2xy + 2wz & 2xz - 2wy \\
    2xy - 2wz & 1 - 2x^2-2z^2 & 2yz + 2wx \\
    2xz + 2wy & 2yz-2wx & 1 - 2x^2 - 2y^2
  \end{bNiceMatrix}
\end{equation}

donde $ w $ es la parte escalar y $ x $, $ y $ y $ z $ la parte vectorial del cuaternión.\\

En el algoritmo \ref{alg:Frame-get_rotation} se presenta la implementación del método \verb|get_rotation()|. El método calcula el \emph{cuaternión} que representa la rotación del eje $ x $ del sistena de coordenadas global hacía el eje $ x $ del sistema de coordenadas local del elemento aporticado.\\

Para esto, se almacena el eje global $ x $ y el eje local $ x $ en las variables \verb|v_from| y \verb|v_to| respectivamente. El eje global $ x $ es igual a $ (1, 0, 0) $ mientras que el eje local $ x $ se calcula mediante el método \verb|get_direction_cosines()| (véase el algoritmo \ref{alg:Frame-get_direction_cosines}).\\

Después, se verifica si las variables \verb|v_from| y \verb|v_to| son iguales entre sí, o si una variable es el inverso aditivo de la otra.\\

En el caso que las variables sean iguales entre sí, no hay rotación y el ángulo $ \theta $ es igual a cero, por lo tanto el cuaternión es igual a $ (1, 0 \times \mathbf{n}) $ (véase la ecuación \ref{eq:quaternion}). En caso contrario, el ángulo $ \theta $ que describe la rotación es igual a 180\degree. Como eje se asume el eje global $ z $, por lo que el cuaternión es igual a $ (0, 1 \times (0, 0, 1)) $.\\

Si las variables \verb|v_from| y \verb|v_to| no son iguales entre sí, y una no es el inverso aditivo de la otra, entonces se calcula el eje y el ángulo que describen la rotación del eje global $ x $ hacia el eje local $ x $ del elemento aporticado, aplicando las expresiones de ecuación \eqref{eq:quaternion_from_axis}.\\

Para calcular el eje se halla el producto cruz entre el eje global $ x $ y el eje local $ x $ mediante la función \verb|cross()|. Después se normaliza el resultado dividiendolo por su norma, con ayuda de la función \verb|linalg.norm()|.\\

El ángulo se halla calculando el arcocoseno del producto punto entre el eje global $ x $ y el eje local $ x $. Esto se calcula mediante las funciones \verb|dot()| y \verb|arccos()| respectivamente.\\

Finalmente, se aplican las expresiones de la ecuación \eqref{eq:quaternion} para crear un objeto tipo \verb|Rotation|, mediante la función \verb|Rotation.from_quat()|.\\

Según \cite{2020SciPy-NMeth}, la función \verb|from_quat()| permite crear objetos tipo \verb|Rotation|, los cuales son una interfaz para inicializar y representar rotaciones en el espacio, mediante un cuaternión.\\

\begin{lstlisting}[language=Python,caption=Método \texttt{get\_rotation()} de la clase \texttt{Frame}.,label=alg:Frame-get_rotation, frame=single]
def get_rotation(self):
    """Get rotation"""
    v_from = np.array([1, 0, 0])
    v_to = self.get_direction_cosines()

    if np.all(v_from == v_to):
        return Rotation.from_quat([0, 0, 0, 1])

    elif np.all(v_from == -v_to):
        return Rotation.from_quat([0, 0, 1, 0])

    else:
        w = np.cross(v_from, v_to)
        w = w / linalg.norm(w)
        theta = np.arccos(np.dot(v_from, v_to))

        return Rotation.from_quat([x * np.sin(theta/2) for x in w] + [np.cos(theta/2)])
\end{lstlisting}

\subsubsection{get\_rotation\_matrix()}

El método \verb|get_rotation_matrix()| de la clase \verb|Frame| permite calcular la matriz de transformación de rotación de los elementos aporticados, representados por objetos tipo \verb|Frame|, con respecto al sistema de coordenadas global.\\

Según \cite{weaver1990matrixanalysis}, la matriz de transformación de rotación $ R_T $ para un elemento aporticado es
\begin{equation}
  \mathbf{R_T} =
  \begin{bNiceMatrix}
    \mathbf{R} & \mathbf{0} & \mathbf{0} & \mathbf{0} \\
    \mathbf{0} & \mathbf{R} & \mathbf{0} & \mathbf{0} \\
    \mathbf{0} & \mathbf{0} & \mathbf{R} & \mathbf{0} \\
    \mathbf{0} & \mathbf{0} & \mathbf{0} & \mathbf{R} \\
  \end{bNiceMatrix}
\end{equation}

donde $ \mathbf{R} $ es la matriz de rotación presentada en \eqref{eq:matriz-rotacion}.\\

En el algoritmo \ref{alg:Frame-get_rotation_matrix} se presenta la implementación del método \verb|get_rotation_matrix()|. El método recibe como argumentos de entrada un \emph{array} que indica para cada grado de libertad si está o no activado. Según los grados de libertad activados se genera la matriz de transformación de rotación de los elementos aporticados.\\

La matriz de transformación de rotación se genera a partir de la matriz de rotación del elemento aporticado, calculada con el método \verb|get_rotation().as_dcm()| de la clase \verb|Frame| (véase el algoritmo \ref{alg:Frame-get_rotation}), y la función \verb|bsr_matrix()|.\\

Según \cite{2020SciPy-NMeth}, el método \verb|as_dcm()| de la clase \verb|Rotation| calcula la matriz de rotación de los objetos tipo \verb|Rotation| y la función \verb|bsr_matrix()| crea matrices dispersas con submatrices densas describiendolas en la representación estándar \verb|(data, indices, indptr)|. En dicha representación los índices de la columna de cada submatriz en la fila \verb|i| de la matriz dispersa están almacenados en \verb|indices[indptr[i]:indptr[i+1]]| y los valores correspondientes almacenados en \verb|data[indptr[i]:indptr[i+1]]|.\\

En las variable \verb|indptr| e \verb|indices| se almacenan los \emph{arrays} \verb|[0, 1, 2]| y \verb|[0, 1]| respectivamente. Estas variables describen en la representación estándar las posiciones que ocupan dos submatrices en la diagonal principal de una matriz dispersa.\\

En la primer fila hay una submatriz en la primer columna (\verb|indices[indptr[0]:indptr[1]]|\ \verb|->|\verb|indices[0:1]|\verb|->|\verb|indices[0]->0|) y en la segunda fila hay una submatriz en la segunda columna (\verb|indices[|\verb|indptr[1]:indptr[2]]|\verb|->|\verb|indices[1:2]|\verb|->|\verb|indices[1]->1|).\\

Inicialmente se calcula la matriz de transformación de rotación para un solo nodo. Después se seleccionan las filas y las columnas de esta matriz asociadas a los grados de libertad activados. Finalmente se crea toda la matriz de transformación de rotación duplicando los valores seleccionados.\\

Para crear la matriz de transformación de rotacion para un solo nodo se duplica la matriz de rotación del elemento aporticado, mediante la función \verb|tile|, y se almacena en la variable \verb|data|. Después se pasa junto con las variables \verb|indptr| e \verb|indices| a la función \verb|bsr_matrix()|.\\

La matriz de transformación de rotación del elemento aporticado se crea al indicar dos matrices de transformación de rotación para un solo nodo en la diagonal principal de una matriz dispersa, después de haber seleccionado las filas y columnas asociadas a los grados de libertad activados. El tamaño de la matriz de transformación de rotación se calcula en función de la cantidad de grados de libertad activados.\\

\begin{lstlisting}[language=Python,caption=Método \texttt{get\_rotation\_matrix()} de la clase \texttt{Frame}.,label=alg:Frame-get_rotation_matrix, frame=single]
def get_rotation_matrix(self, flag_active_joint_displacements):
    """
    Get rotation matrix

    Parameters
    ----------
    flag_active_joint_displacements : array
        Flags active joint's displacements
    """
    # rotation as direction cosine matrix
    indptr = np.array([0, 1, 2])
    indices = np.array([0, 1])
    data = np.tile(self.get_rotation().as_dcm(), (2, 1, 1))

    # matrix rotation for a joint
    t1 = bsr_matrix((data, indices, indptr), shape=(6, 6)).toarray()

    flag_active_joint_displacements = np.nonzero(flag_active_joint_displacements)[0]
    n = 2 * np.size(flag_active_joint_displacements)
    
    t1 = t1[flag_active_joint_displacements[:, None], flag_active_joint_displacements]
    data = np.tile(t1, (2, 1, 1))

    return bsr_matrix((data, indices, indptr), shape=(n, n)).toarray()
\end{lstlisting}

\subsubsection{get\_local\_stiffness\_matrix()}
El método \verb|get_local_stiffness_matrix()| de la clase \verb|Frame| permite calcular la matriz de rigidez de los elementos aporticados, representados por objetos tipo \verb|Frame|, con respecto al sistema de coordenadas local.\\

Según \cite{weaver1990matrixanalysis}, \eqref{eq:matriz-rigidez-elemento-portico} es la matrix de rigidez del elemento aporticado en coordenadas locales, donde $ E $ es el módulo de Young y $ G $ es el módulo de elasticidad a cortante del material, $ L $ es la longitud del elemento y $ A_x $, $ I_x $, $ I_y $ e $ I_z $ son el área, la constante de torsión y los momentos principales de inercia de la sección transversal.\\

\begin{equation}
  \begin{bNiceArray}{CCCCCCCCCCCC}[small,
    first-row,
    first-col,
    code-for-first-row = \mathbf{\arabic{jCol}},
    code-for-first-col = \mathbf{\arabic{iRow}}
    ]
    & & & & & & & & & & & & \\
    & \frac{EA_x}{L} & 0 & 0 & 0 & 0 & 0 & -\frac{EA_x}{L} & 0 & 0 & 0 & 0 & 0 \\
    & & \frac{12EI_z}{L^3} & 0 & 0 & 0 & \frac{6EI_z}{L^2} & 0 & -\frac{12EI_z}{L^3} & 0 & 0 & 0 & \frac{6EI_z}{L^2} \\
    & & & \frac{12EI_y}{L^3} & 0 & -\frac{6EI_y}{L^2} & 0 & 0 & 0 & -\frac{12EI_y}{L^3} & 0 & -\frac{6EI_y}{L^2} & 0 \\
    & & & & \frac{GI_x}{L} & 0 & 0 & 0 & 0 & 0 & -\frac{GI_x}{L} & 0 & 0 \\
    & & & & & \frac{4EI_y}{L} & 0 & 0 & 0 & \frac{6EI_y}{L^2} & 0 & \frac{2EI_y}{L} & 0 \\
    & & & & & & \frac{4EI_z}{L} & 0 & -\frac{6EI_z}{L^2} & 0 & 0 & 0 & \frac{2EI_z}{L} \\
    & & & & & & & \frac{EA_x}{L} & 0 & 0 & 0 & 0 & 0 \\
    & & & & & & & & \frac{12EI_z}{L^3} & 0 & 0 & 0 & -\frac{6EI_z}{L^2} \\
    & & & & & & & & & \frac{12EI_y}{L^3} & 0 & \frac{6EI_y}{L^2} & 0 \\
    & & & & & & & & & & \frac{GI_x}{L} & 0 & 0 \\
    & & & & & & & & & & & \frac{4EI_y}{L} & 0 \\
    & \emph{sim.} & & & & & & & & & & & \frac{4EI_z}{L}
    \label{eq:matriz-rigidez-elemento-portico}
  \end{bNiceArray}
\end{equation}

En el algoritmo \ref{alg:Frame-get_local_stiffness_matrix} se presenta la implementación del método \verb|get_local_stiffness_|\ \verb|matrix()|. El método recibe como argumentos de entrada un \emph{array} que indica para cada grado de libertad si está o no activado. Según los grados de libertad activados se genera la matriz de rigidez en el sistema de coordenadas local.\\

La matriz de rigidez en el sistema de coordenadas local se calcula con los atributos del material, de la sección tranversal, de los nodos de los elementos aporticados y la función \verb|coo_matrix()|.\\

Según \cite{2020SciPy-NMeth}, con la función \verb|coo_matrix()| se pueden crear matrices dispersas en el formato coordenado, también conocido como el formato \emph{ijv} o el formato triple. En este formato los indices de las filas, de las columnas y los respectivos valores de la matriz dispersa son almacenados en tres \emph{arrays} independientes \verb|i|, \verb|j| y \verb|data| de tal manera que se cumpla \verb|A[i[k], j[k]] = data[k]|.\\

En las variables \verb|length|, \verb|e|, \verb|iy| y \verb|iz| se almacenan la longitud del elemento aporticado (véase el algoritmo \ref{alg:Frame-get_length}), el módulo de Young del material y las inercias principales de la sección transversal con respecto a los ejes $ y $ y $ z $ del sistema de coordenadas local.\\

Después se calcula el módulo de Young dividido entre varias potencias de la longitud del elemento aporticado y los resultados se almacena en las variables \verb|el|, \verb|el2| y \verb|el3| respectivamente. El número al final del nombre de estas variables indica la potencia de la longitud del elemento.\\

Con estas variables se calculan los términos $ EA / L $, $ GI_x / L $, $ EI_y / L $, $ EI_z / L $, $ 6EI_y / L^2 $, $ 6EI_z / L^2 $, $ 12EI_y / L^3 $ y $ 12EI_z / L^3 $, los cuales son almacenados en las variables \verb|ael|, \verb|gjl|, \verb|e_iy_l|, \verb|e_iz_l|, \verb|e_iy_l2|, \verb|e_iz_l2|, \verb|e_iy_l3| y \verb|e_iz_l3|, respectivamente.\\

Con estas variables se describe la matriz de rigidez en coordenadas locales como una matriz dispersa en el formato \emph{ijv}. Los indices de las filas y las columnas se almacenan en los \emph{arrays} \verb|rows| y \verb|cols|, respectivamente, mientras que los valores de la matriz se almacenan en el \emph{array} \verb|data|.\\

Por ejemplo, para describir los términos de la matriz de rigidez asociados a las solicitaciones axiales, se le pasa a los \emph{arrays} \verb|rows| y \verb|cols| los valores \verb|[0, 6, 0, 6]| y \verb|[0, 6, 6, 0]|, y al \emph{array} \verb|data| se le pasa los valores \verb|[ael, ael, -eal, -eal]|. Para los otros términos de la matriz de rigidez se procede de manera similar.\\

Finalmente, se genera la matriz de rigidez del elemento aporticado en el sistema de coordenadas local y se seleccionan las filas y columnas asociadas a los grados de libertad activados.\\

\begin{lstlisting}[language=Python,caption=Método \texttt{get\_local\_stiffness\_matrix()} de la clase \texttt{Frame}.,label=alg:Frame-get_local_stiffness_matrix, frame=single]
def get_local_stiffness_matrix(self, active_joint_displacements):
    """
    Get local stiffness matrix

    Parameters
    ----------
    active_joint_displacements : array
        Flags active joint's displacements
    """
    length = self.get_length()

    e = self.material.E

    iy = self.section.Iy
    iz = self.section.Iz

    el = e / length
    el2 = e / length ** 2
    el3 = e / length ** 3

    ael = self.section.A * el
    gjl = self.section.Ix * self.material.G / length

    e_iy_l = iy * el
    e_iz_l = iz * el

    e_iy_l2 = 6 * iy * el2
    e_iz_l2 = 6 * iz * el2

    e_iy_l3 = 12 * iy * el3
    e_iz_l3 = 12 * iz * el3

    rows = np.empty(40, dtype=int)
    cols = np.empty(40, dtype=int)
    data = np.empty(40)

    # AE / L
    rows[:4] = np.array([0, 6, 0, 6])
    cols[:4] = np.array([0, 6, 6, 0])
    data[:4] = np.array([ael, ael, -ael, -ael])

    # GJ / L
    rows[4:8] = np.array([3, 9, 3, 9])
    cols[4:8] = np.array([3, 9, 9, 3])
    data[4:8] = np.array([gjl, gjl, -gjl, -gjl])

    # 12EI / L^3
    rows[8:12] = np.array([1, 7, 1, 7])
    cols[8:12] = np.array([1, 7, 7, 1])
    data[8:12] = np.array([e_iz_l3, e_iz_l3, -e_iz_l3, -e_iz_l3])

    rows[12:16] = np.array([2, 8, 2, 8])
    cols[12:16] = np.array([2, 8, 8, 2])
    data[12:16] = np.array([e_iy_l3, e_iy_l3, -e_iy_l3, -e_iy_l3])

    # 6EI / L^2
    rows[16:20] = np.array([1, 5, 1, 11])
    cols[16:20] = np.array([5, 1, 11, 1])
    data[16:20] = np.array([e_iz_l2, e_iz_l2, e_iz_l2, e_iz_l2])

    rows[20:24] = np.array([5, 7, 7, 11])
    cols[20:24] = np.array([7, 5, 11, 7])
    data[20:24] = np.array([-e_iz_l2, -e_iz_l2, -e_iz_l2, -e_iz_l2])

    rows[24:28] = np.array([2, 4, 2, 10])
    cols[24:28] = np.array([4, 2, 10, 2])
    data[24:28] = np.array([-e_iy_l2, -e_iy_l2, -e_iy_l2, -e_iy_l2])

    rows[28:32] = np.array([4, 8, 8, 10])
    cols[28:32] = np.array([8, 4, 10, 8])
    data[28:32] = np.array([e_iy_l2, e_iy_l2, e_iy_l2, e_iy_l2])

    # 4EI / L
    rows[32:36] = np.array([4, 10, 5, 11])
    cols[32:36] = np.array([4, 10, 5, 11])
    data[32:36] = np.array([4 * e_iy_l, 4 * e_iy_l, 4 * e_iz_l, 4 * e_iz_l])

    rows[36:] = np.array([10, 4, 11, 5])
    cols[36:] = np.array([4, 10, 5, 11])
    data[36:] = np.array([2 * e_iy_l, 2 * e_iy_l, 2 * e_iz_l, 2 * e_iz_l])

    k = coo_matrix((data, (rows, cols)), shape=(12, 12)).toarray()

    active_frame_displacement = np.nonzero(np.tile(active_joint_displacements, 2))[0]

    return k[active_frame_displacement[:, None], active_frame_displacement]
\end{lstlisting}

\subsubsection{get\_global\_stiffness\_matrix()}

El método \verb|get_global_stiffness_matrix()| de la clase \verb|Frame| permite calcular la matriz de rigidez de los elementos aporticados, representados por objetos tipo \verb|Frame|, con respecto al sistema de coordenadas global.\\

Según \cite{weaver1990matrixanalysis}, la matriz de rigidez de los elementos aporticados con respecto al sistema de coordenadas global se puede calcular como
\begin{equation}
  \mathbf{S_{MS} = R_T S_M R_T^T}
  \label{eq:matriz-rigidez-global}
\end{equation}

donde $ \mathbf{R_T} $ y $ \mathbf{S_M} $ son la matriz de transformación de rotación y la matriz de rigidez en el sistema de coordenadas local del elemento aporticado.\\

En el algoritmo \ref{alg:Frame-get_global_stiffness_matrix} se presenta la implementación del método \verb|get_global_stiffness_|\ \verb|matrix()|. El método recibe como argumento de entrada un \emph{array} que indica para cada grado de libertad si está o no activado. Según los grados de libertad activados se genera la matriz de rigidez en el sistema de coordenadas global.\\

La matriz de rigidez en el sistema de coordenadas global se calcula con la matriz de rigidez en el sistema de coordenadas local y la matriz de transformación de rotación del elemento aporticado. Estas matrices son calculadas con los métodos \verb|get_matrix_rotation()| (véase el algoritmo \ref{alg:Frame-get_rotation_matrix}) y \verb|get_local_stiffness_matrix()| (véase el algoritmo \ref{alg:Frame-get_local_stiffness_matrix}), y almacenadas en las variables \verb|k| y \verb|t|, respectivamente.\\

Finalmente, se operan las matrices obtenidas según \eqref{eq:matriz-rigidez-global} para calcular la matriz de rigidez del elemento aporticado en el sistema de coordenadas global con las funciones \verb|dot()| y \verb|transpose()|.\\

\begin{lstlisting}[language=Python,caption=Método \texttt{get\_global\_stiffness\_matrix()} de la clase \texttt{Frame}.,label=alg:Frame-get_global_stiffness_matrix, frame=single]
def get_global_stiffness_matrix(self, active_joint_displacements):
    """
    Get the global stiffness matrix

    Parameters
    ----------
    active_joint_displacements : array
        Flags active joint's displacements
    """
    k = self.get_local_stiffness_matrix(active_joint_displacements)
    t = self.get_rotation_matrix(active_joint_displacements)

    return np.dot(np.dot(t, k), np.transpose(t))
\end{lstlisting}

\subsection{Support}
La clase \verb|Support| representa los apoyos de la estructura, al establecer los desplazamientos restringidos de los nodos.\\

En el algortimo \ref{alg:Support} se presenta la implementación de la clase \verb|Support|. Como mecánismo de optimización, se asigna una tupla con los elementos \verb|'ux'|, \verb|'uy'|, \verb|'uz'|, \verb|'rx'|, \verb|'ry'| y \verb|'rz'| al atributo \verb|__slots__| de la clase para indicarle a Python que limite la cantidad de atributos que puede tener una instancia.\\

El constructor de la clase recibe seis argumentos de entrada opcionales, para cada uno de los grados de libertad, los cuales tienen \verb|False| como valor por defecto. El usuario debe indicar qué grados de libertad están restrigidos.\\

Finalmente, la clase \verb|Support| implementa el método \verb|get_restrains()| que genera un \emph{array} que indica para cada grado de libertad activado si está o no restrigido.\\

\begin{lstlisting}[language=Python,caption=Clase \texttt{Support} implementada en el archivo \texttt{primitives.py}.,label=alg:Support, frame=single]
class Support(AttrDisplay):
    """
    Point of support

    Attributes
    ----------
    ux : bool
        Flag restrain x-axis translation.
    uy : bool
        Flag restrain y-axis translation.
    uz : bool
        Flag restrain z-axis translation.
    rx : bool
        Flag restrain x-axis rotation.
    ry : bool
        Flag restrain y-axis rotation.
    rz : bool
        Flag restrain z-axis rotation.

    Methods
    -------
    get_restrains()
        Get flag restrains.
    """

    __slots__ = ('ux', 'uy', 'uz', 'rx', 'ry', 'rz')
    
    def __init__(self, ux=False, uy=False, uz=False, rx=False, ry=False, rz=False):
        """
        Instantiate a Support object

        Parameters
        ----------
        ux : bool
            Flag restrain x-axis translation.
        uy : bool
            Flag restrain y-axis translation.
        uz : bool
            Flag restrain z-axis translation.
        rx : bool
            Flag restrain x-axis rotation.
        ry : bool
            Flag restrain y-axis rotation.
        rz : bool
            Flag restrain z-axis rotation.
        """
        self.ux = ux
        self.uy = uy
        self.uz = uz
        self.rx = rx
        self.ry = ry
        self.rz = rz

    def get_restrains(self, flag_joint_displacements):
        """
        Get restrains

        Attributes
        ----------
        flag_joint_displacements : array
            Flag active joint displacements.
        """
        return np.array([getattr(self, name) for name in self.__slots__])[flag_joint_displacements]
\end{lstlisting}

\subsection{LoadPattern}

La clase \verb|LoadPattern| representa los patrones de carga a los que está sometida la estructura, al establecer la magnitud de las fuerzas y las cargas distribuidas que actúan en los nodos y en los elementos aporticados, respectivamente. En la figura \ref{fig:pyFEM-LoadPattern} se presentan los métodos y atributos de esta clase.\\

\begin{figure}[ht]
  \centering
  \begin{tikzpicture}
    \node (clase) [class] {\textbf{LoadPattern}};

    % attrs
    \node (atrrs) [left=of clase, xshift=-1cm]{};

    % atributos
    \matrix [
    matrix of nodes,
    below=of atrrs,
    nodes={
      anchor=center
    }] (atributos) {
      \node {loads\_at\_joints,};\\
      \node {distributed\_loads};\\
    };
    
    % methods
    \node (meths) [right=of clase, xshift=1cm]{};
    \matrix [
    matrix of nodes,
    below=of meths,
    nodes={
      anchor=center
    }] (metodos) {
      \node {add\_point\_load\_at\_joint(),};\\
      \node {add\_distributed\_load(),};\\
      \node {get\_number\_point\_loads\_at\_joints(),};\\
      \node {get\_number\_distributed\_loads(),};\\
      \node {get\_f(),};\\
      \node {get\_f\_fixed()};\\
    };
    
    \node [atributos, fit=(atributos)] {};
    \node [metodos, fit=(metodos)] {};

    \draw[myarrow] (atributos.north) -- ++(0,0) |- ( clase.west);
    \draw[myarrow] (metodos.north) -- ++(0,0) |- (clase.east);
  \end{tikzpicture}
  \caption{Métodos y atributos de la clase \texttt{LoadPattern}.}
  \label{fig:pyFEM-LoadPattern}
\end{figure}

Como mecánismo de optimización, se asigna una tupla con los elementos \verb|loads_at_joints| y \verb|distributed_loads| al atributo \verb|__slots__| de la clase para indicarle a Python que limite la cantidad de atributos que puede tener una instancia.\\

En el algoritmo \ref{alg:LoadPattern-init} se presenta el constructor de la clase \verb|LoadPattern|. El constructor de la clase no tiene argumentos de entrada. Sin embargo, asigna un diccionario vacío a los atributos \verb|loads_at_joints| y \verb|distributed_loads|.\\

\begin{lstlisting}[language=Python,caption=Constructor de la clase \texttt{LoadPattern}.,label=alg:LoadPattern-init, frame=single]
def __init__(self):
  """Instantiate a LoadPatter object"""
  self.loads_at_joints = {}
  self.distributed_loads = {}
\end{lstlisting}
\bigskip
A continuación se presentan los métodos de la clase \verb|LoadPattern|, con los cuales se puede, entre otras cosas, calcular el vector de fuerzas en los nodos de la estructura del caso de carga.\\

\subsubsection{add\_point\_load\_at\_joint()}

El método \verb|add_point_load_at_joint()| de la clase \verb|LoadPattern| permite agregar fuerzas en los nodos.\\

En el algoritmo \ref{alg:LoadPattern-add_point_load_at_joint} se presenta la implementación del método \verb|add_point_load_at_joint()|. Los argumentos de entrada opcionales \verb|*args| y \verb|**kwargs| son pasados al constructor de la clase \verb|PointLoad|, mientras que el argumento \verb|joint| es usado como llave para almacenar el objeto creado en el diccionario \verb|loads_at_joints|.\\

\begin{lstlisting}[language=Python,caption=Método \texttt{add\_point\_load\_at\_joint()} de la clase \texttt{LoadPattern}.,label=alg:LoadPattern-add_point_load_at_joint, frame=single]
def add_point_load_at_joint(self, joint, *args, **kwargs):
  """
  Add a point load at joint

  Parameters
  ----------
  joint : Joint
      Joint.
  """
  self.loads_at_joints[joint] = PointLoad(*args, **kwargs)
\end{lstlisting}

\subsubsection{add\_distributed\_load()}

El método \verb|add_distributed_load()| de la clase \verb|LoadPattern| permite agregar cargas distribuidas en los elementos aporticados.\\

En el algoritmo \ref{alg:LoadPattern-add_distributed_load} se presenta la implementación del método \verb|add_distributed_load()|. Los argumentos de entrada opcionales \verb|*args| y \verb|**kwargs| son pasados al constructor de la clase \verb|DistributedLoad|, mientras que el argumento \verb|frame| es usado como llave para almacenar el objeto creado en el diccionario \verb|distributed_loads|.\\

\begin{lstlisting}[language=Python,caption=Método \texttt{add\_distributed\_load()} de la clase \texttt{LoadPattern}.,label=alg:LoadPattern-add_distributed_load, frame=single]
def add_distributed_load(self, frame, *args, **kwargs):
  """
  Add a distributed load at frame
  Parameters
  ----------
  frame : Joint
      Frame
  """
  self.distributed_loads[frame] = DistributedLoad(*args, **kwargs)
\end{lstlisting}

\subsubsection{get\_number\_point\_loads\_at\_joints()}

El método \verb|get_number_point_loads_at_joints()| de la clase \verb|LoadPattern| calcula el número de nodos cargados.\\

En el algoritmo \ref{alg:LoadPattern-get_number_point_loads_at_joints} se presenta la implementación del método \verb|get_number_point_loads| \verb|_at_joints()|. El método calcula la cantidad de entradas que tiene el diccionario \verb|loads_at_| \verb|joints|.\\

\begin{lstlisting}[language=Python,caption=Método \texttt{get\_number\_point\_loads\_at\_joints()} de la clase \texttt{LoadPattern}.,label=alg:LoadPattern-get_number_point_loads_at_joints, frame=single]
def get_number_point_loads_at_joints(self):
  """Get number loads at joints"""
  return len(self.loads_at_joints)
\end{lstlisting}

\subsubsection{get\_number\_distributed\_loads()}

El método \verb|get_number_distributed_loads()| de la clase \verb|LoadPattern| calcula el número de elementos aporticados cargados.\\

En el algoritmo \ref{alg:LoadPattern-get_number_distributed_loads} se presenta la implementación del método \verb|get_number_distributed_| \verb|loads()|. El método calcula la cantidad de entradas que tiene el diccionario \verb|distributed_| \verb|loads|.\\

\begin{lstlisting}[language=Python,caption=Método \texttt{get\_number\_distributed\_loads()} de la clase \texttt{LoadPattern}.,label=alg:LoadPattern-get_number_distributed_loads, frame=single]
def get_number_distributed_loads(self):
  """Get number distributed loads"""
  return len(self.distributed_loads)
\end{lstlisting}

\subsubsection{get\_f()}

El método \verb|get_f()| de la clase \verb|LoadPattern| calcula el vector de fuerzas total en los nodos de la estructura del caso de carga, representado por objetos tipo \verb|LoadPattern|, con respecto al sistema de coordenadas global.\\

Según \cite{weaver1990matrixanalysis}, el vector de fuerzas equivalente en los nodos de la estructura $ A_E $ debido a las cargas en los elementos aporticados se calcula como

\begin{equation}
  \mathbf{A_E} = - \sum_{i=1}^m \mathbf{A_{MSi}}
\end{equation}

donde $ A_{MSi} $ es el vector de acciones fijas en los nodos del elemento aporticado $ i $ en el sistema de coordenadas global. Este vector de fuerzas equivalentes se suma con el vector de fuerzas aplicadas en los nodos de la estructura para formar el vector de fuerzas total.\\

En el algoritmo \ref{alg:LoadPattern-get_f} se presenta la implementación del método \verb|get_f()|. El método recibe los argumentos de entrada obligatorios \verb|flag_displacements| e \verb|indexes|. La variable \verb|flag_displacements| indica para cada grado de libertad si está o no activado, mientras que la variable \verb|indexes| relaciona los objetos tipo \verb|Joint| con sus respectivos grados de libertad. Según los grados de libertad activados se genera el vector de fuerzas total en los nodos de la estructura del caso de carga.\\

El vector de fuerzas aplicadas en los nodos de la estructura se ensambla, a partir de las fuerzas aplicadas en cada nodo de la estructura y sus respectivos grados de libertad, con la función \verb|coo_matrix()|.\\

Para esto, primero se calcula la cantidad de grados de libertad activados y nodos cargados, con la función \verb|count_nonzero| y el método \verb|get_number_point_loads_at_joints()| (véase el algoritimo \ref{alg:LoadPattern-get_number_point_loads_at_joints}), y se almacenan en las variables \verb|no| y \verb|n|, respectivamente.\\

Con estos valores se dimensionan los \emph{arrays} \verb|rows|, \verb|cols| y \verb|data|. Los \emph{arrays} \verb|rows| y \verb|data| se crean con valores arbitrarios, con la función \verb|np.zeros()|, para almacenar los grados de libertad y las fuerzas en los nodos respectivamente, mientras que el \emph{array} \verb|cols| se crea con ceros en todas sus entradas, con la función \verb|np.zeros()|, debido a que el vector de fuerzas en los nodos de la estructura es un vector columna.\\

Finalmente, se crea el vector de fuerzas del caso de carga en los nodos de la estructura, pasando a la función \verb|coo_matrix()| los \emph{arrays} \verb|rows|, \verb|cols| y \verb|data|, y se le resta el vector de fuerzas equivalentes del caso de carga en los nodos, calculada con el método \verb|get_f_fixed()|.\\

\begin{lstlisting}[language=Python,caption=Método \texttt{get\_f()} de la clase \texttt{LoadPattern}.,label=alg:LoadPattern-get_f, frame=single]
def get_f(self, flag_displacements, indexes):
  """
  Get the load vector

  Attributes
  ----------
  flag_displacements : array
      Flags active joint's displacements
  indexes : dict
      Key value pairs joints and indexes.
  """
  no = np.count_nonzero(flag_displacements)

  n = self.get_number_point_loads_at_joints()

  rows = np.empty(n * no, dtype=int)
  cols = np.zeros(n * no, dtype=int)
  data = np.empty(n * no)

  for i, (joint, point_load) in enumerate(self.loads_at_joints.items()):
      rows[i * no:(i + 1) * no] = indexes[joint]
      data[i * no:(i + 1) * no] = point_load.get_load(flag_displacements)

  return coo_matrix((data, (rows, cols)), (no * len(indexes), 1)) - self.get_f_fixed(flag_displacements, indexes)
\end{lstlisting}

\subsubsection{get\_f\_fixed()}

El método \verb|get_f_fixed()| de la clase \verb|LoadPattern| calcula el vector de fuerzas equivalente en los nodos de la estructura del caso de carga, representado por objetos tipo \verb|LoadPattern|, con respecto al sistema de coordenadas global.\\

En el algoritmo \ref{alg:LoadPattern-get_f_fixed} se presenta la implementación del método \verb|get_f_fixed()|. El método recibe los argumentos de entrada obligatorios \verb|flag_displacements| e \verb|indexes|. La variable \verb|flag_displacements| indica para cada grado de libertad si está o no activado, mientras que la variable \verb|indexes| relaciona los objetos tipo \verb|Joint| con sus respectivos grados de libertad. Según los grados de libertad activados se genera el vector de fuerzas equivalentes en los nodos de la estructura del caso de carga.\\

El vector de fuerzas equivalentes en los nodos de la estructura se ensambla, a partir de las cargas aplicadas en los elementos aporticados y sus respectivos grados de libertad, con la función \verb|coo_matrix()|.\\

Para esto, primero se calcula la cantidad de grados de libertad activados y elementos aporticados cargados, con la función \verb|count_nonzero| y el método \verb|get_number_distributed_loads()| (véase el algoritmo \ref{alg:LoadPattern-get_number_distributed_loads}), y se almacenan en las variables \verb|no| y \verb|n|, respectivamente.\\

Con estos valores se dimensionan los \emph{arrays} \verb|rows|, \verb|cols| y \verb|data|. Los \emph{arrays} \verb|rows| y \verb|data| se crean con valores arbitrarios, para almacenar los grados de libertad y las fuerzas en los nodos respectivamente, mientras que el \emph{array} \verb|cols| se crea con ceros entodas sus entradas, debido a que el vector de fuerzas equivalente en los nodos de la estructura es un vector columna.\\

\begin{lstlisting}[language=Python,caption=Método \texttt{get\_f\_fixed()} de la clase \texttt{LoadPattern}.,label=alg:LoadPattern-get_f_fixed, frame=single]
def get_f_fixed(self, flag_joint_displacements, indexes):
  """
  Get the f fixed.

  Attributes
  ----------
  flag_joint_displacements : array
      Flags active joint's displacements.
  indexes : dict
      Key value pairs joints and indexes.
  """
  no = np.count_nonzero(flag_joint_displacements)

  n = self.get_number_distributed_loads()

  rows = np.empty(2 * n * no, dtype=int)
  cols = np.zeros(2 * n * no, dtype=int)
  data = np.empty(2 * n * no)

  for i, (frame, distributed_load) in enumerate(self.distributed_loads.items()):
      joint_j = frame.joint_j
      joint_k = frame.joint_k

      rows[i * 2 * no:(i + 1) * 2 * no] = np.concatenate((indexes[joint_j], indexes[joint_k]))
      data[i * 2 * no:(i + 1) * 2 * no] = distributed_load.get_f_fixed(flag_joint_displacements, frame)

  return coo_matrix((data, (rows, cols)), (no * len(indexes), 1))
\end{lstlisting}

\subsection{PointLoad}

La clase \verb|PointLoad| representa las fuerzas aplicadas en los nodos de la estructura, al establecer el valor de las fuerzas en el sistema de coordenadas global.\\

En el algoritmo \ref{alg:PointLoad} se presenta la implementación de la clase \verb|PointLoad|. Como mecánismo de optimización, se asigna una tupla con los elementos \verb|'fx'|, \verb|'fy'|, \verb|'fz'|, \verb|'mx'|, \verb|'my'| y \verb|'mz'| al atributo \verb|__slots__| de la clase para indicarle a Python que limite la cantidad de atributos que puede tener una instancia.\\

El constructor de la clase recibe seis argumentos de entrada opcionales, para cada uno de los grados de libertad, los cuales tienen 0 como valor por defecto. El usuario debe indicar el valor de las fuerzas diferentes de cero.\\

Finalmente, la clase \verb|PointLoad| implementa el método \verb|get_load()| que genera un \emph{array} que indica, para cada grado de libertad activado, el valor de la fuerza.

\begin{lstlisting}[language=Python,caption=Clase \texttt{PointLoad} implementada en el archivo \texttt{primitives.py}.,label=alg:PointLoad, frame=single]
class PointLoad(AttrDisplay):
    """
    Point load

    Attributes
    ----------
    fx : float
        Force along 'x' axis.
    fy : float
        Force along 'y' axis.
    fz : float
        Force along 'z'axis.
    mx : float
        Force around 'x' axis.
    my : float
        Force around 'y' axis.
    mz : float
        Force around 'z' axis.

    Methods
    -------
     get_load(flag_joint_displacements)
        Get the load vector.
    """
    __slots__ = ('fx', 'fy', 'fz', 'mx', 'my', 'mz')

    def __init__(self, fx=0, fy=0, fz=0, mx=0, my=0, mz=0):
        """
        Instantiate a PointLoad object

        Parameters
        ----------
        fx : float
            Force along 'x' axis.
        fy : float
            Force along 'y' axis.
        fz : float
            Force along 'z' axis.
        mx : float
            Force around 'x' axis.
        my : float
            Force around 'y' axis.
        mz : float
            Force around 'z' axis.
        """
        self.fx = fx
        self.fy = fy
        self.fz = fz

        self.mx = mx
        self.my = my
        self.mz = mz

    def get_load(self, flag_joint_displacements):
        """
        Get load

        Parameters
        ----------
        flag_joint_displacements : array
            Flags active joint's displacements.
        """

        return np.array([getattr(self, name) for name in self.__slots__])[flag_joint_displacements]
\end{lstlisting}

\subsection{DistributedLoad}

La clase \verb|DistributedLoad| representa las cargas distribuidas aplicadas en los elementos aporticados de la estructura, al establecer el valor de las cargas en el sistema de coordenadas local.\\

En el algoritmo \ref{alg:DistributedLoad} se presenta la implementación de la clase \verb|DistributedLoad|. Como mecánismo de optimización, se asigna una tupla con los elementos \verb|'system'|, \verb|'fx'|, \verb|'fy'| y \verb|'fz'| al atributo \verb|__slots__| de la clase para indicarle a Python que limite la cantidad de atributos que puede tener una instacia.\\

El constructor de la clase recibe tres argumentos de entrada opcionales, para cada una de las cargas distribuidas a lo largo de los ejes principales del sistema de coordenadas local. El usuario debe indicar el valor de las cargas distribuidas diferentes de cero.\\

Finalmente, la clase \verb|DistributedLoad| implementa el método \verb|get_f_fixed()| que genera un \emph{array} que indica, para cada grado de libertad activado, las fuerzas equivalentes en los nodos de la estructura en el sistema de coordenadas global.\\

\begin{lstlisting}[language=Python,caption=Clase \texttt{DistributedLoad} implementada en el archivo \texttt{primitives.py}.,label=alg:DistributedLoad, frame=single]
    class DistributedLoad(AttrDisplay):
    """
    Distributed load

    Attributes
    ----------
    system: str
        Coordinate system ('local' by default).
    fx : float
        Distributed force along 'x' axis.
    fy : float
        Distributed force along 'y' axis.
    fz : float
        Distributed force along 'z' axis.

    Methods
    -------
    get_load()
        Get the load vector.
    """
    __slots__ = ('system', 'fx', 'fy', 'fz')

    def __init__(self, fx=0, fy=0, fz=0):
        """
        Instantiate a Distributed object

        Parameters
        ----------
        fx : float
            Distributed force along 'x' axis.
        fy : float
            Distributed force along 'y' axis.
        fz : float
            Distributed force along 'z' axis.
        """
        self.system = 'local'

        self.fx = fx
        self.fy = fy
        self.fz = fz

    def get_f_fixed(self, flag_joint_displacements, frame):
        """
        Get f fixed.

        Parameters
        ----------
        flag_joint_displacements : array
            Flags active joint's displacements.
        frame : Frame
            Frame.
        """
        length = frame.get_length()

        fx = self.fx
        fy = self.fy
        fz = self.fz

        f_local = [-fx * length / 2, -fy * length / 2, -fz * length / 2, 0, fz * length ** 2 / 12, -fy * length ** 2 / 12]
        f_local += [fx * length / 2, -fy * length / 2, -fz * length / 2, 0, -fz * length ** 2 / 12, fy * length ** 2 / 12]

        return np.dot(frame.get_rotation_matrix(flag_joint_displacements), f_local)
\end{lstlisting}

\subsection{Displacement}

La clase \verb|Displacement| representa los desplazamientos de los nodos de la estructura, al establecer el valor de las translaciones y rotaciones en el sistema de coordenadas global.\\

En el algoritmo \ref{alg:Displacement} se presenta la implementación de la clase \verb|Displacement|. Como mecánismo de optimización, se asigna una tupla con los elementos \verb|'ux'|, \verb|'uy'|, \verb|'uz'|, \verb|'rx'|, \verb|'ry'| y \verb|'rz'| al atributo \verb|__slots__| de la clase para indicarle a Python que limite la cantidad de atributos que puede tener una instancia.\\

El constructor de la clase recibe seis argumentos de entrada opcionales, para cada uno de los posibles desplazamientos de los nodos en el sistema de coordenadas global. El usuario debe indicar el valor de los desplazamientos diferentes de cero.\\

Finalmente, la clase \verb|Displacement| implementa el método \verb|get_displacements()| que genera un \emph{array} que indica, para cada grado de libertad activado, el valor del desplazamiento.\\

\begin{lstlisting}[language=Python,caption=Clase \texttt{Displacement} implementada en el archivo \texttt{primitives.py}.,label=alg:Displacement, frame=single]
class Displacement(AttrDisplay):
    """
    Displacement

    Attributes
    ----------
    ux : float
        Translation along 'x' axis.
    uy : float
        Translation along 'y' axis.
    uz : float
        Translation along 'z' axis.
    rx : float
        Rotation around 'x' axis.
    ry : float
        Rotation around 'y' axis.
    rz : float
        Rotation around 'z' axis.

    Methods
    -------
    get_displacements()
        Get the displacement vector.
    """
    __slots__ = ('ux', 'uy', 'uz', 'rx', 'ry', 'rz')

    def __init__(self, ux=0, uy=0, uz=0, rx=0, ry=0, rz=0):
        """
        Instantiate a Displacement

        Parameters
        ----------
        ux : float
            Translation along 'x' axis.
        uy : float
            Translation along 'y' axis.
        uz : float
            Translation along 'z' axis.
        rx : float
            Rotation around 'x' axis.
        ry : float
            Rotation around 'y' axis.
        rz : float
            Rotation around 'z' axis.
        """
        self.ux = ux
        self.uy = uy
        self.uz = uz

        self.rx = rx
        self.ry = ry
        self.rz = rz

    def get_displacement(self, flag_joint_displacements):
        """Get displacements"""
        return np.array([getattr(self, name) for name in self.__slots__])[flag_joint_displacements]
\end{lstlisting}

\subsection{Reaction}

La clase \verb|Reaction| representa las reacciones de los apoyos de la estructura, al establecer el valor de las reacciones en el sistema de coordenadas global.\\

En el algoritmo \ref{alg:Reaction} se presenta la implementación de la clase \verb|Reaction|. Como mecánismo de optimización, se asigna una tupla con los elementos \verb|'fx'|, \verb|'fy'|, \verb|'fz'|, \verb|'mx'|, \verb|'my'| y \verb|'mz'| al atributo \verb|__slots__| de la clase para indicarle a Python que limite la cantidad de atributos que puede tener una instancia.\\

El constructor de la clase recibe seis argumentos de entrada opcionales, para cada una de las posibles reacciones en el sistema de coordenadas global. El usuario debe indicar el valor de las reacciones diferentes de cero.\\

Finalmente, la clase \verb|Reaction| implementa el método \verb|get_reactions()| que genera un \emph{array} que indica, para cada grado de libertad activado, el valor de la reacción.\\

\begin{lstlisting}[language=Python,caption=Clase \texttt{Reaction} implementada en el archivo \texttt{primitives.py}.,label=alg:Reaction, frame=single]
class Reaction(AttrDisplay):
"""
Reaction

Attributes
----------
fx : float
    Force along 'x' axis.
fy : float
    Force along 'y' axis.
fz : float
    Force along 'z' axis.
mx : float
    Moment around 'x' axis.
my : float
    Moment around 'y' axis.
mz : float
    Moment around 'z' axis.

Methods
-------
get_reactions()
    Get the load vector.
"""
__slots__ = ('fx', 'fy', 'fz', 'mx', 'my', 'mz')

def __init__(self, fx=0, fy=0, fz=0, mx=0, my=0, mz=0):
    """
    Instantiate a Reaction

    Parameters
    ----------
    fx : float
        Force along 'x' axis.
    fy : float
        Force along 'y' axis.
    fz : float
        Force along 'z' axis.
    mx : float
        Moment around 'x' axis.
    my : float
        Moment around 'y' axis.
    mz : float
        Moment around 'z' axis.
    """
    self.fx = fx
    self.fy = fy
    self.fz = fz
    self.mx = mx
    self.my = my
    self.mz = mz

def get_reactions(self, flag_joint_displacements):
    """Get reactions"""
    return np.array([getattr(self, name) for name in self.__slots__])[flag_joint_displacements]
\end{lstlisting}

\section{Structure}

La clase \verb|Structure| representa el modelo de estructuras aporticadas tridimensionales sometidas a cargas estáticas, al agregar objetos que describen la geometría de la estructura, sus condiciones de apoyo y las solicitaciones externas. En la figura \ref{fig:pyFEM-Structure-repetida} se presentan los métodos y atributos de esta clase.\\

\begin{figure}[ht]
    \centering
    \begin{tikzpicture}
      \node (clase) [class] {\textbf{Structure}};
  
      % attrs
      \node (atributos) [left=of clase, xshift=-1cm]{};
  
      % grados de libertad
      \matrix [
      matrix of nodes,
      below=of atributos,
      nodes={
        anchor=center
      }] (gradosLibertad) {
        \node {ux,}; & \node {uy,}; & \node {uz,};\\
        \node {rx,}; & \node {ry,}; & \node {rz,};\\
      };
  
      % diccionarios
      \matrix [
      matrix of nodes,
      below=of gradosLibertad,
      yshift=1cm,
      nodes={
        anchor=center
      }] (diccionarios) {      
        \node {materials,};\\
        \node {sections,};\\
        \node {joints,};\\      
        \node {frames,};\\
        \node {supports,};\\      
        \node {load\_patterns,};\\
      };
  
      % resultados
      \matrix [
      matrix of nodes,
      below=of diccionarios,
      yshift=1cm,
      nodes={
        anchor=center
      }] (resultados) {      
        \node {displacements,};\\
        \node {reactions};\\
      };
      
      % methods
      \node (aux2) [right=of clase, xshift=1cm]{};
      \matrix [
      matrix of nodes,
      below=of aux2,
      nodes={
        anchor=center
      }] (metodos1) {
        \node {add\_material(),}; & \node {add\_section(),};\\
        \node {add\_rectangular\_section(),}; & \node {add\_joint(),};\\
        \node {add\_frame(),}; & \node {add\_support(),};\\
        \node {add\_load\_pattern(),}; & \node {add\_load\_at\_joint(),};\\
      };
  
      \matrix [
      matrix of nodes,
      below=of metodos1,
      yshift=1cm,
      nodes={
        anchor=center
      }] (metodos2) {
        \node {add\_distributed\_load(),};\\
        \node {get\_flag\_active\_joint\_displacements(),};\\
        \node {get\_number\_active\_joint\_displacements(),};\\
      };
      
      \matrix [
      matrix of nodes,
      below=of metodos2,
      yshift=1cm,
      nodes={
        anchor=center
      }] (metodos3) {
        \node {get\_number\_joints(),}; & \node {get\_number\_frames(),};\\
        \node {set\_indexes(),}; & \node {get\_stiffness\_matrix(),};\\
      };
  
      \matrix [
      matrix of nodes,
      below=of metodos3,
      yshift=1cm,
      nodes={
        anchor=center
      }] (metodos4) {
        \node {get\_stiffness\_matrix\_with\_support(),};\\
        \node {solve\_load\_pattern(),};\\
        \node {set\_load\_pattern\_displacements(),};\\
      };
  
      \matrix[
      matrix of nodes,
      below=of metodos4,
      yshift=1cm,
      nodes={
        anchor=center
      }] (metodos5) {
        \node {set\_load\_pattern\_reactions(),}; & \node {solve(),};\\
      };
      
      \node [below=of metodos5, yshift=1cm] (metodos6) {export()};
      
      \node [atributos, fit=(gradosLibertad) (diccionarios) (resultados)] {};
      \node [metodos, fit=(metodos1) (metodos2) (metodos3) (metodos4) (metodos5) (metodos6)] {};
  
      \draw[myarrow] (gradosLibertad.north) -- ++(0,0) |- ( clase.west);
      \draw[myarrow] (metodos1.north) -- ++(0,0) |- (clase.east);
    \end{tikzpicture}
    \caption{Métodos y atributos de la clase \texttt{Structure} (repetida).}
    \label{fig:pyFEM-Structure-repetida}
\end{figure}
  
Como se mencionó anteriormente, el constructor de la clase recibe seis argumentos de entrada opcionales, uno para cada grado de libertad, los cuales tienen \verb|False| como valor por defecto. Cuando el usuario crea un objeto de esta clase debe indicar qué grados de libertad quiere tener en cuenta para analizar el modelo (véase el algoritmo \ref{alg:Structure-init}).\\

Con los métodos \verb|add_material()|, \verb|add_section()|, \verb|add_rectangular_section()|, \verb|add_| \verb|joint()|, \verb|add_frame()| y \verb|add_support()| se pueden agregar objetos tipo \verb|Material|, \verb|Section|, \verb|RectangularSection|, \verb|Joint|, \verb|Frame| y \verb|Support|, respectivamente, para describir la geometría y condiciones de apoyo de la estructura.\\

Con los métodos \verb|add_load_pattern()|, \verb|add_load_at_joint()| y \verb|add_distributed_load()| de pueden agregar objetos tipo \verb|LoadPattern|, \verb|PointLoad| y \verb|DistributedLoad|, respectivamente, para describir las cargas de los patrones de carga a los que se encuentra sometida la estructura.\\

A continuación se presentan los demás métodos de la clase \verb|Structure|, con los cuales se puede, entre otras cosas, analizar linealmente el modelo para encontrar los desplazamientos y reacciones de la estructura sometida a los diferentes patrones de carga.\\

\subsection{get\_flag\_active\_joint\_displacements()}

El método \verb|get_flag_active_joint_displacements()| de la clase \verb|Structure| genera un \emph{array} que indica para cada grado de libertad si está o no activado.\\

En el algoritmo \ref{alg:Structure-get_flag_active_joint_displacements} se presenta la implementación del método \verb|get_flag_active_joint_| \verb|displacements()|. El método genera un \emph{array} con los valores de los atributos \verb|ux|, \verb|uy|, \verb|uz|, \verb|rx|, \verb|ry| y \verb|rz| como entradas.

\begin{lstlisting}[language=Python,caption=Método \texttt{get\_flag\_active\_joint\_displacements()} de la clase \texttt{Structure}.,label=alg:Structure-get_flag_active_joint_displacements, frame=single]
def get_flag_active_joint_displacements(self):
    """
    Get active joint displacements
    
    Returns
    -------
    indexes: array
        Flag active joint displacements.
    """
    return np.array([self.ux, self.uy, self.uz, self.rx, self.ry, self.rz])
\end{lstlisting}

\subsection{get\_number\_active\_joint\_displacements()}

El método \verb|get_number_active_joint_displacements()| de la clase \verb|Structure| calcula el número de grados de libertad activados.\\

En el algoritmo \ref{alg:Structure-get_number_active_joint_displacements} se presenta la implementación del método \verb|get_number_active_joint_| \verb|displacements()|. El método calcula la cantidad de entradas iguales a \verb|True| del \emph{array} generado por el método \verb|get_flag_active_joint_displacements()| (véase el algoritmo \ref{alg:Structure-get_flag_active_joint_displacements}).\\

\begin{lstlisting}[language=Python,caption=Método \texttt{get\_number\_active\_joint\_displacements()} de la clase \texttt{Struc} \texttt{ture}.,label=alg:Structure-get_number_active_joint_displacements, frame=single]
def get_flag_active_joint_displacements(self):
    """
    Get active joint displacements
    
    Returns
    -------
    array
        Flags active joint displacements.
    """
    return np.array([self.ux, self.uy, self.uz, self.rx, self.ry, self.rz])
\end{lstlisting}

\subsection{get\_number\_joints()}
El método \verb|get_number_joints()| de la clase \verb|Structure| calcula cantidad de nodos de la estructura.\\

En el algoritmo \ref{alg:Structure-get_number_joints} se presenta la implementación del método \verb|get_number_joints()|. El método calcula la cantidad de entradas que tiene el diccionario \verb|joints|.\\

\begin{lstlisting}[language=Python,caption=Método \texttt{get\_number\_joints()} de la clase \texttt{Structure}.,label=alg:Structure-get_number_joints, frame=single]
def get_number_joints(self):
    """Get number of joints
    
    Returns
    -------
    int
        Number of joints.
    """
    return len(self.joints)
\end{lstlisting}

\subsection{get\_number\_frames()}

El método \verb|get_number_frames()| de la clase \verb|Structure| calcula la cantidad de elementos aporticados de la estructura.\\

En el algoritmo \ref{alg:Structure-get_number_frames} se presenta la implementación del métdo \verb|get_number_frames()|. El método calcula la cantidad de entradas que tiene el diccionario \verb|frames|.\\

\begin{lstlisting}[language=Python,caption=Método \texttt{get\_number\_frames()} de la clase \texttt{Structure}.,label=alg:Structure-get_number_frames, frame=single]
def get_number_frames(self):
    """Get number of frames
    
    Returns
    -------
    int
        Number of frames
    """
    return len(self.frames)
\end{lstlisting}

\subsection{set\_indexes()}

El método \verb|set_indexes()| de la clase \verb|Structure| genera un diccionario donde las llaves son los nodos de la estructura y los valores los respectivos grados de libertad.\\

En el algoritmo \ref{alg:Structure-set_indexes} se presenta la implementación del método \verb|set_indexes()|. El método crea un diccionario donde las llaves son los objetos tipo \verb|Joint| del diccionario \verb|joints| y los valores \emph{arrays} con los respectivos grados de libertad.\\

Los grados de libertad de cada nodo de la estructura se asignan de manera secuencial en función de la cantidad de grados de libertad activados, calculada con el método \verb|get_number_| \verb|active_joint_displacements()| (véase el algoritmo \ref{alg:Structure-get_number_active_joint_displacements}). Al primer nodo se le asignan los primeros \verb|n| indices, comenzando desde cero, al segundo nodo los siguientes \verb|n| indices y así sucesivamente para cada uno de los demás nodos.\\

\begin{lstlisting}[language=Python,caption=Método \texttt{set\_indexes()} de la clase \texttt{Structure}.,label=alg:Structure-set_indexes, frame=single]
def set_indexes(self):
    """Set the indexes"""
    n = self.get_number_active_joint_displacements()

    return {joint: np.arange(n * i, n * (i + 1)) for i, joint in enumerate(self.joints.values())}
\end{lstlisting}

\subsection{get\_stiffness\_matrix()}

El método \verb|get_stiffness_matrix()| de la clase \verb|Structure| permite calcular la matriz de rigidez de la estructura.\\

En el algoritmo \ref{alg:Structure-get_stiffness_matrix} se presenta la implementación del método \verb|get_stiffness_matrix()|. El método recibe como argumentos de entrada el diccionario que relaciona los nodos de la estructura con sus respectivos grados de libertad, calculado con el método \verb|set_indexes()| (véase el algoritmo \ref{alg:Structure-set_indexes}). Según los grados de libertad de los nodos de la estructura se ensamblan las matrices de rigidez de los elementos aporticados con la función \verb|coo_matrix()|.\\

Para cada objeto tipo \verb|Frame| del diccionario \verb|frames| se calcula su matriz de rigidez en el sistema de coordenadas global, mediante el método \verb|get_global_stiffness_matrix()| (véase el algoritmo \ref{alg:Frame-get_global_stiffness_matrix}), y se extraen los grados de libertad de los nodos del elemento aporticado del diccionario \verb|indexes|.\\

Con estas variables se describe la matriz de rigidez en coordenadas locales como una matriz dispersa en el formato \emph{ijv}. Los indices de las filas y las columnas se almacenan en los \emph{arrays} \verb|rows| y \verb|cols|, respectivamente, mientras que los valores de la matriz se almacenan en el \emph{array} \verb|data|.\\

Finalmente, se genera la matriz de rigidez de la estructura indicando que se trata de una matriz cuadrada de tamaño del número de grados de libertad activados por la cantidad de nodos de la estructura.\\

\begin{lstlisting}[language=Python,caption=Método \texttt{get\_stiffness\_matrix()} de la clase \texttt{Structure}.,label=alg:Structure-get_stiffness_matrix, frame=single]
def get_stiffness_matrix(self, indexes):
    """
    Get the stiffness matrix of the structure

    Parameters
    ----------
    indexes : dict
        Key value pairs joints and indexes.
    
    Returns
    -------
    k : coo_matrix
        Stiffness matrix of the structure.
    """
    flag_joint_displacements = self.get_flag_active_joint_displacements()
    number_active_joint_displacements = np.count_nonzero(flag_joint_displacements)

    number_joints = self.get_number_joints()
    number_frames = self.get_number_frames()

    # just for elements with two joints
    n = 2 * number_active_joint_displacements  # change function element type
    n_2 = n ** 2

    rows = np.empty(number_frames * n_2, dtype=int)
    cols = np.empty(number_frames * n_2, dtype=int)
    data = np.empty(number_frames * n_2)

    for i, frame in enumerate(self.frames.values()):
        k_element = frame.get_global_stiffness_matrix(flag_joint_displacements)
        indexes_element = np.concatenate((indexes[frame.joint_j], indexes[frame.joint_k]))
        indexes_element = np.broadcast_to(indexes_element, (n, n))

        rows[i * n_2:(i + 1) * n_2] = indexes_element.flatten('F')
        cols[i * n_2:(i + 1) * n_2] = indexes_element.flatten()
        data[i * n_2:(i + 1) * n_2] = k_element.flatten()

    return coo_matrix((data, (rows, cols)), 2 * (number_active_joint_displacements * number_joints,))
\end{lstlisting}

\subsection{get\_stiffness\_matrix\_with\_support()}

El método \verb|get_stiffness_matrix_with_support()| de la clase \verb|Structure| modifica la matriz de rigidez de la estructura, calculada con el método \verb|get_stiffness_matrix()| (véase el algoritmo \ref{alg:Structure-get_stiffness_matrix}), para tener en cuenta las condiciones de apoyo.\\

Según \cite{reddy1993an}, para tener en cuenta las condiciones de apoyo de la estructura en la matriz de rigidez, se deben reemplazar los valores de las filas y las columnas asociadas a los grados de libertad restringidos por ceros, a excepción de los valores en la diagonal principal, los cuales deben ser reemplazados por 1.\\

En el algoritmo \ref{alg:Structure-get_stiffness_matrix_with_support} se presenta la implementación del método \verb|get_stiffness_matrix_| \verb|with_support()|. El método recibe como argumentos de entrada la matriz de rigidez de la estructura, calculada con el método \verb|get_stiffness_matrix()| (véase el algoritmo \ref{alg:Structure-get_stiffness_matrix}), y el diccionario que relaciona los nodos de la estructura con sus respectivos grados de libertad, calculado con el método \verb|set_indexes()| (véase el algoritmo \ref{alg:Structure-set_indexes}).\\

Para cada objeto tipo \verb|Support| del diccionario \verb|supports| se extraen los grados de libertad del diccionario \verb|indexes| y se calculan las restricciones del apoyo con el método \verb|get_restrains()| (véase el algoritmo \ref{alg:Support}). Estos valores se almacenan en las variables \verb|joint_indexes| y \verb|restrains| respectivamente.\\

Finalmente, para cada grado de libertad restringido se reemplazan los valores asociados de la fila y la columna de la matriz de rigidez de la estructura por ceros y el valor en la diagonal principal por 1.\\
\pagebreak

\begin{lstlisting}[language=Python,caption=Método \texttt{get\_stiffness\_matrix\_with\_support()} de la clase \texttt{Structure}.,label=alg:Structure-get_stiffness_matrix_with_support, frame=single]
def get_stiffness_matrix_with_support(self, stiffness_matrix, indexes):
    """
    Get the stiffness matrix of the structure with supports

    Parameters
    ----------
    stiffness_matrix : ndarray
        Stiffness matrix of the structure.
    indexes : dict
        Key value pairs joints and indexes.

    Returns
    -------
    stiffness_matrix_with_supports : ndarray
        Stiffness matrix of the structure modified by supports.
    """
    flag_joint_displacements = self.get_flag_active_joint_displacements()
    n = np.shape(stiffness_matrix)[0]
    
    for joint, support in self.supports.items():
        joint_indexes = indexes[joint]
        restrains = support.get_restrains(flag_joint_displacements)

        for index in joint_indexes[restrains]:
            stiffness_matrix[index] = stiffness_matrix[:, index] = np.zeros(n)
            stiffness_matrix[index, index] = 1
    
    return stiffness_matrix
\end{lstlisting}

\subsection{solve\_load\_pattern()}

El método \verb|solve_load_pattern()| de la clase \verb|Structure| calcula los vectores de desplazamientos y fuerzas en los nodos de la estructura debidos a las cargas definidas en los patrones de carga.\\

En el algoritmo \ref{alg:Structure-solve_load_pattern} se presenta la implementación del método \verb|solve_load_pattern()|. El método recibe como argumentos de entrada el patrón de carga, representado por objetos tipo \verb|LoadPattern| (véase el algoritmo \ref{alg:LoadPattern-init}), el diccionario que relaciona los nodos de la estructura con sus respectivos grados de libertad, calculado con el método \verb|set_indexes()| (véase el algoritmo \ref{alg:Structure-set_indexes}), la matriz de rigidez de la estructura, calculada con el método \verb|get_stiffness_matrix()| (véase el algoritmo \ref{alg:Structure-get_stiffness_matrix}), y la matriz de rigidez modificada para tener en cuenta las condiciones de apoyo, calculada con el método \verb|get_stiffness_matrix_| \verb|with_support()| (véase el algoritmo \ref{alg:Structure-get_stiffness_matrix_with_support}).\\

Según \cite{reddy1993an}, para tener en cuenta las condiciones de apoyo de la estructura en el vector de fuerzas en los nodos, se deben reemplazar los valores asociado a los grados de libertad restringidos por cero.\\

El vector de fuerzas en los nodos de la estructura del caso de carga se calcula con el método \verb|get_f()| (véase el algoritmo \ref{alg:LoadPattern-get_f}). Para cada objeto tipo \verb|Support| del diccionario \verb|supports| se extraen los respectivos grados de libertad del diccionario \verb|indexes| y se calculan las restricciones del apoyo con el método \verb|get_restrains()| (véase el algoritmo \ref{alg:Support}), para reemplazar los valores asociados a los grados de libertad restringidos del vector de fuerzas en los nodos de la estructura por cero.\\

Finalmente, se calculan los vectores de desplazamientos y fuerzas en los nodos de la estructura, y se almacena los resultados en las variables \verb|u| y \verb|f|, respectivamente.\\

\begin{lstlisting}[language=Python,caption=Método \texttt{solve\_load\_pattern()} de la clase \texttt{Structure}.,label=alg:Structure-solve_load_pattern, frame=single]
def solve_load_pattern(self, load_pattern, indexes, k, k_support):
    """
    Solve load pattern

    Parameters
    ----------
    load_pattern : LoadPattern
        Load pattern object.
    indexes : dict
        Key value pairs joints and indexes.
    k : ndarray
        Stiffness matrix of the structure.
    k_support : ndarray
        Stiffness matrix of the structure modified by supports.

    Returns
    -------
    u : ndarray
        Displacements vector.
    f : ndarray
        Forces vector.
    """
    flag_joint_displacements = self.get_flag_active_joint_displacements()

    f = load_pattern.get_f(flag_joint_displacements, indexes).toarray()

    for joint, support in self.supports.items():
        joint_indexes = indexes[joint]
        restrains = support.get_restrains(flag_joint_displacements)
        for index in joint_indexes[restrains]:
            f[index, 0] = 0

    u = np.linalg.solve(k_support, f)
    f = np.dot(k, u) + load_pattern.get_f_fixed(flag_joint_displacements, indexes).toarray()

    return u, f
\end{lstlisting}

\subsection{set\_load\_pattern\_displacements()}

El método \verb|set_load_pattern_displacements()| de la clase \verb|Structure| almacena los desplazamientos de los nodos de la estructura, debidos a las cargas definidas en los patrones de carga, en el diccionario \verb|displacements|.\\

En el algoritmo \ref{alg:Structure-set_load_pattern_displacements} se presenta la implementación del método \verb|set_load_pattern_displa| \verb|cements()|. El método recibe como argumentos de entrada el patrón de carga, representado por objetos tipo \verb|LoadPattern| (véase el algoritmo \ref{alg:LoadPattern-init}), el diccionario que relaciona los nodos de la estructura con sus respectivos grados de libertad, calculado con el método \verb|set_indexes()| (véase el algoritmo \ref{alg:Structure-set_indexes}), y el vector de desplazamientos de los nodos de la estructura, calculado con el método \verb|solve_load_pattern()| (véase el algoritmo \ref{alg:Structure-solve_load_pattern}).\\

Para cada objeto tipo \verb|Joint| del diccionario \verb|joints| se crea una entrada en el diccionario \verb|load_pattern_displacements|, donde las llaves son los objeto tipo \verb|Joint| y los valores objetos tipo \verb|Displacements|, creados con los respectivos valores del vector de desplazamientos de los nodos de la estructura (véase el algoritmo \ref{alg:Displacement}).\\

Finalmente, el diccionario \verb|load_pattern_displacements| se almacena en el diccionario \verb|displacements| usando el objeto tipo \verb|LoadPattern| como llave.\\

\begin{lstlisting}[language=Python,caption=Método \texttt{set\_load\_pattern\_displacements()} de la clase \texttt{Structure}.,label=alg:Structure-set_load_pattern_displacements, frame=single]
def set_load_pattern_displacements(self, load_pattern, indexes, u):
    """
    Set load pattern displacement

    Parameters
    ----------
    load_pattern : LoadPattern
        Load pattern.
    indexes : dict
        Key value pairs joints and indexes.
    u : ndarray
        Displacements.
    """
    flag_joint_displacements = self.get_flag_active_joint_displacements()
    load_pattern_displacements = {}

    for joint in self.joints.values():
        joint_indexes = indexes[joint]
        displacements = flag_joint_displacements.astype(float)
        displacements[flag_joint_displacements] = u[joint_indexes, 0]
        load_pattern_displacements[joint] = Displacement(*displacements)

    self.displacements[load_pattern] = load_pattern_displacements
\end{lstlisting}

\subsection{set\_load\_pattern\_reactions()}

El método \verb|set_load_pattern_reactions()| de la clase \verb|Structure| almacena las reacciones de los apoyos de la estructura, debidos a las cargas definidas en los patrones de carga, en el diccionario \verb|reactions|.\\

En el algoritmo \ref{alg:Structure-set_load_pattern_reactions} se presenta la implementación del método \verb|set_load_pattern_reac| \verb|tions()|. El método recibe como argumentos de entrada el patrón de carga, representado por objetos tipo \verb|LoadPattern| (véase el algoritmo \ref{alg:LoadPattern-init}), el diccionario que relaciona los nodos de la estructura con sus respectivos grados de libertad, calculado con el método \verb|set_indexes()| (véase el algoritmo \ref{alg:Structure-set_indexes}), y el vector de fuerzas en los nodos de la estructura, calculado con el método \verb|solve_load_pattern()| (véase el algoritmo \ref{alg:Structure-solve_load_pattern}).\\

Para cada objeto tipo \verb|Support| del diccionario \verb|supports| se crea una entrada en el diccionario \verb|load_pattern_reactions|, donde las llaves son los objetos tipo \verb|Joint| y los valores objetos tipo \verb|Reactions|, creados con los respectivos valores del vector de fuerzas en los nodos de la estructura (véase el algoritmo \ref{alg:Reaction}).\\

Finalmente, el diccionario \verb|load_pattern_reactions| se almacena en el dicionario \verb|reactions| usando el objeto tipo \verb|LoadPattern| como llave.\\

\begin{lstlisting}[language=Python,caption=Método \texttt{set\_load\_pattern\_reactions()} de la clase \texttt{Structure}.,label=alg:Structure-set_load_pattern_reactions, frame=single]
def set_load_pattern_reactions(self, load_pattern, indexes, f):
    """
    Set load pattern reactions

    Parameters
    ----------
    load_pattern : LoadPattern
        Load pattern.
    indexes : dict
        Key value pairs joints and indexes.
    f : ndarray
        Forces.
    """
    flag_joint_displacements = self.get_flag_active_joint_displacements()
    load_pattern_reactions = {}

    for joint in self.supports.keys():
        joint_indexes = indexes[joint]
        reactions = flag_joint_displacements.astype(float)
        reactions[flag_joint_displacements] = f[joint_indexes, 0]
        load_pattern_reactions[joint] = Reaction(*reactions)

    self.reactions[load_pattern] = load_pattern_reactions
\end{lstlisting}

\subsection{solve()}

El método \verb|solve()| de la clase \verb|Structure| analiza el modelo de la estructura sometida a los diferentes patrones de carga y almacena los resultados en los diccionarios \verb|displacements| y \verb|reactions|.\\ 

En el algoritmo \ref{alg:Structure-solve} se presenta la implementación del método \verb|solve()|. El método calcula el diccionario que relaciona los nodos de la estructura con sus respectivos grados de libertad, con el método \verb|set_indexes()| (véase el algoritmo \ref{alg:Structure-set_indexes}), la matriz de rigidez de la estructura, con el método \verb|get_stiffness_matrix()| (véase el algoritmo \ref{alg:Structure-get_stiffness_matrix}), y la matriz de rigidez modificada por las condiciones de apoyo, con el método \verb|get_stiffness_matrix_with_support()| (véase el algoritmo \ref{alg:Structure-get_stiffness_matrix_with_support}).\\

Para cada patrón de carga se calculan los vectores de desplazamientos y fuerzas en los nodos de la estructura y los resultados se almacenan en los diccionarios \verb|displacements| y \verb|reactions| respectivamente.\\

\begin{lstlisting}[language=Python,caption=Método \texttt{solve()} de la clase \texttt{Structure}.,label=alg:Structure-solve, frame=single]
def solve(self):
    """Solve the structure"""
    indexes = self.set_indexes()

    k = self.get_stiffness_matrix(indexes).toarray()
    k_support = self.get_stiffness_matrix_with_support(k, indexes)

    for load_pattern in self.load_patterns.values():
        u, f = self.solve_load_pattern(load_pattern, indexes, k, k_support)
        self.set_load_pattern_displacements(load_pattern, indexes, u)
        self.set_load_pattern_reactions(load_pattern, indexes, f)
\end{lstlisting}

\subsection{export()}

El método \verb|export()| de la clase \verb|Structure| genera un archivo de texto en formato JSON con la descripción del modelo para ser interpretado por el programa de computador \emph{FEM.js}.\\

El método almacena los objetos que representan los materiales, las secciones transversales, los nodos, los elementos aporticados, las condiciones de apoyo y los patrones de carga, con sus respectivas cargas, en las entradas \verb|materials|, \verb|sections|, \verb|joints|, \verb|frames|, \verb|supports| y \verb|load_patterns|, respectivamente, usando las mismas llaves con las que fueron agregados al modelo.\\

En el caso donde se usan dichos objetos como llaves para almacenar otros objetos, como es el caso de los objetos tipo \verb|Support| (véase el algoritmo \ref{alg:Structure-add_support}), o como atributos para crear otros, como es el caso de los objetos tipo \verb|Frame| (véase el algoritmo \ref{alg:Structure-add_frame}), se almacenan las llaves con las que fueron agregados al modelo.\\

A continuación se presenta la estructura general que tiene un archivo generado por este método.\\

\begin{lstlisting}[language={}, frame=single]
{
    "materials": {
        "key_material": {
            "E": 0.0,
            "G": 0.0
        },
        ...
    },
    "sections": {
        "key_section": {
            "area": 0.0,
            "Ix": 0.0,
            "Iy": 0.0,
            "Iz": 0.0,
            "type": "Section"
        },
        "another_key": {
            "area": 0.0,
            "Ix": 0.0,
            "Iy": 0.0,
            "Iz": 0.0,
            "type": "RectangularSection",
            width: 0.0,
            height: 0.0
        },
        ...
    },
    "joints": {
        "key": {
            "x": 0.0,
            "y": 0.0,
            "z": 0.0
        },
        ...
    },
    "frames": {
        "key": {
            "j": "key_joint",
            "k": "another_key_joint",
            "material": "key_material",
            "section": "key_section"
        },
        ...
    },
    "supports": {
        "key_joint": {
            "ux": bool,
            "uy": bool,
            "uz": bool,
            "rx": bool,
            "ry": bool,
            "rz": bool
        },
        ...
    },
    "load_patterns": {
        "key_load_pattern": {
            "joints": {
                "key_joint": [
                    {
                        "fx": 0.0,
                        "fy": 0.0,
                        "fz": 0.0,
                        "mx": 0.0,
                        "my": 0.0,
                        "mz": 0.0
                    },
                    ...
                ],
                ...
            },
            "frames": {
                "distributed": {
                    "local": {
                        "key_frame": {
                            "fx": 0.0,
                            "fy": 0.0,
                            "fz": 0.0
                        },
                        ...
                    }
                }
            }
        }
    }
}
\end{lstlisting}

\section{Otras clases}

Las clases presentadas hasta aquí permiten analizar linealmente estructuras aporticadas tridimensionales sometidas a cargas estáticas. Adicional a estas clases, en el archivo \verb|classtools.py| se desarolló la clase \verb|AttrDisplay| y la metaclase \verb|UniqueInstances|, las cuales son heredadas por las demás clases.\\

\subsection{AttrDisplay}

La clase \verb|AttrDisplay| implementa una representación más cómoda de los objetos al redefinir el método \verb|__repr__()|.\\

En el algoritmo \ref{alg:AttrDisplay} se presenta la implementación del método \verb|AttrDisplay|. El método genera una cadena de texto donde aparece el tipo del objeto y entre paréntesis los valores de sus atributos.\\

\begin{lstlisting}[language=Python,caption=Clase \texttt{AttrDisplay} implementada en el archivo \texttt{classtools.py}.,label=alg:AttrDisplay, frame=single]
class AttrDisplay:
    __slots__ = ()
    
    def __repr__(self):
        """
        Get representation object
        
        Returns
        -------
        str
        Object representation.
        """
        return "{}({})".format(self.__class__.__name__,', '.join([repr(getattr(self, name)) for name in self.__slots__]))
\end{lstlisting}
\bigskip

\subsection{UniqueInstances}

La metaclase \verb|UniqueInstances| implementa un mecánismo para evitar crear objetos con los mismos atributos de otros objetos de la misma clase, redefiniendo los métodos \verb|__new__()| y \verb|__call__()|.\\

En el algoritmo \ref{alg:UniqueInstances-new} se presenta la implementación del método \verb|__new__()|. La metaclase redefine la creación de las clases que la implementan, asignándoles un \emph{set}, inicialmente vacío, y \emph{sobrecargando} sus métodos \verb|__setattr__()| y \verb|__del__()|.\\

En el \emph{set} \verb|instances_attrs| se lleva el registro de los atributos de los objetos existentes de la misma clase, mientras que los métodos \verb|__setattr__()| y \verb|__del__()| actualizan el \emph{set} cuando un atributo de cualquier objeto cambia o cuando el objeto es eliminado, respectivamente.\\

\begin{lstlisting}[language=Python,caption=Método \texttt{\_\_new\_\_()} de la metaclase \texttt{UniqueInstances}.,label=alg:UniqueInstances-new, frame=single]
def __new__(mcs, name, bases, dct):
    """
    Create a class

    Parameters
    ----------
    name : str
        Class name.
    bases : tuple
        Parent classes.
    dct : dict
        Namespace's class.
    """
    if '__slots__' in dct:
        dct['instances_attrs'] = set()
        dct['__setattr__'] = UniqueInstances.setattr
        dct['__del__'] = UniqueInstances.delete

        return type.__new__(mcs, name, bases, dct)
    else:
        print("Warning: " +
            "Classes created with the UniqueInstances metaclass must implement the " +
            "'__slots__ ' variable. The class was not created.")
\end{lstlisting}
\bigskip

En el algoritmo \ref{alg:UniqueInstances-setattr} se presenta la implementación de la función \verb|setattr|, la cual redefine el método \verb|__setattr__()| de las clases que implementan la metaclase \verb|UniqueInstances|.\\

Antes que cambie el valor de un atributo de un objeto, este método verifica que los nuevos valores de sus atributos no los tenga ya otro objeto de la misma clase, revisando los elementos almacenados en el \emph{set} \verb|instances_attrs| de la clase.\\

En caso que no existan objetos con los mismos atributos, se cambia el atributo del objeto y se actualiza el \emph{set} \verb|instances_attrs|. En caso contrario, no se modifica el objeto.\\

\begin{lstlisting}[language=Python,caption=Función \texttt{setattr} implementada en la clase \texttt{UniqueInstances}.,label=alg:UniqueInstances-setattr, frame=single]
def setattr(self, key, value):
    """
    Set attribute object if doesn't collide with attributes another object

    Parameters
    ----------
    key : string
        Key's attribute to modified.
    value : value
        Value to assign.
    """
    if hasattr(self, key):
        # get instances attrs and instance attrs
        instances_attrs = getattr(self.__class__, 'instances_attrs')
        instance_attrs = tuple(getattr(self, name) for name in self.__slots__)

        # get possible new instance attrs
        _instance_attrs = tuple((getattr(self, _key) if _key != key
                                    else value for _key in self.__slots__))

        # add new instance attrs if not in instances attrs
        if _instance_attrs in instances_attrs:
            print("Warning: " +
                    "There is another instance of the class " +
                    "'{}'".format(self.__class__.__name__) +
                    " with the same attributes. The object was not changed.")

            return None
        else:
            instances_attrs.remove(instance_attrs)
            instances_attrs.add(_instance_attrs)

    self.__class__.__dict__[key].__set__(self, value)
\end{lstlisting}
\bigskip

En el algoritmo \ref{alg:UniqueInstances-delete} se presenta la implementación de la función \verb|delete()|, la cual redefine el método \verb|__del__()| de las clases que implementan la metaclase \verb|UniqueInstances|.\\

Antes de eliminar todas las referencias a un objeto, este método elimina la entrada asociada del \emph{set} \verb|instances_attrs| de la clase.\\

\begin{lstlisting}[language=Python,caption=Function \texttt{delete} implementada en la clase \texttt{UniqueInstances}.,label=alg:UniqueInstances-delete, frame=single]
def delete(self):
    getattr(self.__class__, 'instances_attrs').remove(tuple(getattr(self, name) for name in self.__slots__))
\end{lstlisting}
\bigskip

Finalmente, en el algoritmo \ref{alg:UniqueInstances-call} se presenta la implementación del método \verb|__call__()|. La metaclase evita que se creen objetos con los mismos atributos de otros objetos de la misma clase ya creados, revisando que los atributos del objeto a crear no se encuentren en el \emph{set} \verb|instances_attrs| de la clase.\\

\begin{lstlisting}[language=Python,caption=Método \texttt{\_\_call\_\_} de la metaclase \texttt{UniqueInstances}.,label=alg:UniqueInstances-call, frame=single]
    def setattr(self, key, value):
def __call__(cls, *args, **kwargs):
    """
    Return an instances if it does not already exist otherwise return None

    """
    # get __init__ class
    init = cls.__init__

    # get init's arguments and default values
    varnames = getattr(getattr(init, '__code__'), 'co_varnames')[len(args) + 1:]
    default = getattr(init, '__defaults__')

    # create list with args
    instance_attrs = list(args)

    # fill instance_attrs with kwargs or init's default values
    for i, key in enumerate(varnames):
        instance_attrs.append(kwargs.get(key, default[i]))

    # from list to tuple
    instance_attrs = tuple(instance_attrs)  # FIXME: i don't need necessary check all params

    # get obj's attrs and instances attrs class
    instances_attrs = getattr(cls, 'instances_attrs')

    # check obj's attrs don't be in instances attrs class
    if instance_attrs in instances_attrs:
        print("Warning: " +
                "There is another instance of the class " +
                "'{}' ".format(cls.__name__) +
                "with the same attributes. The object was not created.")
    else:
        # add obj's attrs to instances attrs
        instances_attrs.add(instance_attrs)

        # create and instantiate the object
        obj = cls.__new__(cls, *args, **kwargs)
        obj.__init__(*args, **kwargs)

        return obj
\end{lstlisting}
% \chapter{Metodología}
\label{chap:metodologia}
En el capítulo anterior se estudiaron las características de diversos programas de computador de análisis matricial, tanto académicos como comerciales, con el fin de
\begin{inparaenum}[$ (a) $]
    \item identificar los elementos que tienen en común y
    \item las ventajas y las desventajas que hay entre estos. \\
\end{inparaenum}

Con base en dicho estudio, el problema que debía solucionar el programa de computador a desarrollar, el cual consiste en resolver modelos de estructuras reticulares tridimensionales sometidas a cargas estáticas, y sus posibles características, se concluyó que el programa de computador debía contar con:

\begin{itemize}
    \item un procesador que analice la información del modelo de la estructura y la resuelva, siendo las rutinas empleadas accesibles a cualquier usuario
    
    \item un ambiente virtual tridimensional en el cual se pueda ingresar la información del modelo, se presente dicha información y la solución del mismo, el cual sea del agrado del usuario
    
    \item rutinas a código abierto, de manera que se pudieran estudiar, modificar y ampliar
\end{itemize} 

Una vez se determinó cómo debía ser el programa de computador, se encontró la necesidad de contar con  ciertas herramientas para dotar al programa de computador con la capacidad de resolver el problema en cuestión y que tuviera las características deseadas. \\

Lo anterior fue el insumo para desarrollar un programa de computador llamado \textit{StressUN}, en honor al programa de computador \textit{STRESS} (\textit{STRuctural Engineering System Solver}), desarrollado por un grupo de ingenieros en cabeza del profesor Fenves en la década de los años sesenta en el \textit{MIT} (de sus siglas en inglés, \textit{Massachusetts Institute of Technology}) \cite{fenves1965referenceUser}, dado que este trabajo comparte muchos de los criterios con los que el profesor concibió su programa. \\

A continuación se presenta la filosofía con la cual fue desarrollado el programa \textit{StressUN}, las herramientas empleadas para dicho desarrollo, se discute el porqué de dicha selección, los diferentes elementos que componen al programa y como funcionan.

\section{Filosofía de \textit{StressUN}}

El programa de computador \textit{StressUN} se concibió como una herramienta para solucionar modelos de estructuras reticulares tridimensionales sometidas a cargas estáticas que combinara las características de los programas tanto académicos como comerciales, para convertirse en una opción llamativa en las universidades. \\

Para lograr este objetivo, el programa \textit{StressUN} se caracteriza por ser práctico, es decir, en él se puede describir el modelo de la estructura y obtener los resultados de la solución de forma fácil e intuitiva y, por otro lado, se puede conocer cada una de las rutinas empleadas. \\

En dirección paralela a solucionar dichos modelos, se busca que \textit{StressUN}, en un futuro, se convierta en una opción para diseñadores e investigadores en su quehacer, el cual sea competitivo frente a programas comerciales, para lo cual, se analizó minuciosamente cada una de las herramientas disponibles que permitieran lograr tal objetivo ulterior. \\

De manera que no basta que solucione modelos de estructuras tridimensionales reticulares sometidas a cargas estáticas, ni que cuente con un ambiente virtual tridimensional donde se pueda ver la información del modelo de la estructura de forma interactiva, si no que se tenga acceso a sus rutinas, se les pueda modificar y agregar otras nuevas. \\

Para cumplir con las expectativas, se eligieron varias herramientas que se encuentran a la vanguardia y se implementarion en el programa de computador \textit{StressUN}, de tal manera que las rutinas de éste fueran fácilmente entendibles, modificables y ampliable.

\section{Componentes de \textit{StressUN}}

El programa de computador \textit{StressUN} está constituido por un \textit{pre} y \textit{pos procesador}, llamado \textit{StressUN}, y un conjunto de rutinas de programación llamado \textit{pyFEM}. El pre y posprocesador \textit{StressUN} consiste en un ambiente virtual tridimensional interactivo, el cual permite el ingreso de la información del modelo de la estructura, la visualización de dicha información y de los resultados de la solución, mientras las rutinas de programación \textit{pyFEM} provee las herramientas para encontrar la solución del modelo. \\

Para desarrollar cada uno de estos componentes, se uso el lenguaje de programación \textit{python} \cite{Rossum}, el ecosistema para ingeniería \textit{SciPy} \cite{Enthought} y el entorno de trabajo para visualizaciones tridimensionales \textit{Panda3D} \cite{TheWaltDisneyCompany}. \\

\subsection{Lenguaje de programación \textit{python}}

Se escogió el lenguaje de programación \textit{python} sobre los demás lenguajes presentados en el capítulo \textit{Antecedentes} porque
\begin{itemize}
    \item es fácil de aprender, lo que permite que se escriban instrucciones deseadas en un menor tiempo en comparación con otros lenguajes,
    \item la sintaxis hace que sea fácil de leer, de manera que el lector entenderá más rápido rutinas ya escritas,
    \item es interpretado, haciendo que el tiempo de desarrollo de rutinas sea menor al evitar el proceso de compilación para evidenciar errores en \textit{tiempo de ejecución}. Adicionalmente, al ser interpretado, se hace necesario tener todas las rutinas que se van a ejecutar (en lugar de un archivo compilado), de manera que pueden ser revisadas en cualquier momento,
    \item es multiproposito, por tanto se puede usar en aplicaciones de ingenieria hasta \textit{aplicaciones web}, pasando por administración de bases de datos, interfaces gráficas de usuario, aprendizaje de máquina, entre muchas otras,
    \item es multiplataforma, así que las mismas rutinas se pueden ejecutar en los principales sistemas operativos (\textit{Windows}, \textit{macOS} y las diferentes distribuciones de \textit{Linux}). Adicionalmente, las interfaces gráficas son \textit{nativas} en cada uno de los diferentes sistemas operativos,
    \item soporta, entre otros, el paradigma de programación orientada a objetos, lo cual permite evitar la redundancia de código, gracias a la generación de múltiples instancias de una misma clase y a la personalización de éstas vía herencia. También permite que los objetos tengan un comportamiento similar a los \textit{objetos primitivos} mediante la \textit{sobrecarga de operadores},
    \item es gratuito y a código abierto, en consecuencia no hay que pagar para usarlo, los programas desarrollados con él no están limitados por licencias y cuenta con una comunidad bastante activa presta a resolver las dudas que se presenten.
\end{itemize}

Sin embargo, una desventaja que presenta este lenguaje de programación es que toma más tiempo para ejecutar las mismas instrucciones que otros lenguaje de programación, como \textit{C++} o \textit{Java}. No obstante, existen implementaciones de \textit{python} como \textit{CPython} que permiten ejecutar instrucciones de \textit{C++} desde \textit{python}, haciendo de tal desventaja pequeña, de manera que no es un criterio para dejar de usar este lenguaje.

\subsection{Pre y pos procesador \textit{StressUN}}

Para desarrollar el ambiente virtual tridimensional interactivo se debía hacer uso de una herramienta que proveyera una \textit{escena}, la cual permite posicionar objetos en un espacio tridimensional, representarlos en la pantalla del computador desde el punto de vista de un observador y poder interactuar con ellos a través del ratón del computador. \\

Se escogió la librería \textit{Panda3D} sobre las demás librerías presentadas en el capítulo \textit{Antecedentes} porque
\begin{itemize}
    \item se puede usar con \textit{python}, de manera que todas las rutinas del programa de computador se hacen en el mismo lenguaje de programación, 
    \item las rutinas que se ejecutan están escritas en \textit{C++}, donde \textit{python} es un \textit{wrapper}, lo que permite un optimo desempeño del programa de computador al ejecutar rutinas compiladas, 
    \item es multiplataforma, así que las mismas rutinas se pueden ejecutar en los principales sitemas operativos (\textit{Windows}, \textit{macOS} y las diferentes distribuciones de \textit{Linux}), 
    \item cuenta con una comunidad activa, así que se puede acudir a ella en caso de tener problmas al tratar de implementar alguna rutina, 
    \item es gratuito y a código abierto, en consecuencia no hay que pagar para para usarlo y, al igual que \textit{python}, los programadas desarrollados con ella se pueden comercializar
\end{itemize}

Sin embargo, una desventaja que presenta esta libreria es que la documentación no está completa, lo que conlleva a especular o ignorar el comportamiento de la misma.

\subsection{pyFEm}

Para desarrollar las rutinas de programación \textit{pyFEM} se debía hacer uso de una herramienta que proveyera \textit{operaciones matriciales}, las cuales permitan sumar, restar y multiplicar matrices. \\

Se escogió el ecosistema para matemáticas, ciencias e ingeniería \textit{sciPy}, el cual está compuesto, entre otras, por las librerías \textit{NumPy}, \textit{SciPy}, \textit{Matplotlib} y \textit{pandas}, las que proveen arreglos \textit{n-dimensionales}, rutinas de programación científica (integración numerica y optimización), gráficas bidimensionales y análisis de datos estructurados, respectivamente. \\

Se escogió dicho ecosistemas sobre las demás librerias presentadas en el capítulo \textit{Antecedentes} porque:
\begin{itemize}
    \item se puede usar con \textit{python}, de manera que todas las rutinas del programa de computador se hacen en el mismo lenguaje de programación, 
    \item es multiplataforma, así que las mismas rutinas se pueden ejecutar en los principales sistemas operativos (\textit{Windows}, \textit{macOS} y las diferentes distribuciones de \textit{Linux}), 
    \item cuenta con una comunidad bastante activa, así que se puede acudir a ella en caso de tener probemas al tratar de implementar alguna rutina, 
    \item se puede integrar con \textit{C++} y \textit{Fortran}, de manera que se puede usar rutinas escritas en estos lenguajes o implementar las propias para optimizar el tiempo de cálculo, 
\end{itemize}

Sin embargo, una desventaja que se presenta en esta librería es que toma más tiempo para ejecutar las mismas instrucciones que otras librerías en otros lenguajes de programación, como \textit{C++} o \textit{C}. No obstante, esta librería está bastante optimizada, tanto así que si se llega a presentar que el tiempo de ejecución es bastante grande, esto se presentaría en otros lenguajes.

\subsection{Integración entre \textit{python} y las librerías}

En las secciones anteriores se presentaron las herramientas que fueron usadas para desarrollar el programa de computador \textit{StressUN}. A modo de resumen, dichas herramientas consisten en el lenguaje de programación \textit{python}, la liberría \textit{Panda3D} y el ecosistema \textit{sciPy}. \\

Estas herramientas cuentan con la gran ventaja que todas se pueden usar bajo el mismo lenguaje de programación, de manera que no hay necesidad de aprender más de un lenguaje de programación. \\

Así mismo, \textit{python} cuenta con la ventaje que, al ser interpretado, se deba tener las rutinas necesarias para ejecutar el programa, de manera tal que estas pueden ser leídas, estudiadas y modificadas en cualquier momento. \\

\subsection{Sistema de control de versiones}

Durante el desarrollo del programa de computador \textit{StressUN} se evidenció la necesidad de llevar el control de cada una de las nuevas rutinas implementadas, para lo cual se optó por usar el sistema de versión de controles \textit{GitHub}. \\

\textit{Github} permite guardar estados en el desarrollo del código, de manera que se puede experimentar con rutinas de programación y, una vez sean existosas, crear un punto de control al cual se puede regresar en cualquier momento.

\subsection{Classtools}

\begin{codigoprog}
class AttrDisplay:
    def gather_attrs(self):
    def __repr__(self):
\end{codigoprog}

\begin{codigoprog}
class Collection:
    def __init__(self):
    def add(self, obj):
    def _labels(self):
    def __getitem__(self, item):
    def __iter__(self):
    def __len__(self):
    def __repr__(self):
\end{codigoprog}

\subsection{Primitives}

\begin{codigoprog}
class Material(AttrDisplay):
    def __init__(self, label, modulus_elasticity, modulus_elasticity_shear):
    def __eq__(self, other):
\end{codigoprog}

\begin{codigoprog}
class Section(AttrDisplay):
    def __init__(self, label, material, area, moment_inertia_y, moment_inertia_z, torsion_constant):
    def __eq__(self, other):
\end{codigoprog}

\begin{codigoprog}
class Node(AttrDisplay):
    def __init__(self, label, x, y, z):
    def set_degrees_freedom(self, u):
    def __eq__(self, other):
\end{codigoprog}

\begin{codigoprog}
class Truss(AttrDisplay):
    number_dimensions = 3
    number_nodes = 2
    number_degrees_freedom_per_node = 3

    def __init__(self, label, node_i, node_j, section):
    def get_local_vector(self):
    def get_orientation(self):
    def get_matrix_transformation(self):
    def get_local_stiff_matrix(self):
    def get_global_stiff_matrix(self):
    def get_modulus(self):
    def get_area(self):
    def get_length(self):
    def get_forces(self, load_pattern):
    def __eq__(self, other):
\end{codigoprog}

\begin{codigoprog}
class Frame(Truss):
    number_degrees_freedom_per_node = 6
    
    def get_local_stiff_matrix(self):
\end{codigoprog}

\begin{codigoprog}
class Support(AttrDisplay):
    def __init__(self, node, ux, uy, uz, rx, ry, rz):
    def __eq__(self, other):
\end{codigoprog}

\begin{codigoprog}
class PointLoad(AttrDisplay):
    def __init__(self, node, fx, fy, fz):
    def __eq__(self, other):
\end{codigoprog}

\begin{codigoprog}
class DistributedLoad(AttrDisplay):
    def __init__(self, frame, fx, fy, fz):
    def __eq__(self, other):
        return self.frame == other.frame
\end{codigoprog}

\begin{codigoprog}
class Displacement(AttrDisplay):
    def __init__(self, load_pattern, ux, uy, uz, rx, ry, rz):
\end{codigoprog}

\begin{codigoprog}
class Reaction(AttrDisplay):
    def __init__(self, load_pattern, reactions):
\end{codigoprog}

\begin{codigoprog}
class LoadPattern(AttrDisplay):
    def __init__(self, label, parent):
    def get_f(self):
    def __eq__(self, other):
\end{codigoprog}

\begin{codigoprog}
class PointLoads(Collection):
    def __init__(self, parent):
    def add(self, node, fx, fy, fz):
\end{codigoprog}

\begin{codigoprog}
class DistributedLoads(Collection):
    def __init__(self, parent):
    def add(self, frame, fx, fy, fz):
\end{codigoprog}

\begin{codigoprog}
class Displacements(Collection):
    def __init__(self):
    def add(self, load_pattern, ux, uy, uz, rx, ry, rz):
\end{codigoprog}

\begin{codigoprog}
class Reactions(Collection):
    def __init__(self):
    def add(self, load_pattern, reactions):
\end{codigoprog}


 \subsection{core}
 
\begin{codigoprog}
class Materials(Collection):
    def __init__(self, parent):
    def add(self, label, modulus_elasticity, modulus_elasticity_shear):
\end{codigoprog}

\begin{codigoprog}
class Sections(Collection):
    def __init__(self, parent):
    def add(self, label, material, area, inertia_y, inertia_z, torsion_constant):
\end{codigoprog}

\begin{codigoprog}
class Nodes(Collection):
    def __init__(self, parent):
    def add(self, label, x, y, z):
\end{codigoprog}

\begin{codigoprog}
class Trusses(Collection):
    def __init__(self, parent):
    def add(self, label, node_i, node_j, section):
\end{codigoprog}

\begin{codigoprog}
class Frames(Trusses):
    def __init__(self, parent):
    def add(self, label, node_i, node_j, section):
\end{codigoprog}

\begin{codigoprog}
class Supports(Collection):
    def __init__(self, parent):
    def add(self, node, ux, uy, uz, rx, ry, rz):
\end{codigoprog}

\begin{codigoprog}
class LoadPatterns(Collection):
    def __init__(self, parent):
    def add(self, label):
\end{codigoprog}

\begin{codigoprog}
class Structure:
    number_degrees_freedom_per_node = 6
    number_dimensions = 3

    def __init__(self):
    def set_degrees_freedom(self):
    def get_k(self):
    def solve(self):
    def __repr__(self):
\end{codigoprog}



% \section{StressUN}


% Los programas de computador comerciales para el análisis de estructuras mediante el método matricial que se encuentran vigentes cuentan, en general, con un entorno gráfico que le permite al usuario introducir los datos del modelo de forma interactiva, corregirlos, procesarlos y visualizar los resultados. \\

% Dado que la intención de este trabajo es desarrollar un programa de computador a código abierto que cuente con características similares a los programas comerciales, en este capítulo se inicia identificando dichas características. Así mismo, se estudian las características de los programas a código abierto. \\

% \section{Revisión de los programas comerciales}
% La intensión de este trabajo fue realizar un programa de computador llamado \textit{StressUN} similar a los programas comerciales, de tal manera que se procedió a estudiar los diferentes elementos que los caracterizan. \\

% Por ejemplo, SAP2000\textsuperscript{\textregistered} es un de estos programas de computador, ampliamente conocido en el ámbito colombiano. Para usarlo, el usuario primeramente debe definir las características de la estructura mediante una serie de menús, como se muestra en la figura \ref{fig:sap2000_toolbar}. El programa permite establecer tipos de materiales, secciones transversales de los elementos, patrones de carga, entre otras características.

% \begin{figure}[ht]
%     \centering
%     \includegraphics[width=1\textwidth]{metodologia/sap2000_toolbar.png}
%     \caption{Menús del programa de computador SAP2000\textsuperscript{\textregistered} para definir las características de la estructura.}
%     \label{fig:sap2000_toolbar}
% \end{figure}

% Una vez definidas las características de la estructura, el usuario procede a ingresar los diferentes elementos del modelo de la estructura en el entorno virtual de manera interactiva, mediante la ayuda de una serie de ejes y de los menús del programa. \\

% El entorno virtual consiste en un ambiente tridimensional donde el usuario puede ver, ingresar e interactuar con los diferentes elementos visuales del modelo.\\

% La serie de ejes consiste en un grupo de líneas en el espacio que se interceptan en un conjunto de puntos, los cuales sirven de referencia para que el usuario pueda ingresar los diferentes elementos del modelo al entorno virtual. En la figura \ref{fig:sap2000_axes} se presenta un conjunto de dichos ejes en el entorno virtual.\\
% \begin{figure}[ht]
%     \centering
%     \includegraphics[width=1\textwidth]{metodologia/sap2000_axes.png}
%     \caption{Entorno virtual de SAP2000\textsuperscript{\textregistered} con un conjunto de ejes definido.}
%     \label{fig:sap2000_axes}
% \end{figure}

% Los menús del programa de computador que permiten ingresar los elementos del modelo consisten en aquellos que permiten escoger el tipo de elemento, como se muestra en la figura \ref{fig:sap200_draw}, y aquellos que permiten modificar los elementos ya ingresados, los cuales permiten moverlos, copiarlos, o eliminarlos. \\
% \begin{figure}[ht]
%     \centering
%     \includegraphics[width=1\textwidth]{metodologia/sap2000_draw.png}
%     \caption{Menú de SAP2000\textsuperscript{\textregistered} que permite ingresar diferentes tipos de elementos.}
%     \label{fig:sap200_draw}
% \end{figure}

% Una vez el usuario haya agregado los diferentes elementos de la estructura puede modificar sus condiciones de apoyo, las cargas, entre otros, mediante el uso de menús, de manera que el modelo esté listo para que el programa ejecute el análisis correspondiente. En la figura \ref{fig:sap2000_model} se presenta el modelo de una estructura terminado.
% \begin{figure}[ht]
%     \centering
%     \includegraphics[width=1\textwidth]{metodologia/sap2000_model.png}
%     \caption{Modelo de una estructura en  SAP2000\textsuperscript{\textregistered}.}
%     \label{fig:sap2000_model}
% \end{figure}

% Una vez se han surtido los pasos anteriores, el usuario puede correr el análisis para obtener los resultados del mismo. El usuario puede visualizar los resultados en el entorno virtual, como se muestra en la figura \ref{fig:sap2000_deformed}, o mediante tablas.
% \begin{figure}[ht]
%     \centering
%     \includegraphics[width=1\textwidth]{metodologia/sap2000_deformed.png}
%     \caption{Deformación de la estructura después del análisis de SAP2000\textsuperscript{\textregistered}.}
%     \label{fig:sap2000_deformed}
% \end{figure}

% Así como el programa SAP2000\textsuperscript{\textregistered}, los otros programas comerciales, como ETABS \textsuperscript{\textregistered}, CSIBridge\textsuperscript{\textregistered}, midas\textsuperscript{\textregistered}, entre otros, cuentan con herramientas similares para que el usuario pueda analizar modelos de estructuras.

% \section{Revisión de los programas a código abierto}
% Existen gran variedad de programas a código abierto. Entre los programas que más se han estudiado para este trabajo se encuentra \textit{ANALEST}. \\

% El programa \textit{ANALEST} comprende una serie de subprogramas cuyo objeto es servir de ayuda en el análisis y diseño de estructuras. El programa está programado en \texttt{BASIC} y su característica principal es analizar 
% \begin{inparaenum}[$ (1) $]
%     \item vigas continúas, 
%     \item armaduras planas, 
%     \item armaduras en el espacio, 
%     \item pórticos planos, 
%     \item pórticos en el espacio, 
%     \item parrillas planas.
% \end{inparaenum}

% Para usar los subprogramas, se debe
% \begin{inparaenum}[$ (1) $]
%     \item introducir los datos del modelo de la estructura, revisarlos y guardarlos, 
%     \item introducir los datos de carga, revisarlos y guardarlos, 
%     \item ejecutar el análisis, guardar los resultados y presentarlos
% \end{inparaenum}

% Ademas, los subprogamas de vigas y armaduras permiten encontrar la envolvente de momentos en los apoyos y de fuerzas axiales, respectivamente, como paso preliminar para el diseño de los miembros. \\

% Guardar los datos estructurales y los datos de carga es muy útil en aquellos casos en que conviene modificar las dimensiones de los elementos con base en los resultados del primer análisis. Por otra parte, guardar los resultados facilita el estudio de las combinaciones de carga disminuyendo por una parte los datos requeridos y por otra el tiempo de ejecución, ya que se aprovecha el principio de superposición. Adicionalmente, guardar dichos resultados facilita el enlace de los subprogramas con programas de diseño, bien sean adquiridos o desarrollados por el usuario. \\

% \section{Identificación de los elementos a programar}

% De la revisión de los diferentes programas comerciales se dedujo que el programa StressUN debía dividirse en tres partes y que tenían que trabajar en conjunto. Dichas partes son: \textit{preproceso}, \textit{proceso} y \textit{posproceso}. \\

% El preproceso consiste en la adquisición de todos los datos relevantes del modelo de la estructura a analizar. El proceso realiza el tratamiento de los datos del modelo, mientras que el posproceso presenta los resultados del análisis del modelo. \\

% Tanto el preproceso como el posproceso necesitan de un ambiente gráfico, mientras que el proceso necesita de las rutinas propias del análisis matricial. \\

% \section{Selección de las herramientas de programación}
% Los dos grandes problemas a solucionar consistieron en desarrollar el ambiente gráfico, el cual se compone de menús y un entorno virtual tridimensional, y el núcleo de StressUN. Para esto, se consultó las diferentes herramientas de programación disponibles, donde se decidió utilizar la librería \textit{frames}, la cual está programada en el \textit{Java}, y el conjuto de librerías \textit{Scipy}, la cual está programada en \textit{Python}.\\

% La librería frames consiste en un conjunto de herramientas para crear un entorno virtual bidimensional o tridimensional interactivo. Dicha librería trabaja como una extensión de la librería \textit{processing}, la cual también está programada en Java. \\

% La librería processing es un conjunto de herramientas dirigida a solucionar los problemas relacionados con la computación gráfica, permitiendo a los desarrolladores desde crear imágenes hasta entornos virtuales tridimensionales. \\

% Por otro lado, el conjunto de librerías Scipy consiste en herramientas para solucionar problemas relacionados algebra matricial. Dicho conjunto de librerías está conformado por \textit{Numpy}, \textit{Scipy}, \textit{matplotlib}, entre otros. La librería Numpy provee arreglos multidimensionales y operaciones entre ellos. La librería \textit{Scipy} trata problemas del algebra matricial, como son la solución de sistemas de ecuaciones líneales, mientras que matplotlib permite generar diferentes tipos de gráficos. \\

% \section{Desarrollo del programa de computador}

% Una vez se identificaron los diferentes elementos necesarios con los que debía contar el programa de computador y se escogieron las herramientas de trabajo, se realizó una revisión bibliográfica de la formulación matemática de los métodos matriciales aplicados al análisis estructural, enfocada al análisis de estructuras tridimensionales de respuesta lineal. Adicionalmente, se realizó una revisión a la documentación de la librería frames.\\

% De dicho ejercicio se identificaron los datos de entrada que el usuario necesita definir y los datos de salida que espera obtener
% \begin{itemize}
%     \item \textit{Preproceso}
%     \begin{itemize}
%         \item Definición de los materiales.
%         \item Definición de las secciones transversales.
%         \item Definición de los nudos.
%         \item Definición de los elementos.
%         \item Definición de las condiciones de apoyo.
%         \item Definición de los casos de carga.
%         \item Definición de las combinaciones de carga.
%         \item Definición de las patrones de carga
%     \end{itemize}
%     \item \textit{Posproceso}
%     \begin{itemize}
%         \item Visualización de los desplazamientos en los nudos.
%         \item Visualización de las reacciones.
%         \item Visualización de las fuerzas internas.
%     \end{itemize}
    
% \end{itemize}

%  \textit{StressUN} se desarrolló usando el paradigma de \textit{programación orientada a objetos}, \textit{OOP} (de sus siglas en inglés \textit{object-oriented programming}), y en forma modular, de manera que el \textit{preproceso} y el \textit{posproceso} son independientes del \textit{proceso}.\\

% A medida que se definieron los diferentes elementos a tener en cuenta, se fueron programando. Es decir, se desarrollaron las clases \textit{primitivas} del programa, las cuales representan la abstracción de los elementos de las entidades más sencillas del problema, las cuales consisten en las clases
% \begin{itemize}
%     \item \textit{Material}.
%     \item \textit{Section}.
%     \item \textit{Node}.
%     \item \textit{Frame}.
%     \item \textit{Support}.
%     \item \textit{PointLoad}.
%     \item \textit{LoadPattern}.
%     \item \textit{Displacement}.
% \end{itemize}

% Las clases anteriormente listadas, tienen su representación tanto en el preproceso y posproceso, como en el proceso. Es decir, en el preproceso y posproceso dichas clases tienen su representación gráfica, mientras que en el proceso, éstos tienen su representación matemática. Sin embargo, dichas clases son separadas unas de las otras, de manera que se asegure que el preproceso y posproceso son independientes del proceso.

% Una vez programados las entidades más básicas del programa, se procedió a crear la clase \textit{Structure}, la cual se encarga de administrar otros objetos. Dicho paradigma de programación se conoce como \textit{composición}. Los objetos que contiene la clase \textit{Structure} son

% \begin{itemize}
%     \item \textit{Materials}.
%     \item \textit{Sections}.
%     \item \textit{Nodes}.
%     \item \textit{Frames}.
%     \item \textit{Supports}.
%     \item \textit{LoadPatterns}.
% \end{itemize}

% Cada una de las anteriores clases es la interfaz entre el programa y el usuario, donde este último podrá agregar nuevos materiales, secciones, nudos, elementos tipo pórtico, condiciones de apoyo y condiciones de carga. Dichas clases componen el \textit{núcleo} del programa.\\

% El proceso anteriormente descrito se desarrolló bajo los diferentes mecanismos que provee la programación orientada objeto, los cuales son encapsulación, herencia y sobre carga. Estas herramientas, bien aplicadas, permiten la reutilización del código y el mantenimiento del mismo. Así mismo, se utilizó la herramienta de \textit{Git}, la cual es un sistema de control de versiones, la cual permite llevar el control absoluto durante el proceso de desarrollo del código.\\

% \section{Verificación del programa}

% Una vez se llegó a una versión estable del programa, este se puso a prueba mediante la solución de diferentes problemas que aparecen en la bibliografía, donde se comparó la respuesta obtenida con la presentada. Así mismo, se comparó el desempeño del programa frente a otros programas, tanto comerciales como académicos, en cuanto a la facilidad de uso como al tiempo de computo. \\
% \chapter{Ejemplos de aplicación}
\label{chap:ejemplos_aplicacion}

Con el fin de presentar el funcionamiento de los programas desarrollados, en este capítulo se presentan diferentes modelos de estructuras tridimensionales de elementos rectos sometidos a cargas estáticas, los cuales se resuelven usando la librería \textit{pyFEM} o el programa de computador \textit{StressUN}. \\

\section{Funcionamiento de la librería \textit{pyFEM}} \label{sec:libreriaPyFem}
Se presentan varios ejemplos provenientes de los libros de la referencia bibliográfica, los cuales se resuelven usando el programa \textit{StressUN} o la librería \textit{pyFEM}, con el objetivo de comparar la respuesta obtenida con la presentada en dichos textos.

\subsection{Cercha plana}

En el libro \textit{Microcomputadores en Ingeniería Estructural}, del ingeniero \textit{Jairo Uribe Escamilla} (\cite{escamilla1995microcomputadores}), se presenta el ejemplo y la solución del modelo de una cercha simple plana estáticamente determinada (una cercha simple plana es una estructura compleja generada a partir de una estructura triangular base \cite{beer1997mecanica}). El ejercicio consiste en encontrar los desplazamientos de los nudos, las fuerzas de las reacciones y las fuerzas de cada elemento de la cercha mostrada en la figura \ref{fig:primer_punto}. La solución presenta las matrices de rigidez de cada uno de los elementos de la cercha, la matriz de rigidez de toda la estructura, la solución de los desplazamientos de los nodos, las reacciones de los apoyos y las fuerzas internas de cada elemento.\\

A continuación se presenta la solución de dicho ejemplo, el cual se resuelve usando la librería \texttt{pyFEM}. \\

\textit{\textbf{Ejemplo 7.1 -} Resuelva completamente la cercha mostrada por el método matricial de los desplazamientos. El material es acero estructural con $ E = 2040\; t / cm^2 $. Las áreas están dadas entre paréntesis en $ cm^2 $.}\\

\textit{\textbf{Solución -}} Para solucionar el ejemplo anterior usando la librería \texttt{pyFEM}, se ingresa la información necesaria para describir completamente el modelo de la estructura mediante instrucciones en el lenguaje de programación \texttt{python}, las cuales se almacenan en un archivo de texto plano. Dicha información consiste en:
\begin{inparaenum} [$ (1) $]
    \item los materiales,
    \item las secciones transversales,
    \item los nodos,
    \item los elementos tipo cercha,
    \item los apoyos y
    \item las cargas.
\end{inparaenum} \\

\begin{figure}[t]
    \centering
    
    \begin{tikzpicture}
        % Scaling
        % \scaling{.25};
        
        % Points
        \point{1}{0}{0};
        \point{2}{8}{0};
        \point{3}{4}{3};
        \point{4}{4}{0};
        
        \point{p}{4}{-1.3};
        \point{q}{5.05}{3.7875};
        
        \point{1-3}{2}{1.5};
        \point{1-4}{2}{0};
        \point{3-2}{6}{1.5};
        \point{4-2}{6}{0};
        \point{4-3}{4}{1.5};
        
        % Beams
        \beam{2}{1}{3}[0][0];
        \beam{2}{1}{4}[1][1];
        \beam{2}{3}{2}[1][1];
        \beam{2}{4}{2}[1][1];
        \beam{2}{4}{3}[1][1];
        
        % Supports
        \support{1}{1};
        \support{2}{2};
        
        % Joints
        \hinge{1}{1};
        \hinge{1}{2};
        \hinge{1}{3};
        \hinge{1}{4};
        
        % Loads
        \load{1}{q}[216.87];
        \load{1}{p}[90];
        
        % notation
        \notation{1}{1}{1}[above left];
        \notation{1}{2}{2}[above right];
        \notation{1}{3}{3}[above left];
        \notation{1}{4}{4}[below left];
        
        \notation{1}{3}{$ \SI{5}{\tonne} $}[above right=10mm];
        \notation{1}{4}{$ \SI{20}{\tonne} $}[below right=5mm and 0mm];
        
        \notation{5}{1}{3}[$ (100) $ ][.5][above][0.5];
        \notation{5}{1}{4}[$ (40) $ ][.5][above][0.5];
        \notation{5}{3}{2}[$ (150) $ ][.5][above][0.5];
        \notation{5}{4}{2}[$ (40) $ ][.5][above][0.5];
        \notation{5}{4}{3}[$ (30) $ ][.5][right][1];
        
        
        % Dimensions
        \dimensioning{1}{1}{4}{-1.5}[$ \SI{4}{m} $];
        \dimensioning{1}{4}{2}{-1.5}[$ \SI{4}{m} $];
        \dimensioning{2}{2}{3}{9}[$ \SI{3}{m} $];
    \end{tikzpicture}
    
    \caption{Cercha simple plana del \textit{Ejemplo 7.1} de \cite{escamilla1995microcomputadores}.}
    \label{fig:primer_punto}
\end{figure}

En el algoritmo \ref{alg:cercha_plana} se presentan las instrucciones que debe recibir \texttt{pyFEM} para solucionar el modelo de la estructura. Las instrucciones consisten en crear un nuevo objeto tipo \texttt{Structure}, al cual se le ha dado el nombre de \textit{structure} y, seguidamente, se agregan los materiales, las secciones, los nodos, los elementos tipo cercha, los apoyos, los patrones de carga y las cargas en los nodos. Una vez los datos del modelo de la estructura han sido ingresados, se puede solucionar el modelo mediante la instrucción \texttt{structure.solve()}. \\

\begin{lstlisting}[language=Python,caption=Ingreso de los datos del modelo de la estructura a \textit{pyFEM}.,label=alg:cercha_plana, frame=single]
# create structure
structure = Structure()

# add material
structure.materials.add("acero", 2040e4)

# add sections
structure.sections.add("section1", "acero", 30e-4)
structure.sections.add("section2", "acero", 40e-4)
structure.sections.add("section3", "acero", 100e-4)
structure.sections.add("section4", "acero", 150e-4)

# add nodes
structure.nodes.add('1', 0, 0, 0)
structure.nodes.add('2', 8, 0, 0)
structure.nodes.add('3', 4, 3, 0)
structure.nodes.add('4', 4, 0, 0)

# add trusses
structure.trusses.add('1-3', '1', '3', "section3")
structure.trusses.add('1-4', '1', '4', "section2")
structure.trusses.add('3-2', '3', '2', "section4")
structure.trusses.add('4-2', '4', '2', "section2")
structure.trusses.add('4-3', '4', '3', "section1")

# add support
structure.supports.add('1', True, True, True)
structure.supports.add('2', False, True, True)
structure.supports.add('3', False, False, True)
structure.supports.add('4', False, False, True)

# add load pattern
structure.load_patterns.add("point loads")

# add point loads
structure.load_patterns["point loads"].point_loads.add('4', 0, -20, 0)
structure.load_patterns["point loads"].point_loads.add('3', 5*0.8, 5*0.6, 0)

# solve the structure
structure().solve()
\end{lstlisting}

Cuando se ejecuta la instrucción \texttt{structure.solve()}, \texttt{pyFEM} comienza a solucionar el modelo de la estructura con base en la información ingresada. Los pasos que se efectúan para solucionar la estructura consisten en: 
\begin{inparaenum}[$ (1) $]
    \item asignar los grados de libertad de los nodos, 
    \item ensamblar la matriz de rigidez del modelo de la estructura, 
    \item imponer las condiciones de apoyo en la matriz de rigidez del modelo, 
    \item ensamblar el vector de fuerzas en los nodos para cada uno de los patrones de carga, 
    \item imponer las condiciones de apoyo en el vector de fuerzas en los nodos para cada caso de carga, 
    \item encontrar los desplazamientos de los nodos para cada patrón de carga, 
    \item encontrar las reacciones en los apoyos para cada patrón de carga y
    \item guardar la solución en los nodos y en los apoyos para cada patrón de carga.
\end{inparaenum} \\

A continuación se presenta el resultado de cada uno de los pasos realizados por \texttt{pyFEM} para solucionar la cercha del ejemplo 7.1. \\

\subsubsection{Grados de libertad de los nodos}
Para realizar el ensamblaje de la matriz de rigidez del modelo de la estructura y del vector de fuerzas de los nodos, se asignan números a los grados de libertad de los nodos de la estructura en el orden en que fueron ingresados. \\

Con base en ésto y en el algoritmo \ref{alg:cercha_plana}, el nodo \textit{'1'} se le han asignado los grados de libertad \textit{0, 1} y \textit{2}, al nodo \textit{'2'} los grados de libertad \textit{3, 4} y \textit{5}, y así sucesivamente.

\subsubsection{Matrices de rígidez}
Una vez se establecen los grados de libertad de los nodos del modelo de la estructura, se ensambla la matriz de rigidez del modelo de la estructura. Este proceso consiste en calcular una a una las matrices de rigidez de los elementos ensamblandolas en la matriz de rigidez del modelo, la cual, inicialmente, es una matriz de ceros. \\

Aunque no se almacenan las matrices de rigidez de cada uno de los elementos del modelo de la estructura, el usuario puede consultarlas. En la tabla \ref{tab:k_1_3} se presenta la matriz de rigidez en coordenadas globales del elemento \textit{1-3}, con sus respectivos grados de libertad, la cual se obtiene mediante la instrucción \textit{\textit{structure.trusses['1-3'].get\_global\_stiff\_matrix()}}. \\

\begin{table}[ht]
    \centering
    \begin{blockarray}{ccccccc}
        & \textit{0} & \textit{1} & \textit{2} & \textit{6} & \textit{7} & \textit{8} \\
        \begin{block}{c[cccccc]}
            \textit{0} & 26112 & 19584 & 0 & -261112 & -19584 & 0 \\
            \textit{1} & 19584 & 14688 & 0 & -19584 & -14688 & 0 \\
            \textit{2} & 0 & 0 & 0 & 0 & 0 & 0 \\
            \textit{3} & -26112 & -19584 & 0 & 26112 & 19584 & 0 \\
            \textit{4} & -19584 & -14688 & 0 & 19584 & 14688 & 0 \\
            \textit{5} & 0 & 0 & 0 & 0 & 0 & 0 \\
        \end{block}
    \end{blockarray} \si[per-mode=symbol]{\tonne\per\meter}
    \caption{Matriz de rigidez en coordenadas globales del elemento \textit{1-3}.}
    \label{tab:k_1_3}
\end{table}

En las tablas \ref{tab:k_1_4} a la \ref{tab:k_4_3} se presentan las matrices de rigidez en coordenadas globales de los demás elementos de la estructura, las cuales se obtienen con instrucciones similares a la anterior. \\

\begin{table}[h]
    \centering
    \begin{blockarray}{ccccccc}
        & \textit{0} & \textit{1} & \textit{2} & \textit{9} & \textit{10} & \textit{11} \\
        \begin{block}{c[cccccc]}
            \textit{0} & 20400 & 0 & 0 & -20400 & 0 & 0 \\
            \textit{1} & 0 & 0 & 0 & 0 & 0 & 0 \\
            \textit{2} & 0 & 0 & 0 & 0 & 0 & 0 \\
            \textit{9} & -20400 & 0 & 0 & 20400 & 0 & 0 \\
            \textit{10} & 0 & 0 & 0 & 0 & 0 & 0 \\
            \textit{11} & 0 & 0 & 0 & 0 & 0 & 0 \\
        \end{block} 
    \end{blockarray} \si[per-mode=symbol]{\tonne\per\meter}
    \caption{Matriz de rigidez en coordenadas globales del elemento \textit{1-4}.}
    \label{tab:k_1_4}
\end{table}

\begin{table}[ht]
    \centering
    \begin{blockarray}{ccccccc}
        & \textit{6} & \textit{7} & \textit{8} & \textit{3} & \textit{4} & \textit{5} \\
        \begin{block}{c[cccccc]}
            \textit{6} & 39168 & -29376 & 0 & -39168 & 29376 & 0 \\
            \textit{7} & -29376 & 22032 & 0 & 29376 & -22032 & 0 \\
            \textit{8} & 0 & 0 & 0 & 0 & 0 & 0 \\
            \textit{3} & -39168 & 29376 & 0 & 39168 & -29376 & 0 \\
            \textit{4} & 29376 & -22032 & 0 & -29376 & 22032 & 0 \\
            \textit{5} & 0 & 0 & 0 & 0 & 0 & 0 \\
        \end{block}
    \end{blockarray} \si[per-mode=symbol]{\tonne\per\meter}
    \caption{Matriz de rigidez en coordenadas globales del elemento \textit{3-2}.}
    \label{tab:k_3_2}
\end{table}

\begin{table}[ht]
    \centering
    \begin{blockarray}{ccccccc}
        & \textit{9} & \textit{10} & \textit{11} & \textit{3} & \textit{4} & \textit{5} \\
        \begin{block}{c[cccccc]}
            \textit{9} & 20400 & 0 & 0 & -20400 & 0 & 0 \\
            \textit{10} & 0 & 0 & 0 & 0 & 0 & 0 \\
            \textit{11} & 0 & 0 & 0 & 0 & 0 & 0 \\
            \textit{3} & -20400 & 0 & 0 & 20400 & 0 & 0 \\
            \textit{4} & 0 & 0 & 0 & 0 & 0 & 0 \\
            \textit{5} & 0 & 0 & 0 & 0 & 0 & 0 \\
        \end{block}
    \end{blockarray} \si[per-mode=symbol]{\tonne\per\meter}
    \caption{Matriz de rigidez en coordenadas globales del elemento \textit{4-2}.}
    \label{tab:k_4_2}
\end{table}

\begin{table}[H]
    \centering
    \begin{blockarray}{ccccccc}
        & \textit{9} & \textit{10} & \textit{11} & \textit{6} & \textit{7} & \textit{8} \\
        \begin{block}{c[cccccc]}
            \textit{9} & 0 & 0 & 0 & 0 & 0 & 0 \\
            \textit{10} & 0 & 20400 & 0 & 0 & -20400 & 0 \\
            \textit{11} & 0 & 0 & 0 & 0 & 0 & 0 \\
            \textit{6} & 0 & 0 & 0 & 0 & 0 & 0 \\
            \textit{7} & 0 & -20400 & 0 & 0 & 20400 & 0 \\
            \textit{8} & 0 & 0 & 0 & 0 & 0 & 0 \\
        \end{block}
    \end{blockarray} \si[per-mode=symbol]{\tonne\per\meter}
    \caption{Matriz de rigidez en coordenadas globales del elemento \textit{4-3}.}
    \label{tab:k_4_3}
\end{table}

Al comparar los resultados obtenidos con las matrices de rigidez de cada uno de los elementos presentados en \cite{escamilla1995microcomputadores} se encuentra que se trata de los mismos valores. \\

Así mismo como el usuario puede indagar por la matriz de rigidez en coordenadas globales de cada elemento, puede hacerlo para el modelo de la estructura, mediante la instrucción \textit{\textit{structure.get\_k()}}. Al hacerlo, obtiene los valores presentados en la tabla \ref{tab:k_global}.\\

\begin{table}[ht]
    \centering
    \begin{blockarray}{ccccccccccccc}
        & \textit{0} & \textit{1} & \textit{2} & \textit{3} & \textit{4} & \textit{5} & \textit{6} & \textit{7} & \textit{8} & \textit{9} & \textit{10} & \textit{11} \\
        \begin{block}{c[cccccccccccc]}
            \textit{0} & 46512 & 19584 & 0 & 0 & 0 & 0 & -26112 & -19584 & 0 & -20400 & 0 & 0 \\
            \textit{1} & 19584 & 14688 & 0 & 0 & 0 & 0 & -19584 & -14688 & 0 & 0 & 0 & 0 \\
            \textit{2} & 0 & 0 & 0 & 0 & 0 & 0 & 0 & 0 & 0 & 0 & 0 & 0 \\
            \textit{3} & 0 & 0 & 0 & 59568 & -29376 & 0 & -39168 & 29376 & 0 & -20400 & 0 & 0 \\
            \textit{4} & 0 & 0 & 0 & -29376 & 22032 & 0 & 29376 & -22032 & 0 & 0 & 0 & 0 \\
            \textit{5} & 0 & 0 & 0 & 0 & 0 & 0 & 0 & 0 & 0 & 0 & 0 & 0 \\
            \textit{6} & -26112 & -19584 & 0 & -39168 & 29376 & 0 & 65280 & -9792 & 0 & 0 & 0 & 0 \\
            \textit{7} & -19584 & -14688 & 0 & 29376 & -22032 & 0 & -9792 & 57120 & 0 & 0 & -20400 & 0 \\
            \textit{8} & 0 & 0 & 0 & 0 & 0 & 0 & 0 & 0 & 0 & 0 & 0 & 0 \\
            \textit{9} & -20400 & 0 & 0 & -20400 & 0 & 0 & 0 & 0 & 0 & 40800 & 0 & 0 \\
            \textit{10} & 0 & 0 & 0 & 0 & 0 & 0 & 0 & -20400 & 0 & 0 & 20400 & 0 \\
            \textit{11} & 0 & 0 & 0 & 0 & 0 & 0 & 0 & 0 & 0 & 0 & 0 & 0 \\
        \end{block}
    \end{blockarray} \si[per-mode=symbol]{\tonne\per\meter}
    \caption{Matriz de rigidez en coordenadas globales del modelo de la estructura.}
    \label{tab:k_global}
\end{table}

Una vez más, al comparar los resultados obtenidos con la matriz de rigidez del modelo de la estructura presentado en \cite{escamilla1995microcomputadores} se encuentra que se trata de los mismos valores. Sin embargo, se debe reordenar una de las dos matrices para encontrar los valores en las mismas posiciones de la matriz. \\

Obtenida la matriz de rigidez de la estructura, se procede a imponer las condiciones de apoyo del modelo de la estructura. Esto se realiza modificando la estructura, tal como se mostró en el capítulo \ref{chap:metodologia}. En la tabla \ref{tab:k_global_support_applied} se presenta dicho resultado. \\

\begin{table}[ht]
    \centering
    \begin{blockarray}{ccccccccccccc}
        & \textit{0} & \textit{1} & \textit{2} & \textit{3} & \textit{4} & \textit{5} & \textit{6} & \textit{7} & \textit{8} & \textit{9} & \textit{10} & \textit{11} \\
        \begin{block}{c[cccccccccccc]}
            \textit{0} & 1 & 0 & 0 & 0 & 0 & 0 & 0 & 0 & 0 & 0 & 0 & 0 \\
            \textit{1} & 0 & 1 & 0 & 0 & 0 & 0 & 0 & 0 & 0 & 0 & 0 & 0 \\
            \textit{2} & 0 & 0 & 1 & 0 & 0 & 0 & 0 & 0 & 0 & 0 & 0 & 0 \\
            \textit{3} & 0 & 0 & 0 & 59568 & 0 & 0 & -39168 & 29376 & 0 & -20400 & 0 & 0 \\
            \textit{4} & 0 & 0 & 0 & 0 & 1 & 0 & 0 & 0 & 0 & 0 & 0 & 0 \\
            \textit{5} & 0 & 0 & 0 & 0 & 0 & 1 & 0 & 0 & 0 & 0 & 0 & 0 \\
            \textit{6} & 0 & 0 & 0 & -39168 & 0 & 0 & 65280 & -9792 & 0 & 0 & 0 & 0 \\
            \textit{7} & 0 & 0 & 0 & 29376 & 0 & 0 & -9792 & 57120 & 0 & 0 & -20400 & 0 \\
            \textit{8} & 0 & 0 & 0 & 0 & 0 & 0 & 0 & 0 & 1 & 0 & 0 & 0 \\
            \textit{9} & 0 & 0 & 0 & -20400 & 0 & 0 & 0 & 0 & 0 & 40800 & 0 & 0 \\
            \textit{10} & 0 & 0 & 0 & 0 & 0 & 0 & 0 & -20400 & 0 & 0 & 20400 & 0 \\
            \textit{11} & 0 & 0 & 0 & 0 & 0 & 0 & 0 & 0 & 0 & 0 & 0 & 1 \\
        \end{block}
    \end{blockarray} \si[per-mode=symbol]{\tonne\per\meter}
    \caption{Matriz de rigidez en coordenadas globales del modelo de la estructura con las condiciones de apoyo impuestas.}
    \label{tab:k_global_support_applied}
\end{table}

\subsubsection{Vector de fuerzas}

Así como se deben encontrar las matrices de rigidez de cada uno de los elementos del modelo de la estructura para posteriormente \textit{ensamblarlas}, se deben encontrar las acciones en los nodos de los elementos para cada patrón de carga. \\

El usuario puede indagar por dicho vector, para el patrón de carga \textit{point loads} mediante la instrucción \texttt{structure.load\_patterns['point loads'].get\_f()}. Con dicha instrucción, el usuario obtiene los datos que se muestran en la tabla \ref{tab:f_global}. \\

Obtenido el vector de fuerzas para dicho patrón de carga, se imponen las condiciones de apoyo del modelo de la estructura. Debido a que los desplazamiento en los apoyos son iguales a cero, el vector de fuerzas en los nodos no varia. \\

\begin{table}[ht]
    \centering
    \begin{blockarray}{cccccccccccc}
        \textit{0} & \textit{1} & \textit{2} & \textit{3} & \textit{4} & \textit{5} & \textit{6} & \textit{7} & \textit{8} & \textit{9} & \textit{10} & \textit{11} \\
        \begin{block}{\{cccccccccccc\}}
            0 & 0 & 0 & 0 & 0 & 0 & 4 & 3 & 0 & 0 & -20 & 0 \\
        \end{block}
    \end{blockarray} $ ^{t} $ \si{\tonne}
    \caption{Vector de fuerzas de los nodos del modelo de la estructura para el patrón de carga \textit{point loads}.}
    \label{tab:f_global}
\end{table}

\subsubsection{Vector de desplazamientos}
Al imponerse las condiciones de apoyo a la matriz de rigidez del modelo de la estructura, se determinan los desplazamientos de los nodos de la estructura para cada uno de los patrones de carga, donde se ha calculado el vector de fueras en los nodos de cada patrón de cargas y se le han impuesto las condiciones de apoyo. \\

En la tabla \ref{tab:u_point_loads} se presenta el vector de desplazamientos en los nodos del modelo de la estructura para el patrón de carga \textit{point loads}, los cuales son iguales a los presentados en \cite{escamilla1995microcomputadores}.

\begin{table}[ht]
    \centering
    \begin{blockarray}{cccccccccccc}
        \textit{0} & \textit{1} & \textit{2} & \textit{3} & \textit{4} & \textit{5} & \textit{6} & \textit{7} & \textit{8} & \textit{9} & \textit{10} & \textit{11} \\
        \begin{block}{\{cccccccccccc\}}
            0 & 0 & 0 & \num{1.307} & 0 & 0 & \num{0.645} & \num{-1.337} & 0 & \num{0.654} & \num{-2.317} & 0 \\
        \end{block} 
    \end{blockarray} $ ^{t} $ \SI{1e-3}{\meter} 
    \caption{Vector de desplazamientos de los nodos del modelo de la estructura para el patrón de carga \textit{point loads}.}
    \label{tab:u_point_loads}
\end{table}

\subsubsection{Vector de reacciones}
Una vez se encuentran los desplazamientos en los nodos del modelo de la estructura, para cada uno de los patrones de carga, se procede a encontrar las fuerzas en los nodos producto de dichos desplazamientos. \\

En la tabla \ref{tab:f_point_loads_solution} se presenta el vector de fuerzas en los nodos del modelo de la estructura para el patrón de carga \textit{point loads}, los cuales son iguales a los presentados en \cite{escamilla1995microcomputadores}.

\begin{table}[ht]
    \centering
    \begin{blockarray}{cccccccccccc}
        \textit{0} & \textit{1} & \textit{2} & \textit{3} & \textit{4} & \textit{5} & \textit{6} & \textit{7} & \textit{8} & \textit{9} & \textit{10} & \textit{11} \\
        \begin{block}{\{cccccccccccc\}}
            -4 & 7 & 0 & 0 & 10 & 0 & 4 & 3 & 0 & 0 & -20 & 0 \\
        \end{block} 
    \end{blockarray} $ ^{t} $ \si{\tonne} 
    \caption{Vector de fuerzas de los nodos del modelo de la estructura para el patrón de carga \textit{point loads}.}
    \label{tab:f_point_loads_solution}
\end{table}

\subsubsection{Procesamiento de los resultados}

Hasta aquí, se ha solucionado el modelo de la estructura sometida al patrón de carga \textit{point loads}. Con los resultados almacenados, se pueden determinar otros resultados que son interesantes. Para el caso concreto del modelo objeto de estudio en esta sección, se puede querer determinar las fuerzas internas de los elementos o las defomaciones en los mismos. \\

A modo de ejemplo, en la tabla \ref{tab:internal_forces} se presenta las fuerzas internas de cada uno de los elementos del modelo de la estructura sometidos al patrón de carga \textit{point loads}, los cuales son prácticamente iguales a los presentados en \cite{escamilla1995microcomputadores}.
\begin{table}[h]
    \centering
    \begin{tabular}{|c|c|}
        \hline
        Elemento & Fuerza [\si{\tonne}] \\
        \hline
        1-3 & -11.667 \\
        \hline
        1-4 & 13.333 \\
        \hline
        3-2 & -16.667 \\
        \hline
        4-2 & 13.333 \\
        \hline
        4-3 & 20 \\
        \hline
    \end{tabular}
    \caption{Fuerzas internas de los elementos del modelo de la estructura para el patrón de carga \textit{point loads}.}
    \label{tab:internal_forces}
\end{table}

\subsection{Pórtico tridimensional}

En el libro \textit{Microcomputadores en Ingeniería Estructural}, del ingeniero \textit{Jairo Uribe Escamilla} (\cite{escamilla1995microcomputadores}), se presenta el ejemplo y la solución del modelo de un pórtico tridimensional. A continuación se presenta la solución de dicho ejemplo, el cual se resuelve usando la librería \texttt{pyFEM}. \\

\textit{\textbf{Ejemplo 7.6 -} Resuelva matricialmente el pórtico de la figura.} \\

\begin{figure}[h]
    \setcoords{0}{90}[.5][.5][0.5][225]
    \centering
    \begin{tikzpicture}[coords]
        % Scaling
        \dscaling{1}{3};
        \dscaling{2}{3};
        \dscaling{3}{1};
        \dscaling{4}{1.5};
        \dscaling{5}{2};
        
        % Points
        \dpoint{o}{0}{0}{0};
        
        \dpoint{1}{0}{3}{3};
        \dpoint{2}{5}{3}{3};
        \dpoint{3}{0}{0}{3};
        \dpoint{4}{0}{3}{0};
        
        \dpoint{23}{5}{0}{3};
        \dpoint{24}{5}{3}{0};
        \dpoint{234}{5}{0}{0};
        
        \dpoint{a}{2.5}{3}{3};
        \dpoint{b}{0}{3}{1.5};
        \dpoint{c}{0}{1.5}{3};
        
        % Beams
        \dbeam{1}{1}{2};
        \dbeam{1}{3}{1};
        \dbeam{1}{4}{1};
        
        \dbeam{3}{3}{o};
        \dbeam{3}{4}{o};
        \dbeam{3}{2}{23};
        \dbeam{3}{3}{23};
        \dbeam{3}{2}{24};
        \dbeam{3}{4}{24};
        \dbeam{3}{o}{234};
        \dbeam{3}{24}{234};
        \dbeam{3}{23}{234};
        
        % Global coordinate system
        \daxis{1}{o};
        
        % Supports
        \dsupport{2}{2}[yz];
        \dsupport{2}{3}[xz];
        \dsupport{2}{4}[xy];
        
        % Loads
        \dlineload{1}{xy}{1}{2}[.5][.5];
        \dlineload{1}{yz}{1}{4}[1][1];
        
        % Dimensions
        \ddimensioning{xy}[0]{3}{2}{-8}[$ \SI{5}{\metre} $][3];
        \ddimensioning{yz}[0]{3}{1}{11}[$ \SI{3}{\metre} $][3];
        \ddimensioning{zx}[0]{1}{4}{-12}[$ \SI{3}{\metre} $][3];
        
        % Labels
        \dnotation{1}{1}{ $ 1 $ }[above left];
        \dnotation{1}{2}{ $ 2 $ }[below right];
        \dnotation{1}{3}{ $ 3 $ }[above left];
        \dnotation{1}{4}{ $ 4 $ }[below right];
        
        \dnotation{1}{a}{$ \SI[per-mode=symbol]{2.4}{\tonne\per\metre} $}[above right=10mm and 0mm];
        \dnotation{1}{4}{$ \SI[per-mode=symbol]{3.5}{\tonne\per\metre} $}[above left=8mm and 0mm];
        
        \dnotation{1}{a}{\textit{(0.30, 0.40)}}[below right];
        \dnotation{1}{b}{\textit{(0.40, 0.25)}}[right];
        \dnotation{1}{c}{\textit{(0.30, 0.40)}}[right];
        
        \dnotation{1}{23}{$ E=220\:\mathrm{t/cm^2}$}[above right=5mm and 10mm];
        \dnotation{1}{23}{$ G=85\:\mathrm{t/cm^2}$}[above right=0mm and 10mm];
    % here we construct our structure
    \end{tikzpicture}
    \caption{Pórtico tridimensional del \textit{Ejemplo 7.6} de \cite{escamilla1995microcomputadores}.}
    \label{fig:segundo_punto}
\end{figure}

\textit{\textbf{Solución -}} La figura a la cual hace referencia el ejemplo 7.6 se presenta en la figura \ref{fig:segundo_punto}. En ésta se presenta la geometría del modelo de la estructura, así como las propiedades del material, las cargas a las que están sometidos y las condiciones de apoyo. \\

Para solucionar el ejemplo anterior usando la librería \texttt{pyFEM}, se ingresa la información necesaria para describir completamente el modelo de la estructura mediante instrucciones en el lenguaje de programación \texttt{python}, las cuales se almacenan en un archivo de texto plano. Dicha información consiste en:
\begin{inparaenum}[$ (1) $]
    \item los materiales, 
    \item las secciones transversales, 
    \item los nodos, 
    \item los elementos tipo pórtico, 
    \item los apoyos y
    \item las cargas.
\end{inparaenum}

En el algoritmo \ref{alg:portico_tridimensional} se presenta las instrucciones que debe recibir \texttt{pyFEM} para solucionar el modelo de la estructura. Las instrucciones consisten en crear un nuevo objeto tipo \texttt{Structure}, al cual se le ha dado el nombre de \textit{structure} y, seguidamente, se agregan los materiales, las secciones, los nodos, los elementos tipo pórtico, los apoyos, los patrones de carga y las cargas en los elementos. Una vez los datos del modelo de la estructura han sido ingresado, se puede solucionar el modelo mediante la instrucción \texttt{structure.solve()}.

\begin{lstlisting}[language=Python,caption=Ingreso de los datos del modelo de la estructura a \texttt{pyFEM}.,label=alg:portico_tridimensional, frame=single]
# structure
structure = Structure()

# add material
structure.materials.add('material1', 220e4, 85e4)

# add sections
structure.sections.add('section1', 'material1', 0.12, 9e-4, 1.6e-3, 1.944e-3)
structure.sections.add('section2', 'material1', 0.10, 1.333e-3, 5.208e-4, 1.2734e-3)

# add nodes
structure.nodes.add('1', 0, 3, 3)
structure.nodes.add('2', 5, 3, 3)
structure.nodes.add('3', 0, 0, 3)
structure.nodes.add('4', 0, 3, 0)

# add frames
structure.frames.add('1-2', '1', '2', 'section1')
structure.frames.add('3-1', '3', '1', 'section1')
structure.frames.add('4-1', '4', '1', 'section2')

# add supports
structure.supports.add('2', *6 * (True,))
structure.supports.add('3', *6 * (True,))
structure.supports.add('4', *6 * (True,))

# add load pattern
structure.load_patterns.add("distributed loads")

# add distributed loads
structure.load_patterns["distributed loads"].distributed_loads.add('1-2', 0, -2.4, 0)
structure.load_patterns["distributed loads"].distributed_loads.add('4-1', 0, -3.5, 0)

# solve
structure.solve()
\end{lstlisting}

Cuando se ejecuta la instrucción \texttt{structure.solve()}, \texttt{pyFEM} comienza a solucionar el modelo de la estructura con base en la información ingresada. A continuación se realiza la comparación entre los resultados obtenidos contra los resultados presentados en \cite{escamilla1995microcomputadores}.

\subsubsection{Desplazamientos de los nodos}

En la tabla \ref{tab:nodes_displacement} se presenta el desplazamientos de los nodos del modelo de la estructura para el patrón de carga \textit{distributed loads}, los cuales son iguales a los presentados en \cite{escamilla1995microcomputadores}.
\begin{table}[h]
    \centering
    \begin{tabular}{|c|c|c|c|c|}
        \hline
        Nodo & 1 & 2 & 3 & 4 \\
        \hline
        $ \mathrm{u_x} $ [\si{\meter}] & \num{2.687e-5} & 0 & 0 & 0 \\
        \hline
        $ \mathrm{u_y} $ [\si{\meter}] & \num{-1.157e-4} & 0 & 0 & 0 \\
        \hline
        $ \mathrm{u_z} $ [\si{\meter}] & \num{-1.001e-5} & 0 & 0 & 0 \\
        \hline
        $ \mathrm{r_x} $ [\si{\radian}] & \num{-5.669e-4} & 0 & 0 & 0 \\
        \hline
        $ \mathrm{r_y} $ [\si{\radian}] & \num{7.905e-6} & 0 & 0 & 0 \\
        \hline
        $ \mathrm{r_z} $ [\si{\radian}] & \num{-6.309e-4} & 0 & 0 & 0 \\
        \hline
    \end{tabular}
    \caption{Desplazamientos en los nodos del modelo para el patrón de carga \textit{distributed loads}.}
    \label{tab:nodes_displacement}
\end{table}

\subsubsection{Reacciones de los apoyos}

En la tabla \ref{tab:support_reactions} se presenta las reacciones de los apoyos del modelo de la estructura para el patrón de carga \textit{distributed loads}, las cuales son practicamente iguales a las presentadas en \cite{escamilla1995microcomputadores}.

\begin{table}[h]
    \centering
    \begin{tabular}{|c|c|c|c|}
        \hline
        Nodo & 2 & 3 & 4 \\
        \hline
        $ \mathrm{F_x} $ [\si{\tonne}] & \num{-1.419} & \num{1.438} & \num{-0.02} \\
        \hline
        $ \mathrm{F_y} $ [\si{\tonne}] & \num{6.572} & \num{1.019e1} & \num{5.742} \\
        \hline
        $ \mathrm{F_z} $ [\si{\tonne}] & \num{5.658e-3} & \num{-7.394e-1} & \num{0.734} \\
        \hline
        $ \mathrm{T} $ [\si{\tonne\meter}] & \num{1.873e-1} & \num{-7.350e-1} & \num{-3.146} \\
        \hline
        $ \mathrm{M_y} $ [\si{\tonne\meter}] & \num{1.102e-2} & \num{-4.354e-3} & \num{-0.037} \\
        \hline
        $ \mathrm{M_z} $ [\si{\tonne\meter}] & \num{-5.986} & \num{-1.417} & \num{0.228} \\
        \hline
    \end{tabular}
    \caption{Reacciones de los apoyos del modelo para el patrón de carga \textit{distributed loads}.}
    \label{tab:support_reactions}
\end{table}



% hacer la comparación entre la respuesta obtenida con la presentada en dichos libros, para el primer caso, o para comparar el tiempo de cálculo requerido entre las herramientas objeto de este trabajo con programas comerciales, para el segundo caso.  \\
% Dichos modelos corresponden, bien sea, a 
% \begin{inparaenum}[ $(1)$]
%     \item ejemplos de los libros de las referencias bibliográficas o
%     \item planteados por el autor,
% \end{inparaenum}

% \chapter{Análisis de resultados}
\label{chap:analisis_resultados}

\section{Generalidades de los programas}

\subsection{Lenguaje de programación}

La librería Panda3D permitió escribir la interfaz y el programa de análisis en el mismo lenguaje de programación: \texttt{python}. El uso de un mismo lenguaje de programación para todas las partes de los programas permite un desarrollo y evolución del mismo con mayor fluidez y orden. La colaboración de otros desarrolladores será más controlada y fácil de implementar si se mantiene dicho lenguaje y no se mezclan otros lenguajes y paradigmas. 

El lenguaje de programación \texttt{python} es una herramienta ...

\subsection{Portabilidad}

Los dos programas implementados (\textit{StressUN} y la librería \textit{pyFEM}) se probaron constantemente en las plataformas Windows y Linux. Al finalizar, se hizo una prueba en un PC con sistema operativo MacOS. Las pruebas realizadas en los tres sistemas operativos consistieron en resolver los ejemplos que se presentan con detalle en la sección \ref{libreriaPyFem}. A continuación se presentan algunos aspectos que se consideraron para comparar el funcionamiento de los programas en los sistemas operativos seleccionados.

\begin{table}[htbp]
    \centering
    \begin{tabular}{c|p{6cm}}
        \hline 
        SO & Librerías \\
        \hline 
        Windows & Panda3D, numpy, scipy \\
        MacOS & Panda3D, numpy, scipy \\
        Linux & Panda3D, numpy, scipy \\
        \hline 
    \end{tabular}
    \caption{Librerías necesarias}
    \label{tab:my_label}
\end{table}

\begin{table}[htbp]
    \centering
    \begin{tabular}{c|ccc}
        \hline 
        Ejemplo & \multicolumn{3}{c}{Tiempo ejecución} \\
         & Windows & MacOS & Linux \\ 
         \hline 
        Cercha01 & 5.4 & 5.3 & 5.6 \\ 
        Cercha02 & 5.4 & 5.3 & 5.6 \\ 
        Cercha03 & 5.4 & 5.3 & 5.6 \\
        Portico01 & 5.4 & 5.3 & 5.6 \\
        Portico02 & 5.4 & 5.3 & 5.6 \\
        Portico03 & 5.4 & 5.3 & 5.6 \\
        \hline 
    \end{tabular}
    \caption{Tiempo de ejecución de la librería \textit{pyFEM} en diferentes sistemas operativos}
    \label{tab:my_label}
\end{table}

Los tiempos de ejecución del programa son similares debido a que \texttt{python} es un lenguaje interpretado. Se ejecuta sobre una máquina virtual dentro del sistema operativo, y por eso su velocidad de procesamiento no depende tanto del procesador del PC, sino de la versión de \texttt{python}, de sus librerías y del grado de optimización con el que se haya escrito el programa.


% ------------------------------------------------------------------------------
\section{Interfaz gráfica \textit{StressUN}}

La iterfaz gráfica \textit{StressUN} ha sido diseñada para cubrir todas las necesidades de estudiantes, profesores y profesionales en el campo del análisis estructural: modelado geométrico, definición de propiedades, definición de cargas y casos de carga, definición de restricciones, transferencia de datos al software de análisis, visualización del modelo de análisis y visualización de resultados numéricos.

\subsection{Generalidades de la interfaz}

\subsubsection{Controles del teclado}

\subsubsection{Interacción con el \textit{mouse}}


\subsection{Preproceso}

\subsubsection{Definición de materiales y propiedades geométricas}

\subsubsection{Definición y visualización de nodos y elementos}

\subsubsection{Definición y visualización de restricciones}

\subsubsection{Definición y visualización de cargas y casos de carga}


\subsection{Posproceso (resultados)}

\subsubsection{Visualización de desplazamientos}

\subsubsection{Visualización de reacciones}

\subsubsection{Exportación de resultados en tablas}

\subsubsection{Visualización de acciones internas: axial, cortante, momento}

\subsubsection{Visualización simultánea de diferentes resultados}

\subsubsection{Visualización de variables y parámetros del proceso de análisis}

% ------------------------------------------------------------------------------
\section{Funcionamiento de la librería \textit{pyFEM}} \label{sec:libreriaPyFem}

Se verificó el correcto funcionamiento de la librería de análisis estructural \textit{pyFEM}, mediante la comparación de los resultados obtenidos en este trabajo y por otros autores. A continuación se evidencia que las respuestas obtenidas son prácticamente iguales a las presentados en los ejemplos resueltos en la literatura consultada. En algunos casos, se obtienen respuestas más exactas que las que reportan los autores principales de los ejemplos resueltos. A continuación se presenta una tabla indicando el error obtenido en cada ejemplo.

\begin{table}[htbp]
    \centering
    \begin{tabular}{c|ccc|ccc}
        \hline 
        Nodo & $\varepsilon (d_x)$ & $\varepsilon (d_y)$ & $\varepsilon (d_z)$ & $\varepsilon (R_x)$ & $\varepsilon (R_y)$ & $\varepsilon (R_z)$ \\
        \hline 
        1 & 0.01\% & 0.012\% & 0.003\% & 0.0051\% & 0.024\% & 0.102\% \\
        2 & 0.01\% & 0.012\% & 0.003\% & 0.0051\% & 0.024\% & 0.102\% \\
        3 & 0.01\% & 0.012\% & 0.003\% & 0.0051\% & 0.024\% & 0.102\% \\
        4 & 0.01\% & 0.012\% & 0.003\% & 0.0051\% & 0.024\% & 0.102\% \\
        \vdots & & & & & & \\
        \hline 
    \end{tabular}
    \caption{Errores Cercha01}
    \label{tab:errorCercha01}
\end{table}

\begin{table}[htbp]
    \centering
    \begin{tabular}{c|ccc|ccc}
        \hline 
        Nodo & $\varepsilon (d_x)$ & $\varepsilon (d_y)$ & $\varepsilon (d_z)$ & $\varepsilon (R_x)$ & $\varepsilon (R_y)$ & $\varepsilon (R_z)$ \\
        \hline 
        1 & 0.01\% & 0.012\% & 0.003\% & 0.0051\% & 0.024\% & 0.102\% \\
        2 & 0.01\% & 0.012\% & 0.003\% & 0.0051\% & 0.024\% & 0.102\% \\
        3 & 0.01\% & 0.012\% & 0.003\% & 0.0051\% & 0.024\% & 0.102\% \\
        4 & 0.01\% & 0.012\% & 0.003\% & 0.0051\% & 0.024\% & 0.102\% \\
        \vdots & & & & & & \\
        \hline 
    \end{tabular}
    \caption{Errores Cercha02}
    \label{tab:errorCercha02}
\end{table}

\begin{table}[htbp]
    \centering
    \begin{tabular}{c|ccc|ccc}
        \hline 
        Nodo & $\varepsilon (d_x)$ & $\varepsilon (d_y)$ & $\varepsilon (d_z)$ & $\varepsilon (R_x)$ & $\varepsilon (R_y)$ & $\varepsilon (R_z)$ \\
        \hline 
        1 & 0.01\% & 0.012\% & 0.003\% & 0.0051\% & 0.024\% & 0.102\% \\
        2 & 0.01\% & 0.012\% & 0.003\% & 0.0051\% & 0.024\% & 0.102\% \\
        3 & 0.01\% & 0.012\% & 0.003\% & 0.0051\% & 0.024\% & 0.102\% \\
        4 & 0.01\% & 0.012\% & 0.003\% & 0.0051\% & 0.024\% & 0.102\% \\
        \vdots & & & & & & \\
        \hline 
    \end{tabular}
    \caption{Errores Portico01}
    \label{tab:errorPortico01}
\end{table}

\begin{table}[htbp]
    \centering
    \begin{tabular}{c|ccc|ccc}
        \hline 
        Nodo & $\varepsilon (d_x)$ & $\varepsilon (d_y)$ & $\varepsilon (d_z)$ & $\varepsilon (R_x)$ & $\varepsilon (R_y)$ & $\varepsilon (R_z)$ \\
        \hline 
        1 & 0.01\% & 0.012\% & 0.003\% & 0.0051\% & 0.024\% & 0.102\% \\
        2 & 0.01\% & 0.012\% & 0.003\% & 0.0051\% & 0.024\% & 0.102\% \\
        3 & 0.01\% & 0.012\% & 0.003\% & 0.0051\% & 0.024\% & 0.102\% \\
        4 & 0.01\% & 0.012\% & 0.003\% & 0.0051\% & 0.024\% & 0.102\% \\
        \vdots & & & & & & \\
        \hline 
    \end{tabular}
    \caption{Errores Portico02}
    \label{tab:errorPortico02}
\end{table}

La tabla \ref{tab:errorPromedio} muestra los errores promedio obtenidos para cada ejemplo resuelto. Se puede ver que los resultados obtenidos con la librería implementada no tienen errores significativos. Vale la pena aclarar que algunos de esos errores se deben a que los resultados que presentan los autores principales de la literatura consultada no son exactos, mientras que los obtenidos con \textit{pyFEM} tienen mayor precisión.

\begin{table}[htbp]
    \centering
    \begin{tabular}{c|ccc|ccc}
        \hline 
        Ejemplo & $\varepsilon (d_x)$ & $\varepsilon (d_y)$ & $\varepsilon (d_z)$ & $\varepsilon (R_x)$ & $\varepsilon (R_y)$ & $\varepsilon (R_z)$ \\
        \hline 
        Cercha01 & 0.01\% & 0.012\% & 0.003\% & 0.0051\% & 0.024\% & 0.102\% \\
        Cercha02 & 0.01\% & 0.012\% & 0.003\% & 0.0051\% & 0.024\% & 0.102\% \\
        Portico01 & 0.01\% & 0.012\% & 0.003\% & 0.0051\% & 0.024\% & 0.102\% \\
        Portico02 & 0.01\% & 0.012\% & 0.003\% & 0.0051\% & 0.024\% & 0.102\% \\
        \vdots & & & & & & \\
        \hline 
    \end{tabular}
    \caption{Errores promedio}
    \label{tab:errorPromedio}
\end{table}
% \chapter{Conclusiones y recomendaciones}
En los capítulos anteriores se expuso el desarrollo y la correspondiente aplicación del programa de computador \textit{StressUN}, cuya finalidad es resolver modelos de estructuras tridimensionales de elementos rectos sometidos a cargas estáticas, es decir, calcular:
\begin{inparaenum} [$ (1) $]
    \item los desplazamientos de los nodos,
    \item las reacciones en los apoyos y 
    \item las fuerzas y los desplazamientos internos de los elementos rectos
\end{inparaenum}
de estructuras tridimensionales sometidas a cargas estáticas.\\

\textit{StressUN} se divide en tres partes: el \textit{preprocesador}, el \textit{procesador} y el \textit{posprocesador}. El procesador se encarga de solucionar el modelo de la estructura,  mientras que el preprocesador y el posprocesador, que, en su conjunto, consisten en una interfaz gráfica y en un ambiente virtual tridimensional, le permiten al usuario ingresar la información del modelo de manera interactiva, visualizarla y acceder a los resultados del análisis. \\

Durante el desarrollo de \textit{StressUN} se creó la librería \textit{pyFEM}, la cual provee las herramientas necesarias para resolver modelos de estructuras objeto de este trabajo. De manera que
el usuario también puede estudiar dichas estructuras mediante los diferentes comandos de la  librería, ya sea generando un archivo de instrucciones en el lenguaje de programación \textit{python}, el cual se puede modificar y ejecutar cuantas veces se desee, o haciendo uso del modo interactivo de dicho lenguaje de programación, en el cual cada línea de comandos se ejecuta después de oprimir la tecla \textit{retorno}. \\

Así mismo, el usuario puede emplear sus propias rutinas de programación para resolver modelos de estructuras en conjunto con el preprocesador y el posprocesador de \textit{StressUN}, de tal manera que tiene una herramienta para visualizar los datos de entrada y de salida del modelo para su programa de computador. \\

Debido a la facilidad con la que cuenta el usuario para ingresar la información del modelo de su estructura (a través de la interfaz gráfica interactiva o mediante líneas de instrucciones de \textit{python}), sumada a las diferentes alternativas con las que puede obtener los resultados del análisis (de manera visual en el entorno tridimensional, a través de tablas con formato predefinido, o manipualandolos por su cuenta mediante instrucciones en \textit{python}), las herramientas de computación desarrolladas en este trabajo son del agrado de diferentes tipos de usuarios. \\

Para obtener los resultados de la solución de un modelo de una estructura tridimensional de elementos rectos sometidos a cargas estáticas, el usuario debe sortear una serie de etapas, indiferentemente si decide usar el programa de computador \textit{StressUN} o la librería \textit{pyFEM}. Dichas etapas consisten en describir el modelo de la estructura, la que se conoce como \textit{preproceso}. Después, ejecutar el análisis del mismo, etapa que se conoce como \textit{proceso}, en donde se soluciona el modelo, siempre y cuando no sea inestable. Una vez se obtienen los resultados, el usuario puede acceder a éstos, etapa que se conoce como \textit{posproceso}, de manera que él pueda realizar su conclusiones, donde puede optar por realizar cambios al modelo para reanalizarlo o dar por terminado el análisis. \\

Los pasos descritos en el párrafo anterior se realizaron para diferentes tipos de estructuras, las cuales se presentan en algunos libros de la literatura clásica o fueron planteadas por el autor para comparar los resultados del programa \textit{StressUN} con programas comerciales. Los resultados de dichas estructuras se presentaron en el capítulo \textit{\nameref{chap:analisis_resultados}}, con lo cual se concluye que los resultados presentados por el programa \textit{StressUN} son prácticamente iguales a las respuestas presentadas en los textos y similares a la respuesta obtenida con los programas comerciales. Así mismo, la respuesta del programa objeto de esta tesis se obtuvo en un tiempo menor al tiempo utilizado por los programas comerciales. \\

En la manufactura del programa de computador \textit{StressUN} se usaron varias herramientas, tanto informáticas como matemáticas, que no son propias de la literatura clásica, como son la \textit{programación orientada a objetos} y los \textit{cuaterniones}, las cuales permiten que \textit{StressUN} sea más versátil que otras implementaciones del análisis matricial. \\

El paradigma de programación orientada a objetos, a diferencia de la programación de computadores clásica, más conocida como \textit{programación estructurada}, permite crear \textit{multiples instancias}, \textit{personalizar vía herencia} y \textit{sobrecarga de operadores}, herramientas que se usaron a lo largo de todo el código del programa de computador. \\

Esto permitió que el usuario creara nuevos \textit{objetos}, que se reusara código para evitar su duplicidad y objetos con comportamientos similares a objetos proprios de \textit{python}. Lo dicho anteriormente se comprueba al darse cuenta que se creaban nuevos objetos al ejecutar líneas de código como \textit{structure.nodes.add('0', 0, 0, 0)}, que se reusaba código al crear la \textit{clase} \textit{Frame} al utilizar como base la \textit{clase} \textit{Truss}, o que se podia indexar la lista de objetos mediante la instrucción \textit{structure.frames['0']}, por citar algunos ejemplos.  \\

Los cuaterniones son una extensión de los números reales, similar a la de los números complejos. Entre las aplicaciones de los cuaterniones se encuentra la representación de rotaciones en el espacio, por lo cual se usan a menudo en \textit{gráficos por computador} para representar la orientación de un objeto en un espacio tridimensional. Y aun cuando existen otros elementos matemáticos que cumplen con el mismo objetivo, los cuaterniones cuentan con la ventaja que para una misma rotación se tiene una sola representación, a diferencia de los \textit{ángulos de Euler}, y son más compactos, comparado con las matrices. \\

Aunque dichas características son de interés, tal vez la más importante sea la facilidad con la cual se puede representar las rotaciones en el espacio, ya que, a consideración del autor, ésta ha sido la gran limitante de otros trabajos a la hora de tratar problemas tridimensionales, puesto que los métodos clásicos son bastante confusos, tal como se presenta en \cite{escamilla1995microcomputadores}. \\

Lo anteriormente dicho combinado con un sistema de versiones como \textit{git}, que permite llevar el control del desarrollo del programa, permite que este proyecto no se quede en las líneas de esta trabajo sino que futuras generaciones de ingenieros estructurales e ingenieros de computación gráfica continúen con este proyecto. \\

Los aportes inmediatos deberían dirigirse al cálculo de las fuerzas debidas a los efectos sísmicos y al cálculo de los efectos de segundo orden, de manera que este programa de computador se convierta en una herramienta en las oficinas de diseño especializadas en el diseño de edificios. Así mismo, se debería implementar modelos predefinidos de estructuras, para que el usuario, a partir de la personalización de estos, pueda analizar estructuras típicas invirtiendo menor tiempo en la descripción de la misma. \\

A futuro se debe pensar en implementar nuevos elementos finitos, análisis de estructuras con aumentos de carga, análisis de señales de sismos, módulos de diseño que implementen los diferentes códigos nacionales, generador de reportes, entre otros. \\

Finalmente, los ingenieros de computación gráfica deberían realizar sus aportes orientados a la mejor presentación de los datos, a la manipulación de los objetos, todo esto encaminado a hacer la experiencia del usuario más agradable. \\

\addcontentsline{toc}{chapter}{\numberline{}Bibliograf\'{\i}a}
\printbibliography
\end{document}
