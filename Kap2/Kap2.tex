\chapter{Cap\'{\i}tulo 1}
% Los cap\'{\i}tulos son las principales divisiones del documento. En estos, se desarrolla el tema del documento. Cada cap\'{\i}tulo debe corresponder a uno de los temas o aspectos tratados en el documento y por tanto debe llevar un t\'{\i}tulo que indique el contenido del cap\'{\i}tulo.\\

% Los t\'{\i}tulos de los cap\'{\i}tulos deben ser concertados entre el alumno y el director de la tesis  o trabajo de investigaci\'{o}n, teniendo en cuenta los lineamientos que cada unidad acad\'{e}mica brinda. As\'{\i} por ejemplo, en algunas facultades se especifica que cada cap\'{\i}tulo debe corresponder a un art\'{\i}culo cient\'{\i}fico, de tal manera que se pueda publicar posteriormente en una revista.\\

\section{Metodología}
 El programa \textit{StressUN} se desarrollará usando el paradigma de \textit{Programación Orientada a Objetos}, \textit{OOP} (de sus siglas en inglés \textit{object-oriented programming}). El programa se desarrollará en forma modular, de manera que el \textit{preproceso} y el \textit{posproceso} son independientes del \textit{proceso}.\\

Inicialmente se realiza una revisión bibliográfica de la formulación matemática de los métodos matriciales aplicados al análisis estructural y una revisión documental a la librería \textit{frames}.
La revisión bibliográfica se enfocará en el análisis de estructuras tridimensionales de respuesta lineal, mientras que la revisión documental se orientará en la interacción del usuario con la \textit{escena} tridimensional.\\

Con base en la revisión bibliográfica y documental se implementarán las \textit{clases} del programa y la interacción entre ellas, a partir de las clases ofrecidas en la librería \textit{frames} y las rutinas presentadas en la bibliografia, siempre teniendo en cuenta que tanto el \textit{preproceso} como el \textit{posproceso} son independientes del \textit{proceso}.\\

Todo el trabajo desarrollado se compilará en un documento final donde los usuarios podrán encontrar la formulación matématica, el código desarrollado y las intrucciones para usar el programa. \\

Finalmente, se presentan ejemplos de estructuras tridimensionales con respuesta lineal sometidas a cargas estáticas y se verifican con la ayuda de programas comerciales.


% \subsection{Subt\'{\i}tulos nivel 3}
% De la cuarta subdivisi\'{o}n en adelante, cada nueva divisi\'{o}n o \'{\i}tem puede ser se\~{n}alada con vi\~{n}etas, conservando el mismo estilo de \'{e}sta, a lo largo de todo el documento.\\

% Las subdivisiones, las vi\~{n}etas y sus textos acompa\~{n}antes deben presentarse sin sangr\'{\i}a y justificados.\\

% \begin{itemize}
% \item En caso que sea necesario utilizar vi\~{n}etas, use este formato (vi\~{n}etas cuadradas).
% \end{itemize}