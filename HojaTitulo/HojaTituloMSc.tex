%\newpage
%\setcounter{page}{1}
\begin{center}
\begin{figure}
\centering%
\epsfig{file=HojaTitulo/EscudoUN,scale=1}%
\end{figure}
\thispagestyle{empty} \vspace*{2.0cm} \textbf{\huge
Programa de computador para resolver modelos de estructuras tridimensionales de elementos rectos}\\[6.0cm]
\Large\textbf{Cristian Danilo Ramirez Vargas}\\[5.0cm]
\small Universidad Nacional de Colombia\\
Facultad de Ingeniería, Departamento de Ingeniería Civil y Agrícola\\
Bogotá D. C., Colombia\\
\the\year\\
\end{center}

\newpage{\pagestyle{empty}\cleardoublepage}

\newpage
\begin{center}
\thispagestyle{empty} \vspace*{0cm} \textbf{\huge
Programa de computador para resolver modelos de estructuras tridimensionales de elementos rectos}\\[2.0cm]
\Large\textbf{Cristian Danilo Ramirez Vargas}\\[3.0cm]
\small Tesis o trabajo de grado presentada(o) como requisito parcial para optar al
t\'{\i}tulo de:\\
\textbf{Magister en Estructuras}\\[2.5cm]
Director(a):\\
Ph. D. Martín Estrada Mejia\\[2.0cm]
L\'{\i}nea de Investigaci\'{o}n:\\
Análisis de estructuras\\
Grupo de Investigaci\'{o}n:\\
Análisis, diseño y materiales - GIES\\[2.5cm]
Universidad Nacional de Colombia\\
Facultad de Ingeniería, Departamento de Ingeniería Civil y Agrícola\\
Bogotá D. C., Colombia\\
\the\year\\
\end{center}

\newpage{\pagestyle{empty}\cleardoublepage}

\newpage
\thispagestyle{empty} \textbf{}\normalsize
\\\\\\%
% \textbf{(Dedicatoria o un lema)}\\[4.0cm]

\begin{flushright}
\begin{minipage}{10cm}
    \noindent
    \small
    Hombres inteligentes gran pensantes\\
    Hemos creado un monstruo\\
    Las bombas radiactivas y nucleares\\
    Que descompondrán la humanidad\\
    Quien totalmente se autodestruirá\\
    
    Ya creador no hay para volver a comenzar\\
    Como dijo la sagrada maldición\\
    El universo en siete días lo creo\\
    
    \emph{Razas de todos los colores\\
    Tomemos una reacción\\
    Potencias monopolizadoras\\
    Analicen esta situación\\
    Países tercermundistas\\
    De brazos no nos crucemos\\
    Acabemos pronto con esto}\\
    
    Futuro nunca habrá\\
    Futuro nunca ha habido\\
    Este en mundo que esta perdido\\
    Dependiendo de un botón\\
    Y de la decisión\\
    De un idealista cabrón\\
    
    La tercera guerra mundial\\
    Será un estruendo nuclear\\
    Donde historiadores no podrán narrarla\\
    Y los humanos no podremos resistirla\\
    
    \emph{Las invenciones científicas\\
    Lejos de liberar de la ignorancia\\
    Y del trabajo envilecedor\\
    Lo aumentan\\
    Y hacen más refinada la servidumbre}\\
    
    ---\emph{La ciencia de la autodestrucción}, La Pestilencia (1989)\\
\end{minipage}
\end{flushright}

\newpage{\pagestyle{empty}\cleardoublepage}

\newpage
\thispagestyle{empty} \textbf{}\normalsize
\\\\\\%
\textbf{\LARGE Agradecimientos}
\addcontentsline{toc}{chapter}{\numberline{}Agradecimientos}\\

No habría podido hacer este trabajo sin el acompañamiento del profesor Martín Estrada. Su conocimiento del mundo de la programación me ayudó en momentos decisivos durante el desarrollo del código. No sé como hacer para agradecerle por su paciencia.\\

Este trabajo también se debe al curso \emph{Computación Visual} del profesor Jean Pierre Charalambos. Su descripción de la \emph{escena} y como trabajar con ella fue la que me permitió hacer \emph{FEM.js}.\\

Y por si esto no fuera suficiente, apliqué el concepto de \emph{cuaternión} en el problema de la rotación de los ejes de referencia tiempo después de haberlo estudiado en una de sus clases, lo que me permitió implementar el método de análisis matricial de manera innovadora.\\

Finalmente, quiero agradecer a la profesora Maritzabel Molina ya que mi entendimiento del método de análisis matricial proviene de su curso \emph{análisis estructural avanzado}. A ella nos debemos muchos ingenieros estructurales.

\newpage{\pagestyle{empty}\cleardoublepage}

\newpage
% \textbf{\LARGE Resumen}
% \addcontentsline{toc}{chapter}{\numberline{}Resumen}\\\\
% El resumen es una presentaci\'{o}n abreviada y precisa (la NTC 1486 de 2008 recomienda revisar la norma ISO 214 de 1976). Se debe usar una extensi\'{o}n m\'{a}xima de 12 renglones. Se recomienda que este resumen sea anal\'{\i}tico, es decir, que sea completo, con informaci\'{o}n cuantitativa y cualitativa, generalmente incluyendo los siguientes aspectos: objetivos, dise\~{n}o, lugar y circunstancias, pacientes (u objetivo del estudio), intervenci\'{o}n, mediciones y principales resultados, y conclusiones. Al final del resumen se deben usar palabras claves tomadas del texto (m\'{\i}nimo 3 y m\'{a}ximo 7 palabras), las cuales permiten la recuperaci\'{o}n de la informaci\'{o}n.\\

% \textbf{\small Palabras clave: (m\'{a}ximo 10 palabras, preferiblemente seleccionadas de las listas internacionales que permitan el indizado cruzado)}.\\

% A continuaci\'{o}n se presentan algunos ejemplos de tesauros que se pueden consultar para asignar las palabras clave, seg\'{u}n el \'{a}rea tem\'{a}tica:\\

% \textbf{Artes}: AAT: Art y Architecture Thesaurus.

% \textbf{Ciencias agropecuarias}: 1) Agrovoc: Multilingual Agricultural Thesaurus - F.A.O. y 2)GEMET: General Multilingual Environmental Thesaurus.

% \textbf{Ciencias sociales y humanas}: 1) Tesauro de la UNESCO y 2) Population Multilingual Thesaurus.

% \textbf{Ciencia y tecnolog\'{\i}a}: 1) Astronomy Thesaurus Index. 2) Life Sciences Thesaurus, 3) Subject Vocabulary, Chemical Abstracts Service y 4) InterWATER: Tesauro de IRC - Centro Internacional de Agua Potable y Saneamiento.

% \textbf{Tecnolog\'{\i}as y ciencias m\'{e}dicas}: 1) MeSH: Medical Subject Headings (National Library of Medicine's USA) y 2) DECS: Descriptores en ciencias de la Salud (Biblioteca Regional de Medicina BIREME-OPS).

% \textbf{Multidisciplinarias}: 1) LEMB - Listas de Encabezamientos de Materia y 2) LCSH- Library of Congress Subject Headings.\\

% Tambi\'{e}n se pueden encontrar listas de temas y palabras claves, consultando las distintas bases de datos disponibles a trav\'{e}s del Portal del Sistema Nacional de Bibliotecas\footnote{ver: www.sinab.unal.edu.co}, en la secci\'{o}n "Recursos bibliogr\'{a}ficos" opci\'{o}n "Bases de datos".\\

% \textbf{\LARGE Abstract}\\\\
% Es el mismo resumen pero traducido al ingl\'{e}s. Se debe usar una extensi\'{o}n m\'{a}xima de 12 renglones. Al final del Abstract se deben traducir las anteriores palabras claves tomadas del texto (m\'{\i}nimo 3 y m\'{a}ximo 7 palabras), llamadas keywords. Es posible incluir el resumen en otro idioma diferente al espa\~{n}ol o al ingl\'{e}s, si se considera como importante dentro del tema tratado en la investigaci\'{o}n, por ejemplo: un trabajo dedicado a problemas ling\"{u}\'{\i}sticos del mandar\'{\i}n seguramente estar\'{\i}a mejor con un resumen en mandar\'{\i}n.\\[2.0cm]
% \textbf{\small Keywords: palabras clave en ingl\'{e}s(m\'{a}ximo 10 palabras, preferiblemente seleccionadas de las listas internacionales que permitan el indizado cruzado)}\\